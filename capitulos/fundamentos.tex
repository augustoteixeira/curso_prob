\chapter{Fundamentos}

A probabilidade moderna se baseia fortemente na Teoria da Medida e supomos durante esse curso que o leitor esteja bem familiarizado com conceitos tais como: Medida de Lebesgue, extensões de medida e teoremas de convergência.
Iremos agora justificar brevemente a escolha da Teoria da Medida para o estudo de probabilidade.

No início da Teoria da Probabilidade, a maioria dos fenômenos estudados apresentava apenas um número finito de resultados possíveis, como por exemplo ao se jogar um dado de seis lados ou sortear uma carta em um baralho.
Em tais casos é desnecessário o uso de ferramentas sofisticadas pra modelar tais situações.
Por exemplo, podemos simplesmente dizer que a probabilidade de se obter cada um dos lados do dado é igual a $1/6$.

Mas digamos por exemplo que queremos um modelo para estudar o volume de chuva em uma cidade durante um ano.
Obviamente, esse volume poderia ser qualquer número real positivo e não podemos simplesmente atribuir valores positivos de probabilidade a cada número real (lembramos que somas não enumeráveis de termos positivos são sempre infinitas).
Mas como podemos continuar nossa modelagem se nem ao menos podemos dizer qual é a probabilidade de chover um determinado volume esse ano, por exemplo $(\pi/19)mm$?

A solução para tal dilema, se baseia no fato de que na verdade nunca estamos interessados no exato resultado do nosso experimento.
Gostaríamos sim de responder perguntas do tipo: qual é a probabilidade de que chova entre zero e $37mm$?
Estamos portanto interessados em atribuir probabilidades não a valoers exatos do experimento, mas a certos conjuntos de possíveis valores.
Chamamos tais conjuntos de \emph{eventos}. \index{evento}

Voltando ao caso do dado de seis lados, poderíamos nos interessar por exemplo pela probabilidade dos seguintes eventos: o lado sorteado foi ímpar ($P(\{1,3,5\}) = 1/2$) ou o lado serteado foi dois ($P(\{2\}) = 1/6$).
E percebemos rapidamente que para eventos disjuntos a probabilidade de sua união é a soma de suas probabilidades (no caso acima, $P(\{1,2,3,5\}) = 1/2 + 1/6 = 2/3$).
Esse caráter aditivo da probabilidade certamente nos remete aos conceitos básicos de Teoria da Medida.
Vamos agora formalizar a discussão acima com mais calma, sob a ótica dessa teoria.

\section{Espaços mensuráveis}

Denotaremos sempre por $\Omega$ o nosso \emph{espaço amostral} \index{espaco@espaço!amostral} (à princípio qualquer conjunto).
Um ponto nesse espaço corresponde por exemplo a um possível resultado do nosso experimento aleatório.

\begin{example} Possíveis exemplos de espaço amostral
  \label{x:espacos_amostrais}
  \begin{enumerate}[\quad a)]
  \item $\Omega_1 = \{1, 2, \dots, 6\}$,
  \item $\Omega_2 = \mathbb{R}_+$,
  \item $\Omega_3 = \{f:[0,1] \to \mathbb{R}; \text{$f$ é contínua}\}$.
  \end{enumerate}
\end{example}
Os exemplos acima poderiam ser usados em modelar por exemplo: o resultado de um dado, o volume anual de chuva em uma cidade e o comportamento ao longo do dia do preço de uma ação na bolsa de valores.

Consideraremos sempre $\Omega$'s equipados com uma \emph{$\sigma$-álgebra} \index{sigma-algebra@$\sigma$-álgebra} denotada por $\mathcal{F}$.
Mais precisamente
\begin{definition}
  Dizemos que $\mathcal{F} \subseteq \mathcal{P}(\Omega)$ é uma $\sigma$-álgebra se
  \begin{enumerate}[\quad a)]
  \item $\Omega \in \mathcal{F}$,
  \item $A \in \mathcal{F}$ implica que $A^c \in \mathcal{F}$ e
  \item se $A_1, A_2, \dots \in \mathcal{F}$, então $\cup_i A_i \in \mathcal{F}$.
  \end{enumerate}
\end{definition}
Nesse caso, dizemos que $(\Omega, \mathcal{F})$ é um \emph{espaço mensurável} \index{epaco@espaço!mensuravel@mensurável} e os elementos $A \in \mathcal{F}$ são chamados de \emph{eventos}. \index{evento}

Se $\mathcal{G} \subseteq \mathcal{P}(\Omega)$ (que chamamos de uma classe ou família), denotamos por $\sigma(\mathcal{G})$ a \emph{$\sigma$-álgebra gerada por $\mathcal{G}$} \index{sigma-algebra@$\sigma$-álgebra!gerada por G@gerada por $\mathcal{G}$}, que é a menor $\sigma$-álgebra contendo $\mathcal{G}$.
Um exemplo importante é dado pela \emph{$\sigma$-álgebra de Borel} \index{sigma-algebra@$\sigma$-álgebra!de borel}, gerada pelos abertos de uma topologia em $\Omega$.

\begin{example} Típicos exemplos de $\sigma$-álgebra correspondentes aos espaços amostrais do Exemplo~\ref{x:espacos_amostrais}
  \begin{enumerate}[\quad a)]
  \item $\mathcal{F}_1 = \mathcal{P}(\Omega_1)$,
  \item $\mathcal{F}_2 = \mathcal{B}([0,1])$ e
  \item $\mathcal{F}_3 = \mathcal{B}(C[0,1])$.
  \end{enumerate}
\end{example}

\begin{example} Alguns eventos de $\mathcal{F}_1, \mathcal{F}_2$ e $\mathcal{F}_3$ acima
  \begin{enumerate}[\quad a)]
  \item $\{\text{$x$ é ímpar}\}, \{1\} \subset \Omega_1$
  \item $[0,1/2], \{0\}, (\mathbb{Q} \cap [0,1]) \subset \Omega_2$ e
  \item $\{f:[0,1] \to \mathbb{R}; f(1) > 0\} \subset \Omega_3$.
  \end{enumerate}
\end{example}

\begin{exercise}
  Mostre que $\{f:[0,1] \to \mathbb{R}; f(t) \geq 0 \text{ para todo $t \in [0,1]$}\} \subset \Omega_3$ é um evento (ou seja, pertence a $\mathcal{F}_3$).
\end{exercise}

\begin{notation}
  Se $Q$ for uma condição qualquer sobre candidatos $\omega \in \Omega$, escreveremos $[\text{$\omega$ satisfaz $Q$}]$ \index{[omega satisfaz Q]@$[\text{$\omega$ satisfaz $Q$}]$} para denotar $\{\omega \in \Omega; \text{ $\omega$ satisfaz $Q$}\}$.
\end{notation}

Por exemplo, $\{f:[0,1] \to \mathbb{R}; f(1) > 0\}$ pode ser escrita simplesmente como $[f(1) > 0]$.

\section{Espaços de probabilidade}

Agora estamos prontos para introduzir o conceito moderno do que é uma probabilidade.

\begin{definition}
  Dado $(\Omega, \mathcal{F})$ espaço mensurável, dizemos que $P:\mathcal{F} \to [0,1]$ é uma \emph{probabilidade} \index{probabilidade} se
  \begin{enumerate}[\quad a)]
  \item $P(\Omega) = 1$ e
  \item sempre que $A_1, A_2, \dots \in \mathcal{F}$ forem disjuntos ($A_i \cap A_j = \varnothing$ se $i \neq j$), temos
    \begin{equation}
      P\big({\mcup\nolimits_i} A_i\big) = \sum_i P(A_i).
    \end{equation}
  \end{enumerate}
\end{definition}

Obviamente, isso nada mais é que uma medida que associa massa um ao espaço todo.

\begin{example} Probabilidades nos espaços do Exemplo~\ref{x:espacos_amostrais}
  \begin{enumerate}[\quad a)]
  \item $P_1(A) = (\#A)/6$ em $(\Omega_1, \mathcal{F}_1)$.
    Ou mais geralmente $P_1'(A) = \sum_{i \in A} p_i$, onde $p_i \geq 0$ e $\sum_i p_i = 1$.
  \item $P_2$ pode ser a medida de Lebesgue em $([0,1], \mathcal{B}([0,1]))$.
    Mais geralmente também podemos ter $P_2'(A) = \int_A \rho(x) \d x$, onde $\rho:[0,1] \to \mathbb{R}_+$, chamada densidade, é tal que $\int_{[0,1]} \rho (x) \d x = 1$.
  \item $P_3 = \delta_{0}$, que atribui o valor um se o evento contém a função identicamente nula ($f \equiv 0$) e zero caso contrário.
  \end{enumerate}
\end{example}
Obviamente o terceiro exemplo é bastante artificial (e inútil).
Mas futuramente, estaremos protos para introduzir medidas bem interessantes no espaço $(\Omega_3, \mathcal{F}_3)$.

\begin{proposition}
  Valem as afirmativas
  \begin{enumerate}[\quad a)]
  \item Se $A \subseteq B$ então $P(A) = P(B) - P(B \setminus A) \leq P(B)$,
  \item A cota da união:
    \begin{equation}
      P\big(\mcup\nolimits_i A_i\big) \leq \smash{\sum\limits_i} P(A_i)
    \end{equation}
  \item e o que chamamos de princípio da \emph{inclusão e exclusão} \index{inclusao e exclusao@inclusão e exclusão}
    \begin{equation}
      P\big(\mcup\nolimits_{i \leq n} A_i\big) = \smash{\sum\limits_{k = 1}^n} (-1)^{k-1} \sum\limits_{1 \leq i_1 < \dots < i_k \leq n} P(A_{i_1} \cap \dots \cap A_{i_k}).
    \end{equation}
  \end{enumerate}
\end{proposition}

\begin{proof}
  $a)$ Como $A \cap (B \setminus A) = \varnothing$, então
    \begin{equation}
      \begin{split}
        P(A \cup (B \setminus A)) & = P(A \cup (B \setminus A) \cup \varnothing \cup \dots)\\
        & = P(A) + P(B \setminus A) + 0 + \dots = P(A) + P(B \setminus
        A).
      \end{split}
    \end{equation}

  $b)$ $P(A \cup B) = P (A \cup (B \setminus A)) = P(A) + P(B \setminus
    A) \leq P(A) + P(B)$.
    Deixamos o caso enumerável como exercício abaixo.

  $c)$ Basta mostrar a validade da equação abaixo e depois integrar com
    respeito a $P$.
    \begin{equation}
      \label{e:indicadora_como_produto}
      \1_A(\omega) = \sum_{k=1}^n (-1)^{k - 1} \sum_{\substack{I \subseteq \{1, \dots, n\}\\|I| = k}} \prod_{i \in I} \1_{A_i}(\omega).
    \end{equation}
    Para tanto, observe que para todo $\omega \in \Omega$,
    \begin{equation}
      (\1_A - \1_{A_1}) \cdot \dots \cdot (\1_A - \1_{A_n})(\omega) = 0.
    \end{equation}
    Logo, expandindo o produto acima obtemos
    \begin{equation}
      \1_A + \sum_{k = 1}^n \sum_{\substack{I \subseteq \{1, \dots, n\}\\|I| = k}} (-1)^k \1_{A_k}(\omega) = 0,
    \end{equation}
    que equivale a \eqref{e:indicadora_como_produto}.
\end{proof}

\begin{exercise}
  Mostre que $P\big(\mcup\nolimits_i A_i\big) \leq \sum_i P(A_i)$ no caso enumerável.
\end{exercise}

\begin{exercise}
  Mostre que
  \begin{equation*}
    \begin{split}
      P\big( \mcup\nolimits_{i=1}^n A_i \big) & \leq \sum\limits_{k = 1}^m (-1)^{k-1} \sum\limits_{1 \leq i_1 < \dots < i_k \leq n} P(A_{i_1} \cap \dots \cap A_{i_k}) \text{ se $m$ é ímpar e}\\
      P\big( \mcup\nolimits_{i=1}^n A_i \big) & \geq \sum\limits_{k = 1}^m (-1)^{k-1} \sum\limits_{1 \leq i_1 < \dots < i_k \leq n} P(A_{i_1} \cap \dots \cap A_{i_k}) \text{ se $m$ é par.}
    \end{split}
  \end{equation*}
\end{exercise}



\begin{exercise}
  Seja $n \geq 1$ um número inteiro e considere $\Omega = \{0, 1\}^n$, o hipercubo de dimensão $n$ (cada $\omega \in \Omega$ pode ser visto como uma função $\omega:\{1, \dots, n\} \to \{0,1\}$).
  Para cada $i \in \{1, \dots, n\}$, definimos o evento $A_i = \{ \omega \in \Omega; \omega(i) = 1 \}$.
  Dadas duas probabilidades $P$ e $P'$ em $(\Omega, \mathcal{P}(\Omega))$, mostre que se $P(B) = P'(B)$ para todos conjuntos $B$ dados por interseções de $A_i$'s, então $P = P'$.
\end{exercise}

\begin{proposition}
  \label{p:prob_continua}
  Toda probabilidade $P$ é contínua, isto é:
  \begin{enumerate}[\quad a)]
  \item Se $A_1 \subseteq A_2 \subseteq \dots \in \mathcal{F}$, então $\lim_n P(A_n) = P(\mcup\nolimits_n A_n)$.
  \item Também, se $A_1 \supseteq A_2 \supseteq \dots \in \mathcal{F}$, temos $\lim_n P(A_n) = P(\mcap\nolimits_n A_n)$.
  \end{enumerate}
\end{proposition}

\begin{proof}
  $a)$ Observe que
  \begin{equation}
    \mcup_{m = 1}^\infty A_n = \mcup_{n = 1}^\infty \Big( A_n \setminus \big( \mcup_{i=1}^{n-1} A_i \big) \Big),
  \end{equation}
  que são disjuntos.
  Logo
  \begin{equation}
    \begin{split}
      P\big(\mcup\nolimits_{n = 1}^\infty A_n\big) & = \sum_{n = 1}^\infty P\Big( A_n \setminus \big(\mcup\nolimits_{i=1}^{n-1} A_i \big) \Big)\\
      & = \lim_n P({\mcup\nolimits_{i = 1}^n} A_i) = \lim_n P(A_n).
    \end{split}
  \end{equation}

  $b)$ A prova é análoga à de 1.
\end{proof}

\begin{lemma}[Borel-Cantelli - primeira parte]
  Sejam $A_1, A_2, \dots \in \mathcal{F}$ satisfazendo $\sum_{i = 1}^\infty P(A_i) < \infty$.
  Então
  \begin{equation}
    P[\text{$A_i$ para infinitos $i$}] := P\big({\mcap\nolimits_{n = 1}^\infty} ({\mcup\nolimits_{i \geq n}} A_i)\big) = 0.
  \end{equation}
\end{lemma}

\begin{proof}
  Estimamos
  \begin{equation}
    P \Big( {\mcap_{n = 1}^\infty} \big({\mcup\nolimits_{i \geq n}} A_i \big) \Big) = \lim_n P \big( {\mcup\nolimits_{i \geq n}} A_i \big) \leq \lim_n {\textstyle\sum\limits_{i \geq n}} P(A_i) = 0.
  \end{equation}
  O que termina a prova do lemma.
\end{proof}

Imagine que jogamos todos os dias em uma loteria e que nossa probabilidade de ganhar no dia $i$ é $p_i$.
Então se $\sum_i p_i < \infty$, sabemos que certamente não ganharemos infinitas vezes.

\section{Sistemas \texorpdfstring{$\lambda$-$\pi$}{lambda-pi}}

Uma importante ferramenta para provar fatos teóricos sobre probabilidades é o Teorema de Dynkin que apresentaremos nessa seção.
Ele trata de classes de eventos que não são necessariamente $\sigma$-álgebras, mas sistemas $\sigma$ ou $\pi$ como definidos abaixo.

\begin{definition}
  Dizemos que uma classe $\mathcal{A} \subseteq \mathcal{P}(\Omega)$ é um $\pi$-sistema \index{pi sistema@$\pi$-sistema} se for fechado por interseções finitas, isto é: para todos $A, B \in \mathcal{A}$ temos $A \cap B \in \mathcal{A}$.
\end{definition}

\begin{definition}
  Dizemos que $\mathcal{A} \subseteq \mathcal{P}(\Omega)$ é um $\lambda$-sistema, \index{lambda sistema@$\lambda$-sistema} se
  \begin{enumerate}[\quad a)]
  \item $\Omega \in \mathcal{A}$,
  \item Sempre que $A \in \mathcal{A}$ temos $A^c \in \mathcal{A}$ e
  \item para $A_1, A_2, \dots \in \mathcal{A}$ disjuntos dois a dois, temos $\cup_i A_i \in \mathcal{A}$.
  \end{enumerate}
\end{definition}

\begin{exercise}
  \label{x:lambda_nao_sigma}
  Dê um exemplo de $\lambda$-sistema que não seja uma $\sigma$-álbebra.
\end{exercise}

Definimos para $\mathcal{A} \subseteq \mathcal{P}(~W)$, o menor $\lambda$-sistema contendo $\mathcal{A}$, ou seja
\begin{equation}
  \lambda(\mathcal{A}) = \bigcap_{\substack{\text{$\mathcal{B}$ $\lambda$-sistema}\\\mathcal{A} \subseteq \mathcal{B}}} \mathcal{B}.
\end{equation}
É fácil ver que $\lambda(\mathcal{A})$ é sempre um $\lambda$-sistema.

\begin{theorem}[Dynkin]
  \index{Teorema!de Dynkin}
  \label{t:dynkin}
  Se $\mathcal{A}$ é um $\pi$-sistema, então $\lambda(\mathcal{A}) = \sigma(\mathcal{A})$.
\end{theorem}

Note pelo Exercício~\ref{x:lambda_nao_sigma} que a hipótese de que $\mathcal{A}$ é um $\pi$-sistema é necessária em geral.

\begin{proof}
  Obviamente, basta mostrar é que $\lambda(\mathcal{A})$ é fechado por uniões não necessariamente disjuntas.
  Na verdade, vamos ver que é suficiente provar que
  \begin{equation}
    \label{e:lambda_is_pi}
    \lambda(\mathcal{A}) \text{ é um $\pi$-sistema}.
  \end{equation}
  De fato, caso isso seja provado teremos que $\lambda(\mathcal{A})$ é fechado por diferenças (pois $A \setminus B = A \cap B^c$).
  Assim, podemos mostrar que $\lambda(\mathcal{A})$ é fechado por uniões enumeráveis, pois se $A_1, A_2, \dots \in \lambda(\mathcal{A})$, definimos $B_n = \cup_{i=1}^n A_i = (\cap_{i=1}^n A_i^c)^c \in \lambda(\mathcal{A})$ e escrevemos
  \begin{equation}
    \mcup_n A_n = \mcup_n \big(A_n \setminus B_{n-1} \big),
  \end{equation}
  que é uma união disjunta de termos em $\lambda(\mathcal{A})$, logo está em $\lambda(\mathcal{A})$.
  Isso mostra que $\lambda(\mathcal{A})$ é uma $\sigma$-álgebra e que de fato é suficiente demonstrar \eqref{e:lambda_is_pi}.

  Vamos primeiramente mostrar que $\lambda(\mathcal{A})$ é fechado por interseções com $\mathcal{A}$.
  Para tanto, definimos $\mathcal{B} = \big\{B \in \lambda(\mathcal{A}); \text{$B \cap A \in \lambda(\mathcal{A})$ para todo $A \in \mathcal{A}$})\big\}$ e veremos que
  \begin{equation}
    \label{e:B_igual_lambda}
    B = \lambda(\mathcal{A}).
  \end{equation}
  Obviamente, $\mathcal{A} \subseteq \mathcal{B}$, pois $\mathcal{A}$ é um $\pi$-sistema.
  Então basta mostrar que $\mathcal{B}$ é um $\lambda$-sistema.
  \begin{enumerate}[\quad a)]
  \item $\Omega$ obviamente pertence a $\mathcal{B}$.
  \item Se $B \in \mathcal{B}$ e $A \in \mathcal{A}$, então $B^c \cap A = A \setminus(B \cap A) = (A^c \cup (B \cap A))^c$.
    Mas como $B \in \mathcal{B}$, $(B \cap A) \in \lambda(\mathcal{A})$ e usando o fato que $\lambda$-sistemas são fechados por complementos e uniões disjuntas, $B^c \cap A \in \lambda(\mathcal{A})$.
    Como isso vale para todo $A \in \mathcal{A}$, temos $B^c \in \mathcal{B}$ por definição.
  \item Se $B_1, B_2, \dots \in \mathcal{B}$ são disjuntos e $A \in \mathcal{A}$, então
    \begin{equation}
      \big(\mcup\nolimits_i B_i \big) \cap A = \mcup_i \big(B_i \cup A\big) \in \lambda(\mathcal{A}),
    \end{equation}
    pois a união acima é disjunta.
    Logo $\mcup_i B_i \in \mathcal{B}$.
  \end{enumerate}
  Isso mostra que $\mathcal{B}$ é um $\lambda$-sistema com $\mathcal{A} \subseteq \mathcal{B} \subseteq \lambda(\mathcal{A})$, mostrando \eqref{e:B_igual_lambda}.

  No próximo passo, definimos $\bar{\mathcal{B}} = \{A \in \lambda(A); \text{$B \cap A \in \lambda(A), \; \forall B \in \lambda(A)$}\}$ e mostraremos que
  \begin{equation}
    \label{e:Bbar_igual_lambda}
    \bar{\mathcal{B}} = \lambda(\mathcal{A}),
  \end{equation}
  que vai na direção de provar \eqref{e:lambda_is_pi}.

  Primeiramente, observe que $\mathcal{A} \subseteq \bar{\mathcal{B}}$ pois $\mathcal{B} = \lambda(\mathcal{A})$ (veja a definição de $\mathcal{B}$).
  Mostraremos agora que
  \begin{equation}
    \label{e:B_barra_lambda}
    \text{$\bar{\mathcal{B}}$ é um $\lambda$-sistema}.
  \end{equation}
  Para tanto, verificaremos
  \begin{enumerate}[\quad a)]
  \item $\Omega \in \bar{\mathcal{B}}$, que é claro.
  \item Tomando $A \in \bar{\mathcal{B}}$ e $B \in \lambda(\mathcal{A})$, $A^c \cap B = B \setminus (A \cap B) = \big(B^c \cup (A \cap B)\big)^c \in \lambda(\mathcal{A})$, por um argumento análogo ao apresentado para $\mathcal{B}$.
    Logo $A^c \in \bar{\mathcal{B}}$.
  \item Também o caso de uniões disjuntas é bastante análogo ao feito para $\mathcal{B}$.
  \end{enumerate}
  Isso mostra que $\bar{\mathcal{B}}$ é um $\lambda$-sistema com $\mathcal{A} \subseteq \bar{\mathcal{B}} \subseteq \lambda(\mathcal{A})$, estabelecendo \eqref{e:B_barra_lambda}.

  Finalmente mostraremos que
  \begin{equation}
    \label{e:B_barra_pi}
    \text{$\bar{\mathcal{B}}$ é um $\pi$-sistema}.
  \end{equation}
  De fato, sejam $A_1, A_2 \in \bar{\mathcal{B}}$ e $B \in \lambda(A)$.
  Então $(A_1 \cap A_2) \cap B = (A_1 \cap B) \cap A_2 \in \lambda(\mathcal{A})$, donde $A_1 \cap A_2$ pertence a $\bar{\mathcal{B}}$.
  Logo temos por \eqref{e:B_barra_pi} e \eqref{e:B_barra_lambda} que $\lambda(\mathcal{A})$ é um $\pi$-sistema, ou seja \eqref{e:lambda_is_pi}, terminando a prova do teorema.
\end{proof}

\subsection{Igualdade de probabilidades}

\begin{proposition}
  \label{p:P12_equal_pi}
  Se $P_1$ e $P_2$ são probabilidades em $(\Omega, \mathcal{F})$, tais que $P_1(A) = P_2(A)$ para todo $A \in \mathcal{A}$ e $\mathcal{A}$ é um $\pi$-sistema, então $P_1(B) = P_2(B)$ para todo $B \in \lambda(\mathcal{A})$.
\end{proposition}

\begin{proof}
  Seja $\mathcal{B} = \{A \in \mathcal{F}; P_1(A) = P_2(A)\}$.
  É fácil ver que $\mathcal{B}$ é um $\lambda$-sistema.
  Logo $\mathcal{B}$ contém $\lambda(\mathcal{A})$ que é igual a $\sigma(\mathcal{A})$ por Dynkin.
\end{proof}

\begin{corollary}
  \label{c:produto_e_unico}
  Se $P_1$ e $P_2$ são probabilidades em $(\Omega_1 \times \Omega_2, \mathcal{F}_1 \otimes \mathcal{F}_2)$, tais que
  \begin{equation}
    P_1(A_1 \times A_2) = P_2(A_1 \times A_2), \text{ para todos $A_1 \in \mathcal{F}_1$, $A_2 \in \mathcal{F}_2$,}
  \end{equation}
  então $P_1 = P_2$.
\end{corollary}

\begin{proof}
  Obviamente as caixas do tipo $A_1 \times A_2$ formam um $\pi$-sistema que gera $\mathcal{F}_1 \otimes \mathcal{F}_2$ (por definição).
\end{proof}

\begin{example}
  Observe portanto que é importante que $\mathcal{A}$ seja um $\pi$-sistema na Proposição~\ref{p:P12_equal_pi}.
  Imagine por exemplo que $\Omega = \{0,1\}^2$ e $P_1 = \tfrac 14 \sum_{x \in \Omega} \delta_x$ e $P_2 = \tfrac 12 (\delta_{(0,0)} + \delta_{(1,1)})$.
  Nesse caso
  \begin{equation}
    P_1(A) = P_2(A) = 1/2 = P_1(B) = P_2(B),
  \end{equation}
  com $A = \{(0,0), (0,1)\}$ e $B = \{(0,0), (1,0)\}$.
  Contudo, $P_1 \neq P_2$, mesmo tendo $\mathcal{P}(\Omega) = \sigma(\{A,B\})$.
\end{example}


\section{Elementos aleatórios}

Muitas vezes não estamos interessados no resultado exato do nosso experimento aleatório, mas sim em uma determinada medição ou função de $\omega \in \Omega$.
Por exemplo, no caso do Exemplo~\ref{x:espacos_amostrais} $c)$, talvez não nos interesse toda a função $f$, mas apenas o seu valor no fim do dia $f(1)$.
Essas medições são ditas elementos aleatórios que definimos à seguir.

Seja $(E,\mathcal{A})$ um espaço mensurável.
Nesse caso, se $X: \Omega \to E$ é uma função $(\mathcal{F}, \mathcal{A})$-mensurável, dizemos que $X$ é um \emph{elemento aleatório} \index{elemento aleatorio@elemento aleatório} em $(\Omega, \mathcal{F})$ tomando valores em $E$, ou um $E$-elemento aleatório.

\begin{example} Consideramos os casos
  \begin{enumerate}[\quad a)]
  \item $X:\Omega \to \mathbb{R}$ mensurável é dita variável aleatória. \index{variavel aleatoria@variável aleatória}
  \item $X:\Omega \to \mathbb{R}^d$ mensurável é dito vetor aleatório ($d$-dimensional).
  \item $X:\Omega \to C[0,1]$ mensurável é dita função aleatória.
  \end{enumerate}
\end{example}
Seguindo a motivação do Exemplo~\ref{x:espacos_amostrais} $c)$, poderia ser que, por exemplo, estivéssemos interessados apenas na variável aleatória $X:\Omega_3 \to \mathbb{R}$ dada por $X(f) = f(1)$.

\begin{exercise}
  Mostre que $X:\Omega_3 \to \mathbb{R}$ dada por $X(f) = f(1)$ é uma variável aleatória.
\end{exercise}

Citando Kingman em seu livro Poisson Processes: ``\emph{a random elephant is a function from $\Omega$ into a suitable space of elephants.}''

Relembrando a nossa notação: $P[X \in A] = P(\{\omega \in \Omega; X(\omega) \in A\})$.

\begin{proposition}
  Seja $X:\Omega \to E$ onde $(E, \mathcal{A})$ é um espaço mensurável com $\mathcal{A} = \sigma(\mathcal{G})$.
  Então para verificar que $X$ é um elemento aleatório, basta provar que $X^{-1}(G) \in \mathcal{F}$ para todo $G \in \mathcal{G}$.
\end{proposition}

\begin{proof}
  Teoria da Medida.
\end{proof}

\begin{example}
  Se $\Omega$ e $E$ são espaços topológicos dotados das correspondentes $\sigma$-álgebras de Borel, então toda função contínua é um $E$-elemento aleatório.
\end{example}

\subsection{Distribuição de elementos aleatórios}

\begin{definition}
  Se $X:\Omega \to E$ é um elemento aleatório e $\Omega$ é dotado de uma probabilidade $P$, então denotamos por $X \circ P$, a chamada \emph{distribuição de $X$} \index{distribuicao@distribuição}, a medida de probabilidade
  \begin{equation}
    (X \circ P)(A) := P\big( \{\omega \in \Omega; X(\omega) \in A\} \big) = P[X \in A].
  \end{equation}
  no espaço mensurável $(E,\mathcal{A})$.
\end{definition}

Fica como exercício verificar que $X \circ P$ é de fato uma probabilidade em $E$.

\begin{exercise}
  Seja $X:[0,1] \to \{0,1\}$ dada por $X(\omega) = \1_A (\omega)$.
  Nesse caso, mostre que $X \circ P = \Ber(p)$ para algum $p \in [0,1]$.
  Calcule o valor de $p$.
\end{exercise}

Duas notações importantes nesse contexto são:
\begin{enumerate}[\quad a)]
\item Dizemos que $X \distr Y$, \index{X d Y@$X \distr Y$} quando $X \circ P = Y \circ P'$.
Note que $X$ e $Y$ nem ao menos precisam pertencer ao mesmo espaço de probabilidade para dizermos que são \emph{igualmente distribuídos}, mas precisam ser elementos aleatórios de mesmo tipo (ou seja, possuir o mesmo contradomínio).
\item Escrevemos $X \distr \mu$, \index{X d mu@$X \distr \mu$} que lê-se \emph{$X$ é distribuída como $\mu$}, onde $\mu$ é uma probabilidade em $E$, caso $X \circ P = \mu$.
\end{enumerate}

\begin{exercise}
  Sejam $X$ e $Y$ variáveis aleatórias tais que $X$ é nula quase certamente.
  Mostre que $X + Y$ tem a mesma distribuição de $Y$.
\end{exercise}

O exercício acima é bastante simples, mas o usaremos para fazer uma importante observação sobre como são enunciados tipicamente os resultados de probabilidade.

Raramente encontramos teoremas que explicitam qual é o espaço de probabilidades $\Omega$ em questão.
Como no exercício acima, o contexto de um teorema frequentemente é dado apenas em termos de elementos aleatórios em $\Omega$ e de suas distribuições.
Dessa forma, podemos utilizar o resultado em vários contextos diferentes, desde que possamos encontrar elementos aleatórios que satisfaçam as hipóteses.
Com o tempo, passamos até mesmo a considerar menos relevante a escolha específica do espaço amostral, focando cada vez mais na distribuição de seus elementos aleatórios.

%%% Local Variables:
%%% mode: latex
%%% TeX-master: "../Notas_de_aula"
%%% End:
