\documentclass[../main/Notas_de_aula.tex]{subfiles}

\begin{document}

\section{Espaços produto infinito}
\label{s:Omega_produto}

Nessa seção estudaremos $\Omega$ que são dados por produtos enumeráveis de outros espaços de probabilidade.
Mas antes iremos recordar o Teorema da Extensão de Caratheodory.

\subsection{Recordar é viver...}

Vamos lembrar o enunciado do Teorema da Extensão de Caratheodory \index{Teorema!da Extensao de Caratheodory@da Extensão de Caratheodory}.
Antes, vamos relembrar uma definição definição importante.
Uma família $\mathcal{G} \subseteq \mathcal{P}(\Omega)$ é dita uma álgebra de conjuntos \index{anel de conjuntos} se valem:
\begin{enumerate}[\quad a)]
  \item $\Omega \in \mathcal{G}$.
  \item Se $A \in \mathcal{G}$, então $A^c \in \mathcal{G}$.
  \item Para todo $n \geq 1$, se $A_1, \dots, A_n \in \mathcal{G}$, então $\bigcup_{i=1}^n A_i \in \mathcal{G}$.
\end{enumerate}

\begin{theorem}[Teorema da Extensão de Caratheodory]
  Seja $\mathcal{G} \subseteq \mathcal{P}(\Omega)$ uma álgebra de conjuntos em $\Omega$ e suponha que $\mu: \mathcal{G} \to \mathbb{R}_+$ satisfaça a seguinte propriedade:
  \begin{display}
    \label{e:aditiva_na_algebra}
    Se $(A_i)_{i\in I}$ e uma familia finita ou enumerável de elementos disjuntos de $\mathcal G$ tal que $\cup_{i\in I} A_i \in \mathcal{G}$,\\
  temos $\mu(\cup_{i\in I} A_i) = \sum_{i\in I} \mu(A_i)$.
  \end{display}
  Então existe uma medida $\widebar{\mu}: \sigma(\mathcal{G}) \to \mathbb{R}_+$ tal que $\widebar{\mu}(A) = \mu(A)$ para todo $A \in \mathcal{G}$.
\end{theorem}

Mostraremos agora uma consequência simples do teorema acima, que é muito utilizada em probabilidade.

\begin{lemma}[Extensão por continuidade no vazio]
  \label{l:extensao_vazio}
  \index{continuidade no vazio}
  Seja $\mathcal{G} \subseteq \mathcal{P}(\Omega)$ uma álgebra de conjuntos em $\Omega$ e suponha que $P: \mathcal{G} \to \mathbb{R}_+$ satisfaça as seguintes propriedades:
  \begin{enumerate}[\quad a)]
  \item $P(\Omega) = 1$,
    \item $P$ é finitamente aditiva e
    \item sempre que $B_1 \supseteq B_2 \supseteq \dots \in \mathcal{G}$ forem tais que $\cap_i B_i = \varnothing$ (denotamos isso por $B_i \downarrow \varnothing$), temos que $\lim_i \mu(B_i) = 0$.
  \end{enumerate}
  Então existe uma única medida $\widebar{P}: \sigma(\mathcal{G}) \to \mathbb{R}_+$ tal que $\widebar{P}(A) = P(A)$ para $A \in \mathcal{G}$.
\end{lemma}

Observe que $P(\Omega) = 1$ somente é necessário para provar a unicidade de $\widebar{P}$, então poderíamos tentar mostrar uma versão mais geral desse lema.
Mas no contexto de medidas infinitas, não é de se esperar que $B_i \downarrow \varnothing$ implique $\lim_i \mu(B_i) = 0$, como foi assumido acima (veja também a Proposição~\ref{p:prob_continua}).
Portanto resolvemos escrever o enunciado com probabilidades.

\begin{exercise}
  Dê um exemplo de medida que não satisfaz a segunda hipótese do Lema~\ref{l:extensao_vazio}.
\end{exercise}

\begin{proof}
  Primeiro observe que a unicidade segue da Proposição~\ref{p:P12_equal_pi}, já que $\mathcal{G}$ é um $\pi$-sistema.
  Iremos agora mostrar que a propriedade \eqref{e:aditiva_na_algebra} é válida para $P$, logo tome $A_1, A_2, \dots \in \mathcal{G}$ disjuntos e tais que $A = \cup_{i\in \mathbb{N}} A_i \in \mathcal{G}$.
  Definimos o ``resto da união'' por
  \begin{equation}
    B_n = A \setminus \mcup_{i=1}^n A_i.
  \end{equation}
  Claramente
  \begin{enumerate}[\quad a)]
  \item $B_n \downarrow \varnothing$ e
  \item $B_n \in \mathcal{G}$, pois $\mathcal{G}$ é uma álgebra.
  \end{enumerate}

  Logo podemos escrever $A$ como a união disjunta $A = \bigcup_{i=1}^n A_i \cup B_n$ e já que $P$ é finitamente aditiva,
  \begin{equation}
    P(A) = \sum_{i=1}^n P(A_i) + P(B_n),
  \end{equation}
  mas como $\lim_{n\to \infty} P(B_n) = 0$, temos
  \begin{equation}
    P(\cup_{i=1}^{\infty} A_i) = \sum_{i=1}^{\infty} P(A_i),
  \end{equation}
  mostrando a propriedade \eqref{e:aditiva_na_algebra} e concluindo o teorema.
\end{proof}

\end{document}
