\documentclass[../main/Notas_de_aula.tex]{subfiles}

\begin{document}

\section{Espaços produto infinito}
\label{s:Omega_produto}

Nessa seção estudaremos $\Omega$ que são dados por produtos enumeráveis de outros espaços de probabilidade.
Mas antes iremos recordar o Teorema da Extensão de Caratheodory.

\subsection{Recordar é viver...}

Vamos lembrar o enunciado do Teorema da Extensão de Caratheodory \index{Teorema!da Extensao de Caratheodory@da Extensão de Caratheodory}.
Antes, vamos relembrar uma definição definição importante.
Uma família $\mathcal{G} \subseteq \mathcal{P}(\Omega)$ é dita uma álgebra de conjuntos \index{anel de conjuntos} se valem:
\begin{enumerate}[\quad a)]
  \item $\Omega \in \mathcal{G}$.
  \item Se $A \in \mathcal{G}$, então $A^c \in \mathcal{G}$.
  \item Para todo $n \geq 1$, se $A_1, \dots, A_n \in \mathcal{G}$, então $\bigcup_{i=1}^n A_i \in \mathcal{G}$.
\end{enumerate}

\begin{theorem}[Teorema da Extensão de Caratheodory]
  Seja $\mathcal{G} \subseteq \mathcal{P}(\Omega)$ uma álgebra de conjuntos em $\Omega$ e suponha que $\mu: \mathcal{G} \to \mathbb{R}_+$ satisfaça a seguinte propriedade:
  \begin{display}
    \label{e:aditiva_na_algebra}
    Se $(A_i)_{i\in I}$ e uma familia finita ou enumerável de elementos disjuntos de $\mathcal G$ tal que $\cup_{i\in I} A_i \in \mathcal{G}$,\\
  temos $\mu(\cup_{i\in I} A_i) = \sum_{i\in I} \mu(A_i)$.
  \end{display}
  Então existe uma medida $\widebar{\mu}: \sigma(\mathcal{G}) \to \mathbb{R}_+$ tal que $\widebar{\mu}(A) = \mu(A)$ para todo $A \in \mathcal{G}$.
\end{theorem}

Mostraremos agora uma consequência simples do teorema acima, que é muito utilizada em probabilidade.

\begin{lemma}[Extensão por continuidade no vazio]
  \label{l:extensao_vazio}
  \index{continuidade no vazio}
  Seja $\mathcal{G} \subseteq \mathcal{P}(\Omega)$ uma álgebra de conjuntos em $\Omega$ e suponha que $P: \mathcal{G} \to \mathbb{R}_+$ satisfaça as seguintes propriedades:
  \begin{enumerate}[\quad a)]
  \item $P(\Omega) = 1$,
    \item $P$ é finitamente aditiva e
    \item sempre que $B_1 \supseteq B_2 \supseteq \dots \in \mathcal{G}$ forem tais que $\cap_i B_i = \varnothing$ (denotamos isso por $B_i \downarrow \varnothing$), temos que $\lim_i \mu(B_i) = 0$.
  \end{enumerate}
  Então existe uma única medida $\widebar{P}: \sigma(\mathcal{G}) \to \mathbb{R}_+$ tal que $\widebar{P}(A) = P(A)$ para $A \in \mathcal{G}$.
\end{lemma}

Observe que $P(\Omega) = 1$ somente é necessário para provar a unicidade de $\widebar{P}$, então poderíamos tentar mostrar uma versão mais geral desse lema.
Mas no contexto de medidas infinitas, não é de se esperar que $B_i \downarrow \varnothing$ implique $\lim_i \mu(B_i) = 0$, como foi assumido acima (veja também a Proposição~\ref{p:prob_continua}).
Portanto resolvemos escrever o enunciado com probabilidades.

\begin{exercise}
  Dê um exemplo de medida que não satisfaz a segunda hipótese do Lema~\ref{l:extensao_vazio}.
\end{exercise}

\begin{proof}
  Primeiro observe que a unicidade segue da Proposição~\ref{p:P12_equal_pi}, já que $\mathcal{G}$ é um $\pi$-sistema.
  Iremos agora mostrar que a propriedade \eqref{e:aditiva_na_algebra} é válida para $P$, logo tome $A_1, A_2, \dots \in \mathcal{G}$ disjuntos e tais que $A = \cup_{i\in \mathbb{N}} A_i \in \mathcal{G}$.
  Definimos o ``resto da união'' por
  \begin{equation}
    B_n = A \setminus \mcup_{i=1}^n A_i.
  \end{equation}
  Claramente
  \begin{enumerate}[\quad a)]
  \item $B_n \downarrow \varnothing$ e
  \item $B_n \in \mathcal{G}$, pois $\mathcal{G}$ é uma álgebra.
  \end{enumerate}

  Logo podemos escrever $A$ como a união disjunta $A = \bigcup_{i=1}^n A_i \cup B_n$ e já que $P$ é finitamente aditiva,
  \begin{equation}
    P(A) = \sum_{i=1}^n P(A_i) + P(B_n),
  \end{equation}
  mas como $\lim_{n\to \infty} P(B_n) = 0$, temos
  \begin{equation}
    P(\cup_{i=1}^{\infty} A_i) = \sum_{i=1}^{\infty} P(A_i),
  \end{equation}
  mostrando a propriedade \eqref{e:aditiva_na_algebra} e concluindo o teorema.
\end{proof}

\subsection{Teorema da Extensão de Kolmogorov}

O objetivo desta seção é provar um resultado que nos permitirá construir probabilidades em espaços produtos infinitos.
Antes precisaremos de introduzir algumas notações.
Dada uma coleção de espaços $(E_i)_{i\in \mathbb{N}}$, definimos o espaço produto
\begin{equation}
  \Omega = \prod_{i=1}^{\infty} E_i = \big\{(\omega_i)_{i\in \mathbb{N}} \, : \,  \omega_i \in E_i \text{ para todo $i \geq 1$}\big\}.
\end{equation}
e os mapas $X_i:\Omega \to E_i$, definidos para $i = 1, 2, \dots$ por
\begin{equation}
  X_i(\omega_1, \omega_2, \dots) = \omega_i,
\end{equation}
que chamamos de \emph{coordenadas canônicas} \index{coordenadas canonicas@coordenadas canônicas} associadas ao produto $\Omega$.

Se cada $E_i$ é dotado de uma $\sigma$-álgebra $\mathcal{A}_i$, então definimos
\begin{equation}
  \mathcal{F} = \sigma( (X_i)_{i\geq 1} ),
\end{equation}
que é claramente uma a $\sigma$-álgebra em $\Omega$.
Chamamos $\mathcal{F}$ de $\sigma$-álbegra canônica.

\begin{exercise}
  Mostre que em $(\mathbb{R}^{\mathbb{N}},\mathcal{F})$ temos que os conjuntos
  \begin{enumerate}[\quad a)]
  \item $A = \{ \liminf_{n\to \infty} X_n \notin \{\infty,-\infty\} \}$,
  \item $B = \{ \lim_{n\to \infty} X_n = 4\}$ e
  \item $C = \{ \lim_{n\to \infty} \tfrac{1}{n} X_n \text{ existe}\}$
  \end{enumerate}
  são todos mensuráveis (eventos) com respeito a $\mathcal{F}$.
  Além disso $Y = \1_A \liminf_{n\to \infty} X_n$ é uma variável aleatória em $(\Omega, \mathcal{F})$.
\end{exercise}

\begin{exercise}
  Verifique as seguinte afirmações
  \begin{enumerate}[\quad a)]
  \item $\mathcal{F} = \sigma\big(A_1 \times \dots \times A_k \times E_{k+1} \times E_{k+2} \times \dots\, : \, k \geq 1, A_i \in \mathcal{A}_i, i \leq k\big)$,
  os chamados eventos retangulares.
  \item $\mathcal{F} = \sigma\big(A \times E_{k+1} \times E_{k+2} \times \dots\, : \, k \geq 1, A \in \mathcal{A}_i \otimes \dots \otimes \mathcal{A}_k\big)$,
  conhecidos como eventos cilíndricos.
  \end{enumerate}
\end{exercise}

\begin{definition}
  \label{d:marginal}
  Seja $\Omega = \prod_{i\in I} E_i$ um espaço produto (infinito ou finito) dotado de uma probabilidade $P$.
  Se $X_i$ é uma coordenada canônica, então chamamos a probabilidade $(X_i)_*  P$ de \emph{distribuição marginal} \index{distribuicao@distribuição!marginal} de $P$ na coordenada $i$.
\end{definition}

\begin{theorem}[Extensão de Kolmogorov]
  \index{Teorema!da Extensao de Kolmogorov@da Extensão}
  \label{t:extens_kolmog}
  Seja para cada $n \geq 1$ uma medida de probabilidade $P_n$ em $\mathbb{R}^n$ tal que seja satisfeita a seguinte condição de compatibilidade \index{condicao de compatibilidade@condição de compatibilidade}
  \begin{equation}
    \label{e:consist_kolmog}
    P_{n+1} (A \times \mathbb{R}) = P_n (A), \text{ para todo $A \in \mathcal{B}(\mathbb{R}^n)$}.
  \end{equation}
  Então existe uma única probabilidade $P$ no espaço produto infinito $(\Omega, \mathcal{F})$ tal que $P(A \times \mathbb{R} \times \dots) = P_n (A)$ para todo $n$ e todo boreliano $A$ de $\mathbb{R}^n$.
\end{theorem}

\begin{proof}
  Considere a classe de conjuntos
  \begin{equation*}
    \mathcal{S}_l = \Big\{ \mcup_{j=1}^k [a_{1,j}, b_{1,j}) \times \dots \times [a_{l,j}, b_{l,j}) \subseteq \mathbb{R}^l \, : \,
    a_{i,j} \in \mathbb{R} \cup \{-\infty\},\ b_{i,j} \in \mathbb{R} \cup \{\infty\} \Big\}.
  \end{equation*}
  Que é obviamente uma álgebra em $\mathbb{R}^l$ e seja também
  \begin{equation}
    \mathcal{S} = \big\{ A \times \mathbb{R} \times \dots\, : \, \text{ onde } l \geq 1 \text{ e } A \in \mathcal{S}_l \big\}.
  \end{equation}
  Claramente, $\mathcal{S}$ também é uma álgebra.

  Se $B = A \times \mathbb{R} \times \dots \in \mathcal{S}$ com $A \in \mathcal{S}_l$ como acima, definimos
  \begin{equation}
    P(B) = P_l(A).
  \end{equation}
  Note que por \eqref{e:consist_kolmog} essa definição independe da escolha da escolha de $l$ que usamos na definição de $B$.

  Gostaríamos agora de utilizar o Lemma~\ref{l:extensao_vazio}.
  Para tanto, tome uma sequência encaixada $B_1 \supseteq B_2 \supseteq \dots \in \mathcal{S}$ e, supondo que $P(B_n) \geq \delta > 0$ para todo $n \geq 1$, temos de mostrar que sua interseção não pode ser vazia.

  Como $B_n \in \mathcal{S}$, podemos escrever
  \begin{equation}
    B_n = A_n \times \mathbb{R} \times \dots, \text{ onde $A_n \in \mathcal{S}_{l_n}$ e $n \geq 1$.}
  \end{equation}
  Podemos obviamente supor que
  \begin{equation}
    \label{e:l_n_monotona}
    \text{$l_n$ são estritamente crescentes.}
  \end{equation}

  A fim de obter um ponto na interseção de $B_n$, gostaríamos de aproximá-lo usando conjuntos compactos encaixados.
  Para tanto definimos os conjuntos
  \begin{equation}
    C_n = C_n^* \times \mathbb{R} \times \dots, \text{ com $C_n^* \in \mathcal{S}_{l_n}$}
  \end{equation}
  de forma que $C_n^*$ seja compacto, $C_n^* \subseteq A_n$ e
  \begin{equation}
    P(B_n \setminus C_n) \leq \frac{\delta}{2^{l_n + 1}},
  \end{equation}
  o que pode ser feito graças à continuidade de $P_{l_n}$, que é uma probabilidade.

  Temos ainda um problema, pois os conjuntos $C_n$ não são encaixados, e isso nos impede de utilizar resultados sobre interseções de compactos.
  Introduzimos pois $D_n = \bigcap_{i=1}^n C_i$, que obviamente pertence à álgebra $\mathcal{S}$, e estimamos
  \begin{equation}
    P(B_n \setminus D_n) = P \big( \mcup\nolimits_{i=1}^n (B_n \setminus C_i) \big) \leq \sum_{i=1}^n P(B_n \setminus C_i) \leq \frac{\delta}2,
  \end{equation}
  donde $P(D_n) = P(B_n) - P(B_n \setminus D_n) \geq \delta/2$.
  De forma que os $D_n$ são encaixados e não vazios.

  Nosso próximo obstáculo vem do fato de que os conjuntos $D_n$ estão definidos em $\mathbb{R}^\mathbb{N}$, e gostaríamos de ter conjuntos em espaços de dimensão finita.
  Isso pode ser feito observando que podemos escrever $D_n = D_n^* \times \mathbb{R} \times \mathbb{R} \times \dots$, onde $D_n^* \in \mathcal{S}_{l_n}$ e
  \begin{equation}
    D_n^* = \underbrace{C_n^*}_{\mathclap{\text{compacto}}} \mcap \underbrace{ \Big( \mcap_{i=1}^{n-1} C_i^* \times \mathbb{R}^{l_n - l_i} \Big)}_{\text{fechado}},
  \end{equation}
  de forma que os $D_n^* \subseteq \mathbb{R}^{l_n}$ são compactos  e não vazios.

  Para cada $n \geq 1$ considere um $\omega^n \in D_n$.
  Usando um argumento de diagonal de Cantor, podemos obter um $\omega \in \Omega$
  e uma sub-sequência de $\omega^{n_j}$ que convirja para $\omega \in \Omega$ coordenada a coordenada
  (observe que $\omega^{n_j} \in \mathbb{R}^{\smash{l_{n_j}}}$).  % Tomando subsequências se necessário, podemos supor que $\omega^n$ converge coordenada a coordenada a um certo $\omega \in \Omega$.
 Para concluir a prova mostramos que $\omega \in \bigcap_{n\ge 1} B_n$.
 Para isso e suficiente mostrar (lembramos que por definição $C_n \subseteq B_n$) que para todo $n\in \mathbb{N}$
  \begin{equation*}
\omega = (\omega_1, \omega_2, \dots) \in C_n.
  \end{equation*}
  O que e equivalente a $(\omega_1, \omega_2, \dots, \omega_n) \in C^*_n$, que vale por compacidade.
\end{proof}

Observe que usamos muito poucos atributos de $\mathbb{R}$ na prova.
Poderíamos na verdade substituir $\mathbb{R}$ por um espaço métrico que satisfaça certas propriedades, como por exemplo a existência de uma álgebra cujos conjuntos possam ser aproximados por compactos.
Contudo, decidimos não apresentar essa versão mais geral aqui porque muito em breve obteremos uma versão bem mais geral do Teorema de Kolmogorov usando apenas o resultado para $\mathbb{R}$.

\begin{exercise}
  Mostre que a hipótese \eqref{e:consist_kolmog} pode ser substituida por
  \begin{equation}
    P_{n+1} (I_1 \times \dots, \times I_n \times \mathbb{R}) = P_n (I_1 \times \dots \times I_n),
  \end{equation}
  para todo $n \geq 1$ e $I_i = (-\infty, b_i]$, onde $b_i \in \mathbb{R}$, $i \leq n$.
\end{exercise}

Um importante exemplo do uso deste teorema é o seguinte.

\begin{example}
  Se $P_i$ são probabilidades em $(\mathbb{R}, \mathcal{B}(\mathbb{R}))$, podemos definir $\mathbb{P}_n = \bigotimes_{i=1}^n P_i$ (relembrando, $\mathbb{P}_n$ é a única distribuição em $\mathbb{R}^n$ tal que $\mathbb{P}_n(A_1 \times \dots \times A_n) = \prod_{i=1}^n P_i(A_i)$).
  Não é difícil verificar que essa lei satisfaz as equações de consistência \eqref{e:consist_kolmog}.
  Desta forma, podemos construir uma única $\mathbb{P}$ em $\mathbb{R}^\mathbb{N}$ para os quais as coordenadas canônicas $X_i$ são independentes e possuem distribuições marginais $P_i$.
  Denotamos nesse caso $\mathbb{P} = \bigotimes_{i \geq 1} P_i$.
\end{example}

Mais adiante no texto daremos outros exemplos bastante interessantes do uso do Teorema~\ref{t:extens_kolmog}.

\begin{exercise}
  Mostre que se $p > 0$ e $\mathbb{P} = \bigotimes_{i \geq 1} \Ber(p)$ em $\mathbb{R}^\mathbb{N}$, então
  \begin{equation}
    \text{$\limsup_{n\to \infty} X_n = 1$ quase certamente.}
  \end{equation}
\end{exercise}

\begin{exercise}
  Mostre que se $\mathbb{P} = \bigotimes_{i \geq 1} U_{[0,1]}$ em $\mathbb{R}^\mathbb{N}$, então
  \begin{equation}
    \text{$\limsup_{n\to \infty} X_n = 1$ quase certamente.}
  \end{equation}
\end{exercise}

\begin{exercise}
  Mostre que se $\mathbb{P} = \bigotimes_{i \geq 1} \Exp(i)$ em $\mathbb{R}^\mathbb{N}$, então
  \begin{equation}
    \text{$\limsup_{n\to \infty} X_n < \infty$ quase certamente.}
  \end{equation}
\end{exercise}



\end{document}
