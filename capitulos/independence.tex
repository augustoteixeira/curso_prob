% !TEX encoding = UTF-8 Unicode

\chapter{Independência e espaços produtos}

\section{Independência}


Nossa intuição nos diz que quando jogamos duas moedas, o resultado de cada uma delas não deve depender um do outro.
Dessa forma, a probabilidade de obtermos um determinado resultado (como por exemplo duas caras) deve ser um quarto, ou seja meio vezes meio.

Em geral, definimos dois eventos como independentes da seguinte forma.

\begin{definition}
  Dizemos que dois eventos $A, B \in \mathcal{F}$, são \emph{independentes} \index{independencia@independência!de eventos} se
  \begin{equation}
    P(A \cap B) = P(A) P(B).
  \end{equation}
\end{definition}

\begin{example}
  Se $\Omega = \{1, \dots, 6\}$ é dotada da $\sigma$-álgebra das partes e e $P(A) = \#A/6$, então os eventos $A = [\omega \text{ é impar}]$ e $B = [\omega \geq 5]$ satisfazem
  \begin{equation}
    P(A \cap B) = P(\{5\}) = 1/6 = (1/2) (1/3) = P(A) P(B).
  \end{equation}
  Logo tais eventos são independentes.
\end{example}

\begin{exercise}
  Seja $\Omega = \{0,1\}^n$ com $P(A) = \#A/2^n$ e $X_i(\omega_1, \dots, \omega_n) = \omega_i$ para $i = 1, \dots, n$.
  Mostre que para qualquer $i\ne j$
  \begin{equation}
    P[X_i = a\ ; \ X_j = b] = P[X_i = a] P[X_j = b],
  \end{equation}
  onde $[A\ ; \ B]$ denota a interseção $[A] \cap [B]$.
\end{exercise}

\subsection{Coleções de eventos}

\todo{discutir a alternativa $I = \{1,\dots, k\}$ que é ruim (basta adicionar $\varnothing$ que qq coisa fica indep).}

\begin{definition}
  Sejam $A_1, A_2, \dots, A_k$ eventos.
  Dizemos que eles formam uma coleção independente \index{independencia@independência!de eventos} se para todo $I \subseteq \{1, \dots, k\}$ não vazio
  \begin{equation}
    P\big( \mcap\nolimits_{i \in I} A_i \big) =  \prod\limits_{i \in I} P(A_i).
  \end{equation}
\end{definition}



Vale observar que independência dois a dois não implica independência.
Mais precisamente
\begin{example}
  Seja $\Omega = \{1,2,3,4\}$ com $P(A) = \# A/4$ e sejam os seguintes eventos: $A_1 = \{1,2\}$, $A_2 = \{2,3\}$ e $A_3 = \{1,3\}$.
  Nesse caso,
  \begin{enumerate}[\quad a)]
  \item $P(A_i) = 1/2$ para $i = 1, 2, 3$,
  \item $P(A_i \cap A_j) = 1/4$ para todo $i \neq j$ mas
  \item $P(A_1 \cap A_2 \cap A_3) = 0 \neq 1/8 = P(A_1) P(A_2) P(A_3)$.
  \end{enumerate}
\end{example}

\begin{example}
  Sejam $n$ patos e $m$ caçadores.
  Cada caçador escolhe um pato aleatoriamente e uniformemente e atira (abatendo-o com probabilidade $p$).
  Seja $X = \# \{\text{patos vivos}\}$, que pode ter uma distribuição complicada de calcular, mas
  \begin{equation}
    \begin{split}
      E(X) &= E \Big( \sum_{i=1}^n \1_{[\text{pato $i$ vive}]} \Big) = \sum_{i=1}^n P[\text{pato $i$ vive}]\\
      &= n P[\text{pato $1$ vive}] = P\Big( \mcap_{j=1}^m [\text{caçador $j$ não mata pato $1$}] \Big)\\
      &= n P[\text{caçador $j$ não mata pato $1$}]^m = n \Big(1 - \frac{p}{n}\Big).
    \end{split}
  \end{equation}
  Observe que
  \begin{enumerate}[\quad a)]
  \item acima obtivemos uma igualdade e
  \item $[\text{pato $i$ vive}]$, $i = 1, \dots, n$ não são independentes.
  \end{enumerate}

  Finalmente estimamos (digamos para $n$ par)
  \begin{equation}
    \begin{split}
      & P[\text{patos para o jantar} < n/2] = P[X \geq n/2] \leq \frac{E(X)}{n/2}\\
      & \qquad = 2 \frac{n}{n} \Big( 1 - \frac{p}{n}\Big)^m \leq 2 \exp \{- \frac{pm}{n}\}.
    \end{split}
  \end{equation}
\end{example}

\begin{definition}
  Dizemos que uma coleção infinita de eventos $(A_n)_{n\ge 1}$ é independente \index{independencia@independência!de eventos} se toda sub-coleção finita de tais eventos forem independentes.
\end{definition}

\begin{lemma}
  Se $(A_n)_{n\ge 1}$ forma uma sequencia de eventos independentes, então
  \begin{equation}
    P\Big( \mcap_{i=1}^{\infty} A_i \Big) = \prod\limits_{i=1}^{\infty} P(A_i).
  \end{equation}
\end{lemma}

\begin{proof}
  De fato,
  \begin{equation*}
    P\Big( \mcap_{i=1}^{\infty} A_i \Big) = \lim_{n\to \infty} P\Big( \mcap_{i = 1}^n A_i \Big) = \lim_{n\to \infty} \prod\limits_{i=1}^n P(A_i) = \prod\limits_{i=1}^{\infty} P(A_i). \qedhere
  \end{equation*}
\end{proof}

\begin{exercise}
  Mostre que se $A \in \mathcal{F}$, então $\{B \in \mathcal{F}\, : \, B \text{ é independente de } A\}$ é um $\lambda$-sistema.
  Mostre que este conjunto não e necessariamente uma $\sigma$-álgebra.
\end{exercise}

\begin{exercise}
  Mostre que se $B$ é independente de $A$ para todo $B \in \mathcal{B}$, com $\mathcal{B}$ um $\pi$-sistema, então $B$ é independente de $A$ para todo $B \in \sigma(\mathcal{B})$.
\end{exercise}

\subsection{Independência de elementos aleatórios}


Vamos começar com a noção de $\sigma$-álgebras independentes
\begin{definition}
  Dado um espaço de probabilidade $(\gO,P,\cF)$
Dizemos que as $\sigma$-álgebra $\cF_1,\dots,\cF_n\subset \cF$  
são independentes \index{independencia@independência!de sigma-algebras@de $\sigma$-álgebras}
se 
\begin{equation}
 \forall A_1\in \cF_1,\dots ,A_n \in \cF_n, \ P(\cap_{i=1}^n A_i)=\prod_{i=1}^n P(A_i).
\end{equation}
Nessa definição podemos considerar uma coleção infinita enumerável.
\end{definition}


\begin{exercise}\label{ex:sigmaeventos}
Seja $A_1,\dots A_n$ uma coleção de eventos.
Mostrar que as $\sigma$-álgebras $\cF_1,\dots \cF_n$ definidas for $\cF_i:=\{\emptyset, A_i,A^{\cc}_i, \gO\}$ são independentes se e só se
os eventos $A_1,\dots A_n$ o são.
\end{exercise}



Podemos estender esse conceito a elementos aleatórios, ou seja:
\begin{definition}
Sejam $X_1, \dots, X_k$ elementos aleatórios definidos respetivamente nos espaços $(E_i, \cA_i)$ para $i=1,\dots,k$.
  Dizemos que  $X_1, \dots, X_k$ são  independentes  \index{independencia@independência!de elementos}
 se as respectivas $\sigma$-álgebras $\sigma(X_1), \dots, \sigma(X_k)$ o forem.
 
 \medskip
 
 De jeito equivalente as variáveis são independente se
 \begin{equation}
 \forall A_1\in \cA_1,\dots ,A_k \in \cA_k, \ P([X_1\in A_1]\cap \dots \cap [X_k\in A_k])=\prod_{i=1}^k P[X_ i\in A_i].
\end{equation}
Nessa definição podemos considerar também uma coleção infinita enumerável.
\end{definition}

Quando $X_1, \dots, X_k$ são elementos aleatórios independentes e com a mesma distribuição, escrevemos que $X_i$ 
são \iid (independentes e identicamente distribuídos).

\begin{example}\label{exo:inde}
\begin{itemize}
 \item Consideramos $\gO=\bbR^2$, e $P$ com densidade com respeito a Lebesgue $\rho_1(\go_1)\rho_2(\go_2)$.
Então as variáveis coordenadas $X_1$ e $X_2$ são independentes com distribuições respectivas $P_i(\dd x)=\rho_i(x)\dd x$.
 \item Se $\gO_1$ e $\gO_2$ são espaços enumeráveis e $P_1$ $P_2$ são probabilidades.
 Podemos definir $P$ a probabilidade em $\gO=\gO_1\times \gO_2$ por 
 $$p_{(\go_1,\go_2)}=P_1(\{\go_1\})P_2(\{\go_2\}).$$
 Com essa definição as variáveis $X_i: \gO\to \gO_i$ definidas por $X_i(\go)=\go_i$ são independentes.
 \item Do Exercício \ref{ex:sigmaeventos}, podemos deduzir que se $A_1,\dots,A_n$ são eventos independentes,
 as funções indicadoras associadas são independentes (e reciprocamente).
\end{itemize}


\end{example}





\begin{exercise}
  Com a notação do exercício anterior, mostre que as funções $X_i:\Omega_1 \times \Omega_2 \to \Omega_i$ dadas por
  \begin{equation}
    X_1(x,y) = x \text{ e } X_2 (x,y) = y,
  \end{equation}
  são elementos aleatórios e são independentes.
\end{exercise}

\begin{exercise}
  Mostre que as coordenadas canônicas do exercício anterior no caso $X_i: \mathbb{R}^2 \to \mathbb{R}$ não são independentes segundo a medida $U_{\bbS^1}$.
  Mas o são segundo $U_{[0,1]^2}$ (que é a medida de Lebesgue em $\mathbb{R}^2$ restrita a $[0,1]^2$).
\end{exercise}

\begin{exercise}
  Seja $\Omega = \{0,1\}^n$ com $P(A) = \#A/2^n$ e $X_i(\omega_1, \dots, \omega_n) = \omega_i$ para $i = 1, \dots, n$.
  Mostre que os $X_i$ são independentes.
\end{exercise}

\begin{exercise}\label{ex:indepfunc}
  Sejam $(X_i)_{i \geq 1}$ elementos aleatórios independentes tomando valores em espaços $(E_i)_{i \geq 1}$, respectivamente.
  Mostre que para funções mensuráveis $(f_i)_{i \geq 1}$ temos que $(f_i(X_i))_{i \geq 1}$ são independentes.
\end{exercise}




\begin{exercise}
  Mostre que se $X, Y$ são elementos aleatórios e se $X$ é constante quase certamente então $X$ e $Y$ são independentes.
\end{exercise}

\begin{exercise}
  Sejam $X$ e $Y$ variáveis aleatórias independentes com distribuição $\Exp(1)$, calcule a distribuição de
  \begin{enumerate}[\quad a)]
  \item $\min\{X,Y\}$ e
  \item $X + Y$.
  \end{enumerate}
\end{exercise}


\begin{exercise}
  Sejam $X, Y$ vari\'aveis aleat\'orias tais que
  \begin{equation}
    P[X \leq x, Y \leq y] =
    \begin{cases}
      0 & \quad \text{if $x < 0$,}\\
      (1-e^{-x}) \Big(\frac 12 + \frac 1\pi \tan^{-1} y \Big), & \quad \text{if $x \geq 0$}.
    \end{cases}
  \end{equation}
  \begin{enumerate}[\quad a)]
  \item Mostre que a distribui\c{c}\~ao conjunta $\mu_{(X,Y)}$ \'e
    absolutamente cont\'inua com rela\c{c}\~ao \`a medida de Lebesgue em
    $\mathbb{R}^2$.
  \item Mostre que $X$ e $Y$ s\~ao independentes.
  \end{enumerate}
\end{exercise}

\begin{exercise}
  \label{x:convolucao_densidade}
  Mostre que se $X, Y$ são variáveis aleatórias independentes com distribuições $X \distr f_X(x) \d x$ e $Y \distr f_Y(y) \d y$, então $X + Y$ tem distribuição absolutamente contínua com respeito a Lebesgue e
  \begin{equation}
    f_{X + Y}(z) = \int_{-\infty}^\infty f_Y(z - x) f_X(x) \d x.
  \end{equation}
\end{exercise}

\todo{mandar para depois de produtos infinitos?}

\begin{lemma}[Borel-Cantelli - segunda parte]
  Se $A_1, A_2, \dots \in \mathcal{F}$ são independentes e $p_i = P(A_i)$ satisfazem $\sum_i p_i = \infty$, então
  \begin{equation}
    P[A_i \text{ infinitas vezes}] = 1.
  \end{equation}
\end{lemma}

\begin{proof}
  Queremos mostrar que
  \begin{equation}
    P \Big( \big(\mcap_n \mcup_{i=n}^\infty A_i\big)^c \Big) = 0,
  \end{equation}
  mas
  \begin{equation}
    P \Big( \big(\mcap_n \mcup_{i=n}^\infty A_i\big)^c \Big) = P \Big(\mcup_n \mcap_{i=n}^\infty A_i^c \Big) \leq \sum\limits_n P \Big(\mcap_{i=n}^\infty A_i^c \Big).
  \end{equation}
  Logo basta mostrar que a probabilidade à direita é zero para todo $n$.
  Mas
  \begin{equation}
    \begin{split}
      P \Big(\mcap_{i=n}^\infty A_i^c \Big) & = \prod\limits_{i=n}^\infty P(A_i^c) = \prod\limits_{i=n}^\infty (1 - p_i)\\
      & \leq \prod\limits_{i=n}^\infty \exp\{-p_i\} = \exp\big\{- \sum_{i=n}^\infty p_i\big\} = 0.
    \end{split}
  \end{equation}
  Terminando a prova do lemma.
\end{proof}

\subsection{Esperança e independência}

Provamos agora uma propriedade fundamental de variáveis aleatórias independentes.

\begin{proposition}\label{prop:indep}
  Sejam $X$ e $Y$ variáveis aleatórias independentes e integráveis, então $XY$ e integrável e 
  \begin{equation}
    E(XY) = E(X) E(Y).
  \end{equation}
\end{proposition}


\begin{proof}
Começamos para verificar o resultado para  funções simples. Se $\alpha_1,\dots,\alpha_n$ e $\gb_1,\dots,\gb_m$ são os possíveis valores de $X$ e $Y$
temos 
\begin{multline}
 E[XY]:=\sum_{i=1}^n \sum_{j=1}^m \alpha_i \gb_j P[ X=\alpha_i, Y=\gb_j]
 =  \sum_{i=1}^n \sum_{j=1}^m \alpha_i \gb_j P[ X=\alpha_i] P[Y=\gb_j]\\ = \left( \sum_{i=1}^n \alpha_i \gb_j P[ X=\alpha_i]\right) 
 \left( \sum_{j=1}^m gb_j P[Y=\gb_j] \right)=E[X]E[Y],
\end{multline}
onde tenhamos usado independência de $X$ e $Y$ na segunda igualdade.
Para funções positivas usamos Exercício \ref{ex:indepfunc} para ver que 
$$X_n= 2^{-n} \lfloor 2^n X \rfloor, \text{ e } Y_n= 2^{-n} \lfloor 2^n Y \rfloor $$
são funções simples independentes e convergem monotonicamente para $X$ e $Y$. 
Por Convergência Monótona (usado duas vezes)temos
\begin{equation}
 E[XY]=\lim_{n\to \infty} E[X_nY_n]=\lim_{n\to \infty} E[X_n] E[Y_n]=E[X]E[Y].
\end{equation}
Em particular $XY$ e integrável se $X$ e  $Y$ o são.

\medskip

Finalmente, para $X$ e $Y$ integráveis decompomos $X = X_+ - X_-$ e $Y = Y_+ - Y_-$. 
Temos 
$$XY=X_+Y_+ - X_+Y_- - X_-Y_+ +X_-Y_-$$
Pelo Exercício \ref{ex:indepfunc}, são quatro produtos de variáveis independentes.
em consequência $XY$ e integrável e podemos  concluir por linearidade.
\end{proof}

\begin{remark}
 Obviamente a recíproca não vale. Se $X$ e uma variável com distribuição $U[-1,1]$ e 
 $Y=|X|$, então $E[XY]=0$ mas 
 $$P[X\in [-1/2,1/2] \ ; \ Y>1/2]=0\ne P(X\in [-1/2,1/2])P(Y>1/2)=1/4.$$
 
 
\end{remark}



\begin{corollary}
  Sejam $X_1,\dots,X_n$ variáveis aleatórias independentes e integráveis, então $\prod_{i=1}^n X_i$ e integrável e 
  \begin{equation}
    E[\prod_{i=1}^n X_n] = \prod_{i=1}^n E[X_i].
  \end{equation}
  \end{corollary}
  
  \begin{proof}
   Vamos prosseguir por indução. O resultado já foi provado para $n=1$ e $n=2$.
Supondo o resultado para $n\ge 2$, vamos provar ele para $n+1$.
Sabemos que o vector $(X_1,\dots,X_n)$ e independente de $X_{n+1}$ \ref{ex}. Em consequência $Z_n:=\prod_{i=1}^n X_n$ 
e independente de $X_{n+1}$ (Exercício \ref{ex:indepfunc}). Sabemos que $Z_n$ e integrável pela hipótese de recorrência e temos 
usando o resultado para $n=2$ que e de novo a hipótese de recorrência, concluímos que $\prod_{i=1}^{n+1} X_i$ e integrável e que
\begin{equation}
 E[\prod_{i=1}^{n+1} X_i]=E[Z_n]E[X_{n+1}]=\left(\prod_{i=1}^n E[X_i]\right) E[X_{n+1}].
\end{equation}

   
  \end{proof}






\section{Espaços produto finito}

Dados espaços $\Omega_1, \dots, \Omega_n$ com suas respectivas $\sigma$-álgebras $\mathcal{F}_1, \dots, \mathcal{F}_n$, podemos definir o espaço mensurável produto 
$(\widebar{\Omega}, \widebar{\mathcal{F}})$ da seguinte forma
\begin{equation}
  \widebar{\Omega} = \prod_{i=1}^n \Omega_i \quad \text{e} \quad \widebar{\mathcal{F}} = \sigma \Big( \{  
  A_1 \times \cdots \times A_n  \, : \,  \forall i \in \{1,\dots,n\},\ A_i \in \mathcal{F}_i \} \Big).
\end{equation}
Essa $\sigma$-álgebra e chamada de $\sigma$-álgebra produto e denotaremos ela por $\bigotimes_{i=1}^n \cF_i$,
o $\cF_1\otimes \cF_2$ quando $n=2$.

\begin{proposition}
  Se $(\Omega_1, \mathcal{F}_1, P_1), \dots, (\Omega_n, \mathcal{F}_n, P_n)$ são espaços de probabilidade, então existe uma única probabilidade 
  $\widebar{P}$ no espaço mensurável $(\widebar{\Omega}, \widebar{\mathcal{F}})$ tal que
  \begin{equation}
    \widebar{P}(A_1 \times \cdots \times A_n) = \prod_{i=1}^n P_i(A_i), \text{ para todos $A_i \in \mathcal{F}_i$, $i \leq n$.}
  \end{equation}
  Essa probabilidade é chamada probabilidade produto.
\end{proposition}






\begin{proof}
  Teoria da Medida.
\end{proof}

\begin{notation}
  Usaremos a notação $\bigotimes_{i=1}^n P_i$ o $P_1\otimes P_2 \otimes \dots \otimes P_n$.
\end{notation}



Note que a unicidade do produto pode ser concluída por exemplo usando o Corolário~\ref{c:produto_e_unico}.


\begin{proposition}
Se  $\gO:=\prod_{i=1}^n \Omega_i$ e $P:=\bigotimes_{i=1}^n P_i$ então os elementos aleatórios 
$X_i$ definidos pelas projeções sob as coordenadas $X_i(\go)=\go_i$ são independentes 
de distribuição respectivas $P_i$
 
\end{proposition}


\begin{proof}
Exercício.
\end{proof}



Providenciamos agora um exemplo que da uma aplicação da desigualdade de Markov, Teorema~\ref{t:markov}.

Em vários exemplos importantes, podemos ter dificuldade de calcular probabilidades explicitamente.
Nesses casos, poderíamos gastar nossas energias tentando calculá-las a qualquer custo, ou podemos nos contentar em obter cotas superiores e inferiores para as probabilidades nas quais estamos interessados.

Em vários casos, a segunda estratégia tem uma grande vantagem sobre a primeira, por possibilitar que estudemos problemas mais complexos (e consequentemente mais importantes/interessantes) e muitas vezes sem nos afastarmos da realidade (em vários exemplos as cotas superiores e inferiores são próximas o suficiente para que não nos preocupemos).

\begin{example}
  Sejam $n$ patos e $m$ caçadores.
  Cada caçador escolhe um pato aleatoriamente e uniformemente e atira (abatendo-o com probabilidade $p$).
  Seja $X = \# \{\text{patos vivos}\}$, que pode ter uma distribuição complicada de calcular, mas
  \begin{equation}
    \begin{split}
      E(X) &= E \Big( \sum_{i=1}^n \1_{[\text{pato $i$ vive}]} \Big) = \sum_{i=1}^n P[\text{pato $i$ vive}]\\
      &= n P[\text{pato $1$ vive}] = P\Big( \mcap_{j=1}^m [\text{caçador $j$ não mata pato $1$}] \Big)\\
      &= n P[\text{caçador $j$ não mata pato $1$}]^m = n \Big(1 - \frac{p}{n}\Big).
    \end{split}
  \end{equation}
  Observe que
  \begin{enumerate}[\quad a)]
  \item acima obtivemos uma igualdade e
  \item $[\text{pato $i$ vive}]$, $i = 1, \dots, n$ não são independentes.
  \end{enumerate}

  Finalmente estimamos (digamos para $n$ par)
  \begin{equation}
    \begin{split}
      & P[\text{patos para o jantar} < n/2] = P[X \geq n/2] \leq \frac{E(X)}{n/2}\\
      & \qquad = 2 \frac{n}{n} \Big( 1 - \frac{p}{n}\Big)^m \leq 2 \exp \{- \frac{pm}{n}\}.
    \end{split}
  \end{equation}
\end{example}
\begin{remark}
No caso onde $P_X$ tem suporte em $\{a,b\}$ o resultado e simplesmente a definição da convexidade.
No caso $\varphi(x)=|x|^p$, $p>1$, o resultado pode ser considerado e uma consequência imediata da desigualdade de H\"older.
No caso $\phi(x)=|x|$, o resultado da a desigualdade triangular generalizada (que e valida para integrais com respeito a medidas positivas).
\end{remark}


\begin{exercise}
  Mostre que o produto de $n$ cópias de $(\{0,1\}, \mathcal{P}(\{0,1\}), \Ber(1/2))$ é a distribuição uniforme em $\{0,1\}^n$.
\end{exercise}


\begin{exercise}
  Em um espaço produto $(\Omega_1 \times \Omega_2, \mathcal{F}_1 \otimes \mathcal{F}_2, P_1 \otimes P_2)$, podemos definir
  \begin{equation}
    \begin{split}
      \widebar{\mathcal{F}}_1 & = \{A \times \Omega_2 \, : \, A \in \mathcal{F}_1\},\\
      \widebar{\mathcal{F}}_2 & = \{\Omega_1 \times B \, : \, B \in \mathcal{F}_2\}.
    \end{split}
  \end{equation}
  Mostre que essas $\sigma$-álgebras são independentes.
\end{exercise}


\begin{exercise}
  Seja um espaço produto de medidas $(\Omega_1 \times \Omega_2, \mathcal{F}_1 \otimes \mathcal{F}_2, \mu_1 \otimes \mu_2)$ e defina a probabilidade $P$ atravéz de
  \begin{equation}
    \d P = \rho(x,y) \d (\mu_1 \otimes \mu_2).
  \end{equation}
  Mostre nesse caso que as coordenadas canônicas $X_1$ e $X_2$ são independentes se e somente se existem 
  $\rho_1$ e $\rho_2$ em $\Omega_1$ e $\Omega_2$ respectivamente, tais que $\rho(x,y) = \rho_1(x) \rho_2(y)$ 
  quase certamente com respeito a $\mu_1 \otimes \mu_2$.
\end{exercise}



\begin{exercise}
  Sejam $(X_i)_{i \geq 1}$  elementos aleatórios independentes. Definimos $Y_i:=(X_{2i-1},X_{2i}$.
  Mostrar que os elementos $(Y_i)_{i\ge 1}$ são independentes
  
  \end{exercise}



\begin{exercise}
  Sendo $X_1,\dots, X_n$ variáveis de Bernoulli independentes, dar a distribuição de $Z=\sum_{i=1}^n X_i$.
\end{exercise}





\todosec{Tópico: Uma dinâmica em \texorpdfstring{$[0,1]$}{[0,1]}}{fazer dinâmica 2x mod 1 e relações Lebesgue[0,1] com produtos de bernoulli}

\begin{topics}

\section{Tópico: Lei dos pequenos números}

Nessa seção estudaremos como se comportam limites de algumas variáveis aleatórias bastante importantes, mas primeiramente, uma breve intuição.

Apesar de que descreveremos a nossa motivação a partir desse exemplo do estudo de um material radioativo, podemos encontrar aplicações com justificativas bastante semelhantes para outros problemas, como: chegada de carros em um sinal de trânsito, número de mutações em um gene, número de mortes por ano em uma faixa etária...

Digamos que estamos observando um material radioativo que esporadicamente emite fótons que podemos detectar atravéz de um aparelho.
A razão dessas emissões pode ser aproximada pelo seguinte modelo.
Na amostra temos um número $n$ grande de átomos instáveis ($n \sim 10^{23}$) e em um determinado tempo de observação, cada um deles tem probabilidade muito baixa de decair emitindo um fóton (digamos $p \sim 10^{-23}$).
Nesse caso, supondo que todos decidam emitir de maneira independente, temos para $p \in [0,1]$,
\begin{equation}
  \label{e:Poisson_setup}
  \Omega_n = \{0,1\}^n, \quad \mathcal{F}_n = \mathcal{P}(\Omega) \quad \text{e} \quad P_p = \otimes_{i=1}^n Ber(p).
\end{equation}
Dessa forma, o número total de emissões observadas para $\omega = (\omega_1, \dots, \omega_n) \in \Omega$ é
\begin{equation}
  \label{e:Xn_Poisson}
  X_n(\omega) = \sum_{i=1}^n \omega_i.
\end{equation}
E gostaríamos de entender como se comporta essa distribuição, que nada mais é que $\Bin(n,p)$.

Uma primeira tentativa seria modelar esse processo dizendo que o número de átomos $n$ é tão grande, que somente estamos interessados no comportamento assimtótico quando $n$ vai para infinito.
Mas para manter o número de emissões sob controle, também gostaríamos que $p = p_n$, que converge a zero.
Poderíamos por exemplo escolher
\begin{equation}
  p_n = \frac \lambda n.
\end{equation}
Mas a discussão que se segue é muito mais geral que essa escolha específica.

Como estaremos interessados em um regime assintótico da distribuição de $X_p$ (lembre que apesar do espaço amostral de $X_n$ variar com $n$, sua distribuição é sempre uma probabilidade em $\mathbb{N}$).
Mas para falar de regimes assintóticos, precisamos de definir uma noção de distância entre duas distribuições em $\mathbb{N}$.

\begin{definition}
Dadas duas distribuições $\mu_1$ e $\mu_2$ em $(\Omega, \mathcal{A})$, definimos
\begin{equation}
  \lVert \mu_1 - \mu_2 \rVert_{\VT} = \sup_{A \in \mathcal{A}} |\mu_1(A) - \mu_2(A)|,
\end{equation}
\index{mu1 - mu2@$\lVert \mu_1 - \mu_2 \rVert$} chamada de distância em variação total \index{variacao total@variação total} entre $\mu_1$ e $\mu_2$.
\end{definition}

No nosso caso, $\Omega$ é enumerável.
Vamos ver que nesse caso é possível reescrever a definição acima de modo a ver mais facilmente que se trata de uma distância no espaço de probabilidades em $\Omega$.

\begin{lemma}
\label{l:vt_l1}
Se $\Omega$ for finito o enumerável, então podemos escrever
\begin{equation}
  \lVert \mu_1 - \mu_2 \rVert_{\VT} = \frac{1}{2} \sum_{x\in \gO} |\mu_1(x) - \mu_2(x)|.
\end{equation}
\end{lemma}

\begin{proof}
Para mostrar que o lado esquerdo é maior ou igual ao direito, escolhemos $A = \{ x \in \Omega \, : \, \mu_2(x) \leq \mu_1(x)\}$. Assim
\begin{equation}
  \begin{split}
    \sum_{x \in A} \mu_1(x) - \mu_2(x) & = |\mu_1(A) - \mu_2(A)|\\
    & = |\mu_1(A^c) - \mu_2(A^c)| = \sum_{x \in A^c} \mu_2(x) - \mu_1(x),
  \end{split}
\end{equation}
donde
\begin{equation}
  \lVert \mu_1 - \mu_2 \rVert_{\VT} \geq |\mu_1(A) - \mu_2(A)| = \frac{1}{2} \sum_{i} |\mu_1(x_i) - \mu_2(x_i)|.
\end{equation}

Na outra direção, observe que para todo $B \subseteq \Omega$,
\begin{equation}
  \begin{split}
    \sum_{i} |\mu_1(x_i) - \mu_2(x_i)| & \geq \sum_{x \in B} \mu_1(x) - \mu_2(x) + \sum_{x \in B^c} \mu_1(x) - \mu_2(x)\\
    & = \mu_1(B) - \mu_2(B) + (1 - \mu_2(B)) - (1 - \mu_1(B))\\
    & = 2(\mu_1(B) - \mu_2(B)).
  \end{split}
\end{equation}
O que termina a prova do lema.
\end{proof}

Fica agora claro que $\lVert \mu_1 - \mu_2 \rVert_{\VT}$ determina uma distância.

\begin{exercise}
Mostre um lema análogo ao anterior para $(\Omega, \mathcal{A})$ qualquer, desde que $\mu_1$ e $\mu_2$ sejam absolutamente contínuas com relação à uma medida fixa nesse espaço mensurável. Nesse caso utilizaremos as derivadas de Radon–Nikodym.
\end{exercise}

Como estaremos interessados em variáveis independentes, precisamos de um resultado que relacione a distância em variação total com produtos de medida. Isso é parte do seguinte

\begin{lemma}
\label{l:vt_produto}
Sejam $\mu_1, \mu_2$ distribuições em $\Omega$ e $\nu_1, \nu_2$ distribuições em $y$ ambos enumeráveis. Então
\begin{equation}
  \lVert \mu_1 \otimes \nu_1 - \mu_2 \otimes \nu_2 \rVert_{\VT} \leq \lVert \mu_1 - \mu_2 \rVert_{\VT} + \lVert \nu_1 - \nu_2 \rVert_{\VT}.
\end{equation}
\end{lemma}

\begin{proof}
Basta expandir
\begin{equation}
  \begin{split}
    2\lVert \mu_1 & \otimes \nu_1 - \mu_2 \otimes \nu_2 \rVert_{\VT} = \sum_{x \in \Omega, y \in \gO} |\mu_1(x)\nu_1(y) - \mu_2(x)\nu_2(y)|\\
    & \leq \sum_{x \in \Omega, y \in \gO} |\mu_1(x)\nu_1(y) - \mu_1(x)\nu_2(y)| + |\mu_1(x)\nu_2(y) - \mu_2(x)\nu_2(y)|\\
    & \leq 2\lVert \mu_1 - \mu_2 \rVert_{\VT} + 2\lVert \nu_1 - \nu_2 \rVert_{\VT}.
  \end{split}
\end{equation}
Onde acima nós usamos que $\mu_1$ e $\nu_2$ são probabilidades. Isso termina a prova do lema.
\end{proof}

Finalmente, gostaríamos de entender como a distância de variação total se comporta com respeito à soma de variáveis independentes.
Isso estará ligado à convolução de distribuições:

\begin{definition}
Dadas, $\mu$ e $\nu$ distribuições em $\mathbb{Z}$, definimos a distribuição
\begin{equation}
  (\mu \star \nu)(x) := \sum_{y \in \mathbb{Z}} \mu(x-y) \nu(y).
\end{equation}
\end{definition}

Essa definição se relaciona com a soma de variáveis independentes graças ao seguinte
\begin{exercise}
Se $X \overset{d}\sim \mu$ e $Y \overset{d}\sim \nu$ são variáveis aleatórias inteiras e independentes, então $X + Y \overset{d}\sim \mu \star \nu$.
Dica: particione o espaço amostral nos eventos $[X = j]$, para $j \in \mathbb{Z}$, como na prova do Lema~\ref{l:soma_poisson} abaixo.
\end{exercise}

\begin{corollary}
Se $\mu$ e $\nu$ são distribuições em $\mathbb{Z}$, então $\mu \star \nu = \nu \star \mu$.
\end{corollary}

Como prometido, obtemos a seguinte relação entre a convolução e a distância de  variação total.
\begin{lemma}
\label{l:vt_conv}
Sejam $\mu$, $\nu$ duas medidas em $\gO$ enumerável e $X:\  (\gO,\cP(\gO))\to (E,\cA)$ um elemento aleatório
\begin{equation}
   \lVert X_*\mu - X_*\nu\rVert_{\VT}\le    \lVert \mu - \nu\rVert_{\VT}.
\end{equation}
Em particular se $\mu_1, \mu_2, \nu_1, \nu_2$ são distribuições em $\mathbb{Z}$, então
\begin{equation}
  \lVert \mu_1 \star \nu_1 - \mu_2 \star \nu_2 \rVert_{\VT} \leq \lVert \mu_1 \otimes \nu_1 - \mu_2 \otimes \nu_2 \rVert_{\VT}
\end{equation}
\end{lemma}

\begin{proof}
O segundo ponto segue do primeiro aplicado ao caso $\gO=\bbZ^2$, $E=\bbZ$ e $X:\ (x,y) \mapsto (x+y)$. 
Pelo primeiro, observamos 
\begin{equation}
  \begin{split}
    2\lVert X_*\mu - X_*\nu\rVert_{\VT} &= \sum_{x \in E} \Big| \mu(X(\go)=x) - \nu(X(\go)=x) \Big|\\
    & =  \sum_{x \in E} \big| \sum_{\{\go \in\gO \ : \ X(\go)=x\}}  \mu(\go) - \nu(\go) \big|\\
    & \leq \sum_{\go \in \gO} \big| \mu(\go)- \nu(\go) \big|\\
    & = 2\lVert \mu - \nu\rVert_{\VT}.
  \end{split}
\end{equation}

\end{proof}

Para enunciar o resultado principal dessa seção, vamos apresentar uma distribuição em $\mathbb{N}$ bastante importante, que em particular se comporta muito bem com respeito a somas de variáveis independentes, como veremos.

\begin{definition}
  Uma variável aleatória $X$ é dita ter distribuição de Poisson \index{distribuicao@distribuição!de Poisson} com parâmetro $\lambda$, se
  \begin{equation}
    P[X = k] = \frac{\lambda^k e^{-\lambda}}{k!}, \text{ para $k \geq 0$ inteiro.}
  \end{equation}
  Denotamos isso por $X \overset{d}\sim \Poisson(\lambda)$.
\end{definition}

A distribuição de Poisson se comporta bem com respeito a somas independentes, como mostra o seguinte
\begin{lemma}
\label{l:soma_poisson}
Sejam $X \overset{d}\sim \Poisson(\lambda_1)$ e $Y \overset{d}\sim \Poisson(\lambda_2)$ independentes, então $X+Y \overset{d}\sim \Poisson(\lambda_1 + \lambda_2)$.
\end{lemma}

\begin{proof}
Basta calcular
\begin{equation}
  \begin{split}
    P[X+Y = k] & = \sum_{j = 0}^k P[X = j, Y = k-j] = \sum_{j = 0}^k \frac{\lambda_1^j e^{-\lambda_1} \lambda_2^{k-j} e^{-\lambda_2}}{j! (k-j)!}\\
    & = e^{-(\lambda_1 + \lambda_2)} \frac{1}{k!} \sum_{j = 0}^k \frac{k!}{j! (k-j)!} \lambda_1^j \lambda_2^{k-j} = \frac{e^{(\lambda_1 + \lambda_2)} (\lambda_1 + \lambda_2)^k}{k!},
  \end{split}
\end{equation}
mostrando o resultado.
\end{proof}

Nossa próxima tarefa é estimar a distância entre uma variável aleatória com distribuição $\Ber(p)$ e uma $\Poisson(p)$, como segue.

\begin{lemma}
\index{Lei!dos Pequenos Numeros@dos Pequenos Números}
\label{l:vt_ber_poiss}
Para $p \in [0,1]$, seja $\mu_1 = \Ber(p)$ e $\mu_2 = \Poisson(p)$, então,
\begin{equation}
  \lVert \mu_1 - \mu_2 \rVert_{\VT} \leq p^2.
\end{equation}
\end{lemma}

\begin{proof}
Sabemos que
\begin{equation}
  \begin{split}
    \lVert \mu_1 - \mu_2 \rVert_{\VT} & = \frac{1}{2} \sum_{x} |\mu_1(x) - \mu_2(x)|\\
    & = \frac{1}{2} \Big( |\mu_1(0) - \mu_2(0)| + |\mu_1(1) - \mu_2(1)| + \sum_{x \geq 2} \mu_2(x) \Big)\\
    & = \frac{1}{2} \Big( e^{-p} - (1-p) + p(1-e^{-p}) + (1 - e^{-p} - p e^{-p}) \Big)\\
    & = \frac{2}{2} p (1 - e^{-p}) \leq p^2,
  \end{split}
\end{equation}
terminando a prova.
\end{proof}

O teorema principal de convergência dessa seção concerne a soma de variáveis Bernoulli.

\begin{theorem}[Lei dos Pequenos Números]
  \label{t:lei_peq_numeros}
  Dado, $n \geq 1$ e $p \in [0,1]$, suponha que $\Omega_n$, $\mathcal{F}_n$ e $P_p$ sejam dados como em \eqref{e:Poisson_setup}.
  Então,
  \begin{equation}
    \lVert \Bin(n,p) - \Poisson(pn) \rVert_{\VT} \leq n p^2.
  \end{equation}
\end{theorem}

\begin{proof}
  Basta observar que
  \begin{equation}
    \begin{split}
      \lVert X_n \circ P_p - \Poisson(pn) \rVert_{\VT} & \overset{\text{Lema~\ref{l:soma_poisson}}}= \lVert \Ber(p)^{\star n} - \Poisson(p)^{\star n} \rVert_{\VT}\\
      \overset{\text{Lema~\ref{l:vt_conv}}}\leq & \lVert \Ber(p)^{\otimes n} - \Poisson(p)^{\otimes n} \rVert_{\VT}\\
      \overset{\text{Lema~\ref{l:vt_produto}}}\leq & n \lVert \Ber(p) - \Poisson(p) \rVert_{\VT} \overset{\text{Lema~\ref{l:vt_ber_poiss}}}\leq n p^2,
    \end{split}
  \end{equation}
  provando o teorema.
\end{proof}

\begin{corollary}
  No mesmo contexto do teorema acima, se $p = \lambda/n$, então temos
  \begin{equation}
    \lVert \Bin(n,p) - \Poisson(pn) \rVert_{\VT} \leq \lambda^2 / n,
  \end{equation}
  que converge a zero com $n$.
\end{corollary}
Dizemos que a probabilidade $\Bin(n,\gl/n)$ converge \textsl{em variação total} para $\Poisson(pn)$.
Veremos mais tarde que existem outros tipos de convergência.


\begin{exercise}
  Fixado $\lambda > 0$, seja $N$ uma variável aleatória com distribuição Poisson($\lambda$), isto é
  \begin{equation}
    P[N = k] = \frac{\lambda^k e^{-\lambda}}{k!} \text{ para $k = 0, 1, \dots$}
  \end{equation}
  Considere no mesmo espaço de probabilidade uma sequência de variáveis aleatórias $X_1, X_2, \dots$ que sejam \iid, com distribuição $\Ber(1/2)$ e independentes de $N$.
  \begin{enumerate}[\quad a)]
  \item Calcule a distribuição de $Z = \sum_{i=1}^N X_i$.
  \item Mostre que $Z$ e $N - Z$ são independentes.
  \end{enumerate}
\end{exercise}

\end{topics}

{\color{red}
  Seguem duas seções independentes sobre espaços produto infinitos.
  Escolher uma.
}

\section{Espaços produto infinito}
\label{s:Omega_produto}

Nessa seção estudaremos $\Omega$ que são dados por produtos enumeráveis de outros espaços de probabilidade.
Mas antes iremos recordar o Teorema da Extensão de Caratheodory.

\subsection{Recordar é viver...}

Vamos lembrar o enunciado do Teorema da Extensão de Caratheodory \index{Teorema!da Extensao de Caratheodory@da Extensão de Caratheodory}.
Antes, vamos relembrar uma definição definição importante.
Uma família $\mathcal{G} \subseteq \mathcal{P}(\Omega)$ é dita uma álgebra de conjuntos \index{anel de conjuntos} se valem:
\begin{enumerate}[\quad a)]
  \item $\Omega \in \mathcal{G}$.
  \item Se $A \in \mathcal{G}$, então $A^c \in \mathcal{G}$.
  \item Para todo $n \geq 1$, se $A_1, \dots, A_n \in \mathcal{G}$, então $\bigcup_{i=1}^n A_i \in \mathcal{G}$.
\end{enumerate}

\begin{theorem}[Teorema da Extensão de Caratheodory]
  Seja $\mathcal{G} \subseteq \mathcal{P}(\Omega)$ uma álgebra de conjuntos em $\Omega$ e suponha que $\mu: \mathcal{G} \to \mathbb{R}_+$ satisfaça a seguinte propriedade:
  \begin{display}
    \label{e:aditiva_na_algebra}
    Se $(A_i)_{i\in I}$ e uma familia finita ou enumerável de elementos disjuntos de $\mathcal G$ tal que $\cup_{i\in I} A_i \in \mathcal{G}$,\\
  temos $\mu(\cup_{i\in I} A_i) = \sum_{i\in I} \mu(A_i)$.
  \end{display}
  Então existe uma medida $\widebar{\mu}: \sigma(\mathcal{G}) \to \mathbb{R}_+$ tal que $\widebar{\mu}(A) = \mu(A)$ para todo $A \in \mathcal{G}$.
\end{theorem}

Mostraremos agora uma consequência simples do teorema acima, que é muito utilizada em probabilidade.

\begin{lemma}[Extensão por continuidade no vazio]
  \label{l:extensao_vazio}
  \index{continuidade no vazio}
  Seja $\mathcal{G} \subseteq \mathcal{P}(\Omega)$ uma álgebra de conjuntos em $\Omega$ e suponha que $P: \mathcal{G} \to \mathbb{R}_+$ satisfaça as seguintes propriedades:
  \begin{enumerate}[\quad a)]
  \item $P(\Omega) = 1$,
    \item $P$ é finitamente aditiva e
    \item sempre que $B_1 \supseteq B_2 \supseteq \dots \in \mathcal{G}$ forem tais que $\cap_i B_i = \varnothing$ (denotamos isso por $B_i \downarrow \varnothing$), temos que $\lim_i \mu(B_i) = 0$.
  \end{enumerate}
  Então existe uma única medida $\widebar{P}: \sigma(\mathcal{G}) \to \mathbb{R}_+$ tal que $\widebar{P}(A) = P(A)$ para $A \in \mathcal{G}$.
\end{lemma}

Observe que $P(\Omega) = 1$ somente é necessário para provar a unicidade de $\widebar{P}$, então poderíamos tentar mostrar uma versão mais geral desse lema.
Mas no contexto de medidas infinitas, não é de se esperar que $B_i \downarrow \varnothing$ implique $\lim_i \mu(B_i) = 0$, como foi assumido acima (veja também a Proposição~\ref{p:prob_continua}).
Portanto resolvemos escrever o enunciado com probabilidades.

\begin{exercise}
  Dê um exemplo de medida que não satisfaz a terceira hipótese do Lema~\ref{l:extensao_vazio}.
\end{exercise}

\begin{proof}
  Primeiro observe que a unicidade segue da Proposição~\ref{p:P12_equal_pi}, já que $\mathcal{G}$ é um $\pi$-sistema.
  Iremos agora mostrar que a propriedade \eqref{e:aditiva_na_algebra} é válida para $P$, logo tome $A_1, A_2, \dots \in \mathcal{G}$ disjuntos e tais que $A = \cup_{i\in \mathbb{N}} A_i \in \mathcal{G}$.
  Definimos o ``resto da união'' por
  \begin{equation}
    B_n = A \setminus \mcup_{i=1}^n A_i.
  \end{equation}
  Claramente
  \begin{enumerate}[\quad a)]
  \item $B_n \downarrow \varnothing$ e
  \item $B_n \in \mathcal{G}$, pois $\mathcal{G}$ é uma álgebra.
  \end{enumerate}

  Logo podemos escrever $A$ como a união disjunta $A = \bigcup_{i=1}^n A_i \cup B_n$ e já que $P$ é finitamente aditiva,
  \begin{equation}
    P(A) = \sum_{i=1}^n P(A_i) + P(B_n),
  \end{equation}
  mas como $\lim_{n\to \infty} P(B_n) = 0$, temos
  \begin{equation}
    P(\cup_{i=1}^{\infty} A_i) = \sum_{i=1}^{\infty} P(A_i),
  \end{equation}
  mostrando a propriedade \eqref{e:aditiva_na_algebra} e concluindo o teorema.
\end{proof}

\subsection{Teorema da Extensão de Kolmogorov}

O objetivo desta seção é provar um resultado que nos permitirá construir probabilidades em espaços produtos infinitos.
Antes precisaremos de introduzir algumas notações.
Dada uma coleção de espaços $(E_i)_{i\in \mathbb{N}}$, definimos o espaço produto
\begin{equation}
  \Omega = \prod_{i=1}^{\infty} E_i = \big\{(\omega_i)_{i\in \mathbb{N}} \, : \,  \omega_i \in E_i \text{ para todo $i \geq 1$}\big\}
\end{equation}
e os mapas $X_i:\Omega \to E_i$, definidos para $i = 1, 2, \dots$ por
\begin{equation}
  X_i(\omega_1, \omega_2, \dots) = \omega_i,
\end{equation}
que chamamos de \emph{coordenadas canônicas} \index{coordenadas canonicas@coordenadas canônicas} associadas ao produto $\Omega$.

Se cada $E_i$ é dotado de uma $\sigma$-álgebra $\mathcal{A}_i$, então definimos
\begin{equation}
  \mathcal{F} = \sigma( (X_i)_{i\geq 1} ),
\end{equation}
que é claramente uma $\sigma$-álgebra em $\Omega$.
Chamamos $\mathcal{F}$ de $\sigma$-álbegra canônica.

\begin{exercise}
  Mostre que em $(\mathbb{R}^{\mathbb{N}},\mathcal{F})$ temos que os conjuntos
  \begin{enumerate}[\quad a)]
  \item $A = \{ \liminf_{n\to \infty} X_n \notin \{\infty,-\infty\} \}$,
  \item $B = \{ \lim_{n\to \infty} X_n = 4\}$ e
  \item $C = \{ \lim_{n\to \infty} \tfrac{1}{n} X_n \text{ existe}\}$
  \end{enumerate}
  são todos mensuráveis (eventos) com respeito a $\mathcal{F}$.
  Além disso $Y = \1_A \liminf_{n\to \infty} X_n$ é uma variável aleatória em $(\Omega, \mathcal{F})$.
\end{exercise}

\begin{exercise}
  Verifique as seguinte afirmações
  \begin{enumerate}[\quad a)]
  \item $\mathcal{F} = \sigma\big(A_1 \times \dots \times A_k \times E_{k+1} \times E_{k+2} \times \dots\, : \, k \geq 1, A_i \in \mathcal{A}_i, i \leq k\big)$,
  os chamados eventos retangulares.
  \item $\mathcal{F} = \sigma\big(A \times E_{k+1} \times E_{k+2} \times \dots\, : \, k \geq 1, A \in \mathcal{A}_i \otimes \dots \otimes \mathcal{A}_k\big)$,
  conhecidos como eventos cilíndricos.
  \end{enumerate}
\end{exercise}

\begin{definition}
  \label{d:marginal}
  Seja $\Omega = \prod_{i\in I} E_i$ um espaço produto (infinito ou finito) dotado de uma probabilidade $P$.
  Se $X_i$ é uma coordenada canônica, então chamamos a probabilidade $(X_i)_*  P$ de \emph{distribuição marginal} \index{distribuicao@distribuição!marginal} de $P$ na coordenada $i$.
\end{definition}

\begin{theorem}[Extensão de Kolmogorov]
  \index{Teorema!da Extensao de Kolmogorov@da Extensão}
  \label{t:extens_kolmog}
  Seja para cada $n \geq 1$ uma medida de probabilidade $P_n$ em $\mathbb{R}^n$ tal que seja satisfeita a seguinte condição de compatibilidade \index{condicao de compatibilidade@condição de compatibilidade}
  \begin{equation}
    \label{e:consist_kolmog}
    P_{n+1} (A \times \mathbb{R}) = P_n (A), \text{ para todo $A \in \mathcal{B}(\mathbb{R}^n)$}.
  \end{equation}
  Então existe uma única probabilidade $P$ no espaço produto infinito $(\Omega, \mathcal{F})$ tal que $P(A \times \mathbb{R} \times \dots) = P_n (A)$ para todo $n$ e todo boreliano $A$ de $\mathbb{R}^n$.
\end{theorem}

\begin{proof}
  Considere a classe de conjuntos
  \begin{equation*}
    \mathcal{S}_l = \Big\{ \mcup_{j=1}^k [a_{1,j}, b_{1,j}) \times \dots \times [a_{l,j}, b_{l,j}) \subseteq \mathbb{R}^l \, : \,
    a_{i,j} \in \mathbb{R} \cup \{-\infty\},\ b_{i,j} \in \mathbb{R} \cup \{\infty\} \Big\}.
  \end{equation*}
  Que é obviamente uma álgebra em $\mathbb{R}^l$ e seja também
  \begin{equation}
    \mathcal{S} = \big\{ A \times \mathbb{R} \times \dots\, : \, \text{ onde } l \geq 1 \text{ e } A \in \mathcal{S}_l \big\}.
  \end{equation}
  Claramente, $\mathcal{S}$ também é uma álgebra.

  Se $B = A \times \mathbb{R} \times \dots \in \mathcal{S}$ com $A \in \mathcal{S}_l$ como acima, definimos
  \begin{equation}
    P(B) = P_l(A).
  \end{equation}
  Note que por \eqref{e:consist_kolmog} essa definição independe da escolha de $l$ que usamos na definição de $B$.

  Gostaríamos agora de utilizar o Lemma~\ref{l:extensao_vazio}.
  Para tanto, tome uma sequência encaixada $B_1 \supseteq B_2 \supseteq \dots \in \mathcal{S}$ e, supondo que $P(B_n) \geq \delta > 0$ para todo $n \geq 1$, temos de mostrar que sua interseção não pode ser vazia.

  Como $B_n \in \mathcal{S}$, podemos escrever
  \begin{equation}
    B_n = A_n \times \mathbb{R} \times \dots, \text{ onde $A_n \in \mathcal{S}_{l_n}$ e $n \geq 1$.}
  \end{equation}
  Podemos obviamente supor que
  \begin{equation}
    \label{e:l_n_monotona}
    \text{$l_n$ são estritamente crescentes.}
  \end{equation}

  A fim de obter um ponto na interseção de $B_n$, gostaríamos de aproximá-lo usando conjuntos compactos encaixados.
  Para tanto definimos os conjuntos
  \begin{equation}
    C_n = C_n^* \times \mathbb{R} \times \dots, \text{ com $C_n^* \in \mathcal{S}_{l_n}$}
  \end{equation}
  de forma que $C_n^*$ seja compacto, $C_n^* \subseteq A_n$ e
  \begin{equation}
    P(B_n \setminus C_n) \leq \frac{\delta}{2^{l_n + 1}},
  \end{equation}
  o que pode ser feito graças à continuidade de $P_{l_n}$, que é uma probabilidade.

  Temos ainda um problema, pois os conjuntos $C_n$ não são encaixados, e isso nos impede de utilizar resultados sobre interseções de compactos.
  Introduzimos pois $D_n = \bigcap_{i=1}^n C_i$, que obviamente pertence à álgebra $\mathcal{S}$, e estimamos
  \begin{equation}
    P(B_n \setminus D_n) = P \big( \mcup\nolimits_{i=1}^n (B_n \setminus C_i) \big) \leq \sum_{i=1}^n P(B_n \setminus C_i) \leq \frac{\delta}2,
  \end{equation}
  donde $P(D_n) = P(B_n) - P(B_n \setminus D_n) \geq \delta/2$.
  De forma que os $D_n$ são encaixados e não vazios.

  Nosso próximo obstáculo vem do fato de que os conjuntos $D_n$ estão definidos em $\mathbb{R}^\mathbb{N}$, e gostaríamos de ter conjuntos em espaços de dimensão finita.
  Isso pode ser feito observando que podemos escrever $D_n = D_n^* \times \mathbb{R} \times \mathbb{R} \times \dots$, onde $D_n^* \in \mathcal{S}_{l_n}$ e
  \begin{equation}
    D_n^* = \underbrace{C_n^*}_{\mathclap{\text{compacto}}} \mcap \underbrace{ \Big( \mcap_{i=1}^{n-1} C_i^* \times \mathbb{R}^{l_n - l_i} \Big)}_{\text{fechado}},
  \end{equation}
  de forma que os $D_n^* \subseteq \mathbb{R}^{l_n}$ são compactos  e não vazios.

  Para cada $n \geq 1$ considere um $\omega^n \in D_n$.
  Usando um argumento de diagonal de Cantor, podemos obter um $\omega \in \Omega$
  e uma sub-sequência de $\omega^{n_j}$ que convirja para $\omega \in \Omega$ coordenada a coordenada
  (observe que $\omega^{n_j} \in \mathbb{R}^{\smash{l_{n_j}}}$).  % Tomando subsequências se necessário, podemos supor que $\omega^n$ converge coordenada a coordenada a um certo $\omega \in \Omega$.
 Para concluir a prova mostramos que $\omega \in \bigcap_{n\ge 1} B_n$.
 Para isso e suficiente mostrar (lembramos que por definição $C_n \subseteq B_n$) que para todo $n\in \mathbb{N}$
  \begin{equation*}
\omega = (\omega_1, \omega_2, \dots) \in C_n.
  \end{equation*}
  O que e equivalente a $(\omega_1, \omega_2, \dots, \omega_n) \in C^*_n$, que vale por compacidade.
\end{proof}

Observe que usamos muito poucos atributos de $\mathbb{R}$ na prova.
Poderíamos na verdade substituir $\mathbb{R}$ por um espaço métrico que satisfaça certas propriedades, como por exemplo a existência de uma álgebra cujos conjuntos possam ser aproximados por compactos.
Contudo, decidimos não apresentar essa versão mais geral aqui porque muito em breve obteremos uma versão bem mais geral do Teorema de Kolmogorov usando apenas o resultado para $\mathbb{R}$.

\begin{exercise}
  Mostre que a hipótese \eqref{e:consist_kolmog} pode ser substituida por
  \begin{equation}
    P_{n+1} (I_1 \times \dots, \times I_n \times \mathbb{R}) = P_n (I_1 \times \dots \times I_n),
  \end{equation}
  para todo $n \geq 1$ e $I_i = (-\infty, b_i]$, onde $b_i \in \mathbb{R}$, $i \leq n$.
\end{exercise}

Um importante exemplo do uso deste teorema é o seguinte.

\begin{example}
  Se $P_i$ são probabilidades em $(\mathbb{R}, \mathcal{B}(\mathbb{R}))$, podemos definir $\mathbb{P}_n = \bigotimes_{i=1}^n P_i$ (relembrando, $\mathbb{P}_n$ é a única distribuição em $\mathbb{R}^n$ tal que $\mathbb{P}_n(A_1 \times \dots \times A_n) = \prod_{i=1}^n P_i(A_i)$).
  Não é difícil verificar que essa lei satisfaz as equações de consistência \eqref{e:consist_kolmog}.
  Desta forma, podemos construir uma única $\mathbb{P}$ em $\mathbb{R}^\mathbb{N}$ para os quais as coordenadas canônicas $X_i$ são independentes e possuem distribuições marginais $P_i$.
  Denotamos nesse caso $\mathbb{P} = \bigotimes_{i \geq 1} P_i$.
\end{example}

Mais adiante no texto daremos outros exemplos bastante interessantes do uso do Teorema~\ref{t:extens_kolmog}.

\begin{exercise}
  Mostre que se $p > 0$ e $\mathbb{P} = \bigotimes_{i \geq 1} \Ber(p)$ em $\mathbb{R}^\mathbb{N}$, então
  \begin{equation}
    \text{$\limsup_{n\to \infty} X_n = 1$ quase certamente.}
  \end{equation}
\end{exercise}

\begin{exercise}
  Mostre que se $\mathbb{P} = \bigotimes_{i \geq 1} U_{[0,1]}$ em $\mathbb{R}^\mathbb{N}$, então
  \begin{equation}
    \text{$\limsup_{n\to \infty} X_n = 1$ quase certamente.}
  \end{equation}
\end{exercise}

\begin{exercise}
  Mostre que se $\mathbb{P} = \bigotimes_{i \geq 1} \Exp(i)$ em $\mathbb{R}^\mathbb{N}$, então
  \begin{equation}
    \text{$\limsup_{n\to \infty} X_n < \infty$ quase certamente.}
  \end{equation}
\end{exercise}

\begin{topics}


\section{Construção de espaços produtos infinitos}


Jà vimos na secção anterior que para qualquer coleção finita  $\mu_1,\dots,\mu_n$ de probabilidades definidas em espaços
$E_1,\dots,E_n$, podemos construir um espaço de probabilidade $(\gO,\cF,P)$, tal que existem elementos aleatórios $X_i:\gO\to E_i$,
$i=1,\dots,n$ independentes, tais que $(X_i)_*P$ (a escolha canônica sendo de escolher o espaço produto, a medida produto e as coordenadas).

\medskip

Por essa razão, faz sentido de falar ``consideramos $X_1,\dots,X_n$ independentes de distribuição respetivas $\mu_i$ '', e 
no futuro faremos isso sem nem sempre cuidar do
espaço de probabilidade subjacente.

\medskip

Queremos obter a mesma liberdade para coleções infinitas de elementos aleatórios 
(a utilidade sendo de poder considerar, por exemplo limites de somas de variáveis independentes).

\medskip

Vamos considerar primeiro o caso de variáveis aleatórias, para qual existe uma construção simples.
Primeiro temos que introduzir o espaço e a $\sigma$-álgebra usados na definição de probabilidade produto.

\begin{definition}
Dada uma coleção de espaços $(E_n)_{n\ge 1}$, definimos o espaço produto
\begin{equation}
  \Omega = \prod_{n=1}^{\infty} E_n := \big\{\go=(\go_n)_{n\in \bbN} \, : \, \forall n\ge 1, \   \omega_n \in E_n \big\}.
\end{equation}
e os mapas $X_i:\Omega \to E_n$, definidos para $i\ge 1$ por
\begin{equation}
  X_i(\omega) = \omega_i,
\end{equation}
que chamamos de \emph{coordenadas canônicas} \index{coordenadas canonicas@coordenadas canônicas} associadas ao produto $\Omega$.\\
Se cada $E_i$ é dotado de uma $\sigma$-álgebra $\mathcal{A}_i$, então definimos
\begin{equation}
  \mathcal{F} := \sigma( (X_i)_{i\geq 1} ),
\end{equation}
que é a menor $\sigma$-álgebra em $\gO$ que deixa as applicações coordenadas mensúraveis.
Chamamos $\mathcal{F}$ de $\sigma$-álgebra canônica (o $\sigma$-álgebra produto), pode ser denotada informalmente como $\bigotimes_{n=1}^{\infty}\gO_i$.
\end{definition}

\begin{exercise}
  Mostre que em $(\mathbb{R}^{\mathbb{N}},\mathcal{F})$ temos que os conjuntos
  \begin{enumerate}[\quad a)]
  \item $A = \{ \liminf_{n\to \infty} X_n \notin \{\infty,-\infty\} \}$,
  \item $B = \{ \lim_{n\to \infty} X_n = 4\}$ e
  \item $C = \{ \lim_{n\to \infty} \tfrac{1}{n} X_n \text{ existe}\}$
  \end{enumerate}
  são todos mensuráveis (eventos) com respeito a $\mathcal{F}$.\\
  \noindent Mostre que $Y = f(\liminf_{n\to \infty} X_n)$ onde $f$ e uma função continua com limite en $\infty$ e $-\infty$ é uma variável aleatória em $(\Omega, \mathcal{F})$.
\end{exercise}

\begin{exercise}
  Verifique as seguinte afirmações
  \begin{enumerate}[\quad a)]
  \item $\mathcal{F} = \sigma\big(A_1 \times \dots \times A_k \times E_{k+1} \times E_{k+2} \times \dots\, : \, k \geq 1, A_i \in \mathcal{A}_i, i \leq k\big)$,
  os chamados eventos retangulares.
  \item $\mathcal{F} = \sigma\big(A \times E_{k+1} \times E_{k+2} \times \dots\, : \, k \geq 1, A \in \mathcal{A}_i \otimes \dots \otimes \mathcal{A}_k\big)$,
  conhecidos como eventos cilíndricos.
  \end{enumerate}
\end{exercise}



Com essa noção de $\sigma$-álgebra podemos introduzir a noção de medida produto
\begin{definition}
Dada uma coleção de espaços de probabilidade $(E_n,\cA_n,\mu_n)_{n\ge 1}$, e
sendo $(\gO,\cF)$ o espaço produto associado. Chamamos de probabilidade produto dos $(\mu_n)_{n\ge 1}$ em $\gO$ uma probabilidade $P$ que satisfaz
\begin{multline}
\forall n\in \bbN, \forall A_1\in \cA_1,\dots, A_n\in \cA_n, \\ P(\go_1\in A_1,\dots ,\go_n\in A_n)=\prod_{i=1}^n \mu_i(A_i).
\end{multline}
Isso e equivalente a dizer que as variáveis coordenadas $X_i(\go)$ são independentes e identicamente distribuídas.
Escrevemos $P=\bigotimes_{n\ge 1} \mu_i$.
Essa probabilidade sempre existe e é única.
\end{definition}

A prova da existência exige um trabalho considerável e deixamos ela para mais tarde.
Observe que a unicidade implica que a frase ``$X_n, \ n\ge 1$ e uma sequencia de variáveis independentes com distribuição respectivas $\mu_n, n\ge 1$''
descreve totalmente a distribuição imagem de sequência aleatória $(X_n)_{n\ge 1}$.

\begin{proof}
 Sabemos que $\cF$ e gerada pelos eventos cilíndricos que formam um $\pi$-sistema.
 Como, por definição, duas probabilidade produtos tem que coincidir neste $\pi$-sistema, podemos concluir usando 
 o Teorema $\pi$-$\gl$.
\end{proof}

\begin{exercise}\label{ex:duplo}
Sejam $(X_{n,m})_{n\ge 1,m\ge 1}$ uma sequência dupla de elementos aleatórios independentes.
Mostrar que a sequência de elementos $(Y_n)_{n\ge 1}$ definida por $Y_n:=(X_{n,m})_{m\ge 1}$ e independentes.
\end{exercise}

\end{topics}


\subsection{Construção de uma sequência de variáveis $\Ber(1/2)$ independentes}

 Consideramos $\gO=[0,1]$ e $P=U_{[0,1]}$.
 Para $\go\in[0,1]$, definimos $(X_n)_{n\ge 1}$ a sequencia de $\{0,1\}^{\bbN}$ tal que 
 
 \begin{equation}\label{eq:deco}
 \go:= \sum_{n=1} X_n(\go)2^{-n}.
 \end{equation}
 Se tiver mais que uma, escolhemos a maior pela ordem lexicográfica.

 \begin{proposition}

 As variáveis $(X_n)_{n\ge 1}$ são independentes, com distribuição $\Ber(1/2)$.
 
 \end{proposition}


 \begin{proof}
Verificamos que para qualquer $q\in \bbN$ e$\gep=(\gep_1,\dots,\gep_q)\in \{0,1\}^q$,
  temos
    \begin{equation}\label{bassico}
   P[X_{1}=\gep_1\ ;\dots; \ X_{q}=\gep_q]=2^{-q}.
  \end{equation}
Podemos mesmo verificar que dado uma família de eventos $A_i\subset \{0,1\}$, somando probabilidade usando união disjunta \eqref{bassico} tem como consequência 
   \begin{equation}
   P[X_{1}\in A_1 \ ;\dots; \ X_{q}\in A_q]=2^{-q} \prod_{i=1}^q \# A_q,
  \end{equation}
  o que permite identificar as distribuições marginais e provar independência.
Para mostrar \eqref{bassico} observamos que
  $$ \{ \go \ | \ X_{1}=\gep_1\ ;\dots; \ X_{q}=\gep_q \}= [x(q,\gep),x(q,\gep)+2^{-q}),$$
  onde $x(q,\gep):=\sum_{n=1}^q \gep_n 2^{-n}$, pois
a definição da medida de Lebesgue permite de concluir. 
  
  
  \end{proof}

  
  
  
  
 \subsection{Construção de uma sequência de variáveis independentes com distribuições marginais qualquer}

Seja $(\mu_n)_{n\ge 1}$ uma sequência de medida de probabilidade em $\bbR$ (com a $\sigma$ álgebra de Borel).
Queremos construir um espaço de probabilidade e uma sequência de variáveis independentes $X_n(\go)$ tal que que para todo $n\in\bbN$ 
$(X_n)_*P=\mu_n$.


\medskip

Definimos $\gO:=\{0,1\}^{\bbN}$ e $P$ sendo a medida definida na secção anterior para qual 
as coordenadas $Y_n(\go)$, $n\ge 1$  são Bernoulli independentes de parâmetro $1/2$.
Consideramos $m: \bbN\to \bbN^2$ a bijecção definida por 

$m(p,q):= \frac{(p+q)^2}{2}+\frac{p-q}{2}.$

e definimos 

$Z_n:=\sum_{q=1}^{\infty} 2^{-n} Y_{m(n,q)} $

\begin{lemma}
As variáveis aleatórias $(Z_n)_{n\ge 1}$ são independentes, de distribuição $U[0,1]$.
\end{lemma}
 
 \begin{proof}
Para identificar a distribuição de $Z_n$, observamos que por unicidade da probabilidade produto em $\{0,1\}^{\bbN}$, $Z_n$ tem que ter a mesma distribuição que 
$\go$ em \eqref{eq:deco}.
Segundo, observamos que pelo Exercício \ref{ex:duplo} as sequências $W_n:=(Y_{m(n,q)})_{q\ge 1}$.
Pelo Exercício \ref{ex:indepfun}, os $Z_n$ o são também.
\end{proof}


Para concluir, $F_n$ sendo a a função de distribuição associada a $\mu_n$ e 
$$S_n(u):=\sup\{\ x  \ : \ F_n(x)<u \}$$
definimos $X_n:=F_n(Z_n)$.

\begin{proposition}
 As variáveis $(X_n)_{n\ge 1}$ são independentes e 
 para cada $n$, $(X_n)_* P=\mu_n$.
 \end{proposition}

 \begin{proof}
  Independência vem do Exercício \ref{ex:indepfunc}. A distribuição das marginais segue da prova to Teorema \ref{t:existe_prob_R}.
  
 \end{proof}

 
 
\begin{exercise}
  Marcelo coleciona figurinhas de futebol.
  O álbum completo conter\'a $N$ figurinhas. No $i$-ésimo dia, ele compra uma nova carta $X_i \in \{1, \dots, N\}$.
  A cole\c{c}\~ao $(X_i)_{i \geq 0}$ é distribuída de maneira \iid e uniforme nas figurinhas.
  \begin{enumerate}[\quad a)]
  \item Para $j = 1, \dots, N$, seja $T_j$ o tempo passado até a aquisi\c{c}\~ao da $j$-ésima nova figurinha, i.e.
    \begin{equation}
      T_1 = 1 \quad \text{ e } \quad T_{j+1} = \inf\{i\ge T_{j}+1 \ : \  X_i\notin \{X_k\ : \ k\le T_j\} \}.
    \end{equation}
    Mostre que $T_j$ é finito quase certamente, para todo $j \leq N$.
  \item Calcule a distribuição conjunta de $(T_1, T_2 - T_1, \dots, T_N - T_{N-1})$.
  \item Calcule a esperança de $T_N$ (o dia em que Marcelo completa seu álbum).
  \end{enumerate}
\end{exercise}

\begin{exercise}
  Sejam $X_1, X_2, \dots$ variáveis aleatórias \iid e defina o primeiro tempo de recorde como
  \begin{equation}
    R = \inf\{i \geq 2 \ : \  X_i \geq X_1\}.
  \end{equation}
  Supondo que $P[X_i=x]=0$ para todo $x\in\bbR$, encontre $E(R)$.
\end{exercise}


\section{Lei \texorpdfstring{$\{0, 1\}$}{\{0,1\}} de Kolmogorov}

Ao estudarmos o Lema de Borel-Cantelli, vimos que se os eventos $(A_i)_{i \geq 1}$ são independentes então a probabilidade de $[A_i \text{ infinitas vezes}]$ somente pode assumir os valores zero ou um (dependendo da somabilidade de $P(A_i)$).
Nessa seção iremos estudar outros tipos de evento que assumem apenas esses dois valores.
Esperamos que esse fenômeno se torne intuitivo ao final dessa discussão.

No que se segue, consideraremos um espaço mensurável $\Omega = \prod_{i=1}^\infty E$, com a $\sigma$-álgebra canônica $\mathcal{F}$, 
isto é a $\sigma$-álgebra gerada pelas coordenadas canõnicas $(X_i)_{i=1}^\infty$.
\begin{definition}
  Dizemos que um evento $A \in \mathcal{F}$ é caudal se
  \begin{equation}
    A \in \sigma\big( X_i \ : \ i \geq n\big), \text{ para todo $n \geq 1$}.
  \end{equation}
  Também introduzimos a classe $\mathcal{F}_\infty$ de tais eventos, que claramente é uma $\sigma$-álgebra, pois pode ser escrita como
  \begin{equation}
    \mathcal{F}_\infty = \mcap_{n \geq 1} \sigma\big( X_i \ : \ i \geq n\big).
  \end{equation}
  Chamamos $\mathcal{F}_\infty$ de $\sigma$-álgebra caudal. \index{sigma-algebra@$\sigma$-álgebra!caudal}
\end{definition}

Vejamos que, dados $A_i \in \sigma(X_i)$, $i \geq 1$, temos que $[A_i \text{ infinitas vezes}]$ é caudal.
Para tanto, basta observar que para todo $n \geq 1$, temos que
\begin{equation*}
  [A_i \text{ infinitas vezes}] = \big[\#\{i \geq 1 \, : \, \omega \in A_i\} = \infty\big] = \big[\#\{i \geq n \, : \, \omega \in A_i\} = \infty\big],
\end{equation*}
que obviamente pertence a $\sigma(X_i\, : \, i \geq n)$ para todo $n \geq 1$.

\begin{exercise}
  Mostre que em $\Omega = \mathbb{R}^{\infty}$, são caudais os seguintes eventos
  \begin{enumerate}[\quad a)]
  \item $[X_i \text{ converge}]$,
  \item $\big[\tfrac{1}{n} \sum_{i=1}^n X_i \text{ converge}\big]$ e
  \item $[\#\{i \geq 1\, : \, X_i > 0\} < \infty]$.
  \end{enumerate}
\end{exercise}

Podemos agora enunciar o pricipal teorema dessa seção

\begin{theorem}[Lei $\{0,1\}$ de Kolmogorov]
  \index{Lei!0,1 de Kolmogorov@$\{0,1\}$ de Kolmogorov}
  Se $\Omega = E^{\bbN}$, onde $E$ é um espaço canônico, for provido de uma lei produto $P = \otimes_{i=1}^\infty P_i$, então todo evento caudal tem probabilidade $0$ ou $1$ sob $P$.
\end{theorem}

Quando uma $\sigma$-álgebra $\mathcal{F}$ satisfaz $P(A) \in \{0,1\}$ para todo $A \in \mathcal{F}$, dizemos que $\mathcal{F}$ é trivial. \index{sigma-algebra@$\sigma$-álgebra!trivial}
Uma outra maneira de enunciar a conclusão do teorema acima é dizer que a $\sigma$-álgebra caudal $\mathcal{F}_\infty$ é trivial.

\begin{proof}
  A idéia da prova, apesar de soar um pouco estranha, é mostrar que se $A \in \mathcal{F}_\infty$, então $A$ é independente de si mesmo.
  Em outras palavras, $P(A) = P(A \cap A) = P(A)^2$, donde $P(A) \in \{0,1\}$.
  Mas vamos com calma.

  Fixe $k \geq 1$, $A \in \mathcal{F}_\infty$ e $B \in \sigma(X_1, \dots, X_k)$.
  Nesse caso, como o evento $A$ pertence a $\sigma(X_{k+1}, X_{k+2}, \dots)$, temos que $A$ e $B$ são independentes.
  Fixe agora $A \in \mathcal{F}_\infty$ e considere a classe
  \begin{equation}
    \mathcal{B}_A = \{B \in \mathcal{F}\, : \, \text{ $B$ é independente de $A$}\}.
  \end{equation}
  Já sabemos que $\sigma(X_1, \dots, X_k) \subseteq \mathcal{B}_A$ para todo $k \geq 1$.

  Obviamente $\Omega$ é independente de $A$, assim como $B^c \in \mathcal{B}_A$ sempre que $B \in \mathcal{B}_A$.
  Além disso, suponha que $(B_i)_{i\in I}$ (finito o enumerável) in $\mathcal{B}_A$ são disjuntos, então,
  \begin{equation*}
    P\big( (\mcup_{i\in I} B_i) \cap A \big) = P\big( \mcup_{i\in I} (B_i \mcap A) \big) \overset{\text{disj.}}= \sum_{i\in I} P(B_i \mcap A) \overset{\text{indep.}}
    = P(A) P(\mcup_{i\in I} B_i).
  \end{equation*}
  Logo $\mathcal{B}_A$ é um $\lambda$-sistema.

  Lembrando que $\mathcal{B}_A$ contém o $\pi$-sistema $\bigcup_k \sigma(X_1, \dots, X_k)$, isto é dos eventos cilíndricos, temos que todos eventos são indepentes de $A$, inclusive o próprio $A$.
  Isso termina a prova do teorema.
\end{proof}

\begin{exercise}
  Dizemos que uma probabilidade $P$ no espaço produto $\Omega = E^{\bbN}$ (com a $\sigma$-álgebra canônica) é fortemente misturadora se, para todo $k \geq 1$, temos
  \begin{equation}
    \lim_{n \to \infty} \sup \{ \big| P(A \cap B) - P(A) P(B) \big| \, : \, A \in \sigma(X_1, \dots, X_k), \ B \in \sigma(X_n, X_{n+1}, \dots) \}= 0.
  \end{equation}
  Mostre que nesse caso, a $\sigma$-álgebra dos eventos caudais é trivial.
\end{exercise}


 \begin{topics}
\section{Tópico: Percolação}
\label{s:percolacao}

Imagine que gostaríamos de modelar o movimento de um líquido em um meio poroso, como uma rocha ou uma esponja.
A primeira tarefa nesse estudo seria modelar esse meio poroso de maneira matematicamente rigorosa, que é o que faremos a seguir.

Fixamos uma dimensão $d \geq 1$ e consideramos o seguinte grafo $(\mathbb{Z}^d, E)$, 
onde a rede quadrada $\mathbb{Z}^d$ é o conjunto de vértices e o conjunto de elos é dado por
\begin{equation*}
  E = \big\{ \{x, y\} \subset \mathbb{Z}^d \, : \,  |x - y| = 1 \},
\end{equation*}
onde $|\cdot|$ representa a distância euclideana em $\mathbb{R}^d$.

No nosso modelo, esse grafo pode ser entendido como um cristal periódico onde cada vértice representa uma cavidade do material poroso e os elos são potenciais conexões entre poros vizinhos.

Até agora nosso grafo é apenas uma rede periódica, mas as coisas começam a ficar interessantes à partir de agora.
Imaginamos que nosso material poroso está sujeito a variações durante sua formação.
Isso se reflete no fato que alguns elos de $E$ podem estar abertos ou não aleatoriamente.

Para o nosso modelos, o espaço amostral vai ser $\gO:= \{0,1\}^E$ considerado com a $\sigma$-algebra produto. 
Fixamos um $p \in [0,1]$ e definimos uma coleção de variáveis aleatórias $\go_e$, para $e \in E$, que sejam \iid e com distribuição $\Ber(p)$.
Chamamos $P_p$ a probabilidade corespondente.
Essas variáveis aleatórias induzem um grafo aleatorio $G(\go)=(\mathbb{Z}^d, \mathcal{E}(\go))$, subgrafo do grafo original,  
que corresponde a incluir apenas os elos $e$ com $\go_e = 1$.
Mais precisamente
\begin{equation}
  \mathcal{E}(\go) = \big\{ e \in E\, : \, \go_e = 1 \big\}.
\end{equation}
Podemos ver na Figura~\ref{f:percola} algumas simulações desse grafo aleatório.

\begin{figure}[!ht]
  \centering
  \begin{tikzpicture}[scale=.1]
    \ifdraft{\def\side{5}}{\def\side{30}}
    \draw[step=1, color=gray!50!white] (1,1) grid (\side + 1, \side + 1)
                                       (41, 1) grid (41 + \side, \side + 1) (81, 1) grid (81 + \side, \side + 1);
    \foreach \x in {1,...,\side}
    { \foreach \y in {1,...,\side}
      { \pgfmathrandominteger{\a}{0}{100} \ifthenelse{\a>60}{\draw[thick] (\x, \y) -- (\x, \y + 1);}{}
        \pgfmathrandominteger{\a}{0}{100} \ifthenelse{\a>60}{\draw[thick] (\x, \y) -- (\x + 1, \y);}{} } }
    \foreach \x in {1,...,\side}
    { \foreach \y in {1,...,\side}
      { \pgfmathrandominteger{\a}{0}{100} \ifthenelse{\a>50}{\draw[thick] (40 + \x, \y) -- (40 + \x, \y + 1);}{}
        \pgfmathrandominteger{\a}{0}{100} \ifthenelse{\a>50}{\draw[thick] (40 + \x, \y) -- (40 + \x + 1, \y);}{} } }
    \foreach \x in {1,...,\side}
    { \foreach \y in {1,...,\side}
      { \pgfmathrandominteger{\a}{0}{100} \ifthenelse{\a>40}{\draw[thick] (80 + \x, \y) -- (80 + \x, \y + 1);}{}
        \pgfmathrandominteger{\a}{0}{100} \ifthenelse{\a>40}{\draw[thick] (80 + \x, \y) -- (80 + \x + 1, \y);}{} } }
  \end{tikzpicture}
  \caption{Três simulações do grafo aleatório $(\mathbb{Z}^d, \mathcal{E})$, para valores de $p = 0,4$ (esquerda), $p = 0,5$ (centro) e $p = 0,6$ (direita). Tente imaginar como seria caminhar nesse grafo como se ele fosse um labirinto.}
  \label{f:percola}
\end{figure}

Agora que temos um modelo de meio poroso bem definido, precisamos pensar em quais perguntas nos interessam sobre $\mathcal{G} = (\mathbb{Z}^d, \mathcal{E})$.
Sendo esse um modelo por a passagem de fluido, as primeiras perguntas que faremos concerne a conectividade de $\mathcal{G}$.

\begin{exercise}
  Mostre que quase certamente $G(\go)$ é desconexo.
  Mais precisamente, mostre que existem quase certamente infinitos vértices isolados em $G(\go)$.
\end{exercise}

Como não podemos esperar que $G(\go)$ seja conexo, podemos nos perguntar algo mais fraco, como por exemplo se a 
componente conexa da origem $0 \in \mathbb{Z}^d$ em $G(\go)$ é infinita.

Voltando à Figura~\ref{f:percola} vemos que, dependendo do valor de $p \in [0,1]$, pode ser bem difícil ou bem fácil encontrar um caminho longo à partir da origem.
Isso é uo que estudaremos em mais detalhes no que segue.

Mais precisamente estamos interessados em:
\begin{equation}
  A = \big\{\omega \in \Omega \, : \,  \text{ a componente conexa de $0 \in \mathbb{Z}^d$ em $G(\go)$ é infinita} \big\}.
\end{equation}

Para estudar $A$, vamos fazer uma aproximação de $A$ por eventos mais simples
\begin{equation}
  A_n = \big\{ \omega \in \Omega \, : \, \text{ a componente conexa de $0$ sai da caixa $[-n, n]^d$}\},
\end{equation}
para $n \geq 1$.

\begin{exercise}
  Mostre que $A = \cap_{n=1}^n A_n$ e consequentemente que $A$ é de fato mensurável e $P(A) = \lim_{n\to \infty} P(A_n)$.
\end{exercise}

Definimos portanto a função $\theta:[0,1] \to [0,1]$ por
\begin{equation}
  \theta(p) = P_p(A),
\end{equation}
onde $P_p$ denota a probabilidade correspondente ao valor escolhido de $p \in [0,1]$.

\begin{exercise}
  Mostre que $\theta(p) \leq 1-(1-p)^{2d}$.
\end{exercise}

Nosso objetivo é entender algumas das propriedades de $\theta$.
A nossa intuição diz que quanto maior o valor de $p$, mais elos serão abertos em $\mathcal{G}$ e portanto maior será o valor de $\theta$, ou em outras palavras, $\theta$ deve ser monótona não decrescente.

\begin{exercise}
  Construiremos nosso modelo de uma maneira alternativa num espaço de probabilidade maior.
  Definimos $\gO_0:=[0,1]^E$ (com a $\sigma$-álgebra produto correspondente), e $(U_e)_{e\in E}$ 
  uma coleção de variáveis aleatórias \iid com distribuição $U[0,1]$, e  $\mathbb{P}$ a probabilidade corespondente.
  Definimos para cada $p \in [0,1]$, $X^p : \gO_0 \to \gO$ do jeito seguinte
  \begin{equation}
    X^p_e = \1_{[\go_e \leq p]}.
  \end{equation}
  Mostre que para todo $p \in [0,1]$  $(X^p)_*\bbP=P_p$.
  Use isso para concluir que $\theta$ é monótona não decrescente.
\end{exercise}

Iremos agora mostrar a existência de um regime para o qual a componente conexa da origem não é infinita.

\begin{theorem}
  Para $p < 1/(2d)$, temos que $\theta(p) = 0$.
\end{theorem}

Antes da prova, alguns exercícios.

\begin{exercise}
  Definimos um caminho como sendo uma sequência $x_1$, $\dots$, $x_k$ ($k \in \mathbb{N}$), tal que $\{x_i, x_{i+1}\} \in E$ para todo $i = 1, \dots, k-1$.
  Tal caminho é dito aberto se $\go_{\{x_i, x_{i+1}\}} = 1$ para todo $i \leq k-1$.
  E dizemos que ele é auto-evitante se $x_i \neq x_j$ para todo $1 \leq i < j < k$.
  Mostre que
  \begin{equation*}
    \begin{split}
      & A_n = \Big\{ \omega \in \Omega \, : \,  \text{ existe um caminho aberto $(x_i)_{i=1}^{k}$ com $x_1 = 0$ e $x_k \not \in [-n, n]^d$} \Big\}\\
      & A_n = \big\{ \omega \in \Omega \, : \, \text{ existe um caminho auto-evitante como acima} \big\}.
    \end{split}
  \end{equation*}
\end{exercise}

\begin{proof}
  Dado $p < 1/(2d)$ e $n \in \mathbb{N}$, lembramos que
  \begin{equation*}
    \begin{split}
      \theta(p) & \leq P_p(A_n) = P_p \Big[
      \begin{array}{c}
      \text{existe $k \in \mathbb{N}$ e um caminho auto-evitante $(x_i)_{i=1}^k$ }\\
      \text{aberto e com $x_1 = 0$ e $x_k \not \in [-n, n]^d$}
    \end{array} \Big]\\[2mm]
    & \leq \sum_{k \geq n} \; \; \sum_{(x_i)_{i=1}^k \text{ auto-evit.}} P_p [(x_i)_{i=1}^k \text{ aberto}] = \sum_{k \geq n} \; \; \sum_{(x_i)_{i=1}^k \text{ auto-evit.}} p^k\\
    & \leq \sum_{k \geq n} \; \; \sum_{(x_i)_{i=1}^k \text{ caminho}} P_p [(x_i)_{i=1}^k \text{ aberto}] = \sum_{k \geq n} (2d)^k p^k.
    \end{split}
  \end{equation*}
  Como $p < 1/(2d)$, a soma acima é finita e converge a zero quando $n$ diverge, provando o teorema.
\end{proof}

{\bf Notas} - O teorema acima ajuda a compreender o comportamento que observamos no lado esquerdo da Figura~\ref{f:percola}.
Mais precisamente, ele nos diz que para valores de $p$ baixos (na verdade $0,4$ não é baixo o suficiente para podermos aplicar esse teorema) é difícil encontrar um caminho aberto do centro à borda da caixa.

Na verdade, é possível mostrar que para $d = 2$,
\begin{equation}
  \begin{split}
    & \text{$\theta(p) = 0$ para todo $p \leq 1/2$ e}\\
    & \text{$\theta(p) > 0$ para todo $p > 1/2$,}
  \end{split}
\end{equation}
como foi mostrado por Harris e Kesten, veja por exemplo \cite{Gri99} e \cite{bollobas2006percolation}.
De fato, algo bastante interessante está acontecendo nesse modelo para $p = 1/2$, como nos mostrou o trabalho de grandes matemáticos, como: Oded Schramm, Wendelin Werner, Stanislav Smirnov, entre outros.

\todosec{Tópico: Teorema de Uma Série}{fazer...}





\end{topics}
