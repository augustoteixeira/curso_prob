% !TEX encoding = UTF-8 Unicode

\chapter{Somas de variáveis independentes I: Lei dos Grandes Números}

Nesse capítulo introduziremos várias técnicas e resultados que serão úteis em geral, mas que aparecem naturalmente no estudo da somas de variáveis aleatórias independentes, que por sua vez é um assunto de extrema importância em teoria e aplicações de probabilidade.

\section{Convergência de variáveis aleatórias}

Antes de enunciar qualquer teorema sobre convergência de variáveis aleatórias, temos que definir o que isso significa.

\medskip

E facil de se dar conta que o conjunto $H$ de variáveis aleatórias em um dado espaço de probabilidade e um $\bbR$ (o $\bbC$) espaço vectorial pelas operações usuais de soma e  multiplicação por um escalar.
Mas como jà reparamos que a maioria das propriedades que podemos provar so valem quase certamente, e mais razoável de trabalhar com o espaço quociente  $\cL^0=H/H_0$, onde $H_0$ e o conjunto das variáveis que valem $0$ quase certamente.

\medskip

Com um certo abuso de linguagem, 
agora quando falamos de variável aleatória consideramos um elemento do espaço $\cL^0$. 

\subsection{Espaços $\cL^p(P)$}

Dado $p\ge 1$ arbitrário, consideramos $\cL^p(P)$ o espaço quociente associado as variáveis tais que $|X|^p$ e integrável.

\medskip

Neste espaço reparamos que 

$$\| X\|_p:= \left(E\left [ |X|^p \right]\right)^{1/p}.$$ 
define uma norma (a desigualdade triangular e simplesmente a desigualdade de Minkovski, e a positividade segue da nossa operação quociente).

\medskip

\begin{definition}
Seja $(X_n)_{n\ge 1}$ uma sequencia de variáveis aleatórias.
Falaremos que uma sequência de variáveis $X_n$ converge para $X$
 em $\cL^p$ se $X_n\in \cL^p$ para todos $n\ge 1$ e que a sequencia converge pela topologia associada a norma $\| \cdot \|_p$, o de jeito equivalente se
 \begin{equation}
 \lim_{n\to \infty}  E[ |X_n-X|^p]=0.
 \end{equation}

\end{definition}


\begin{remark}
Do mesmo jeito podemos definir convergência em $\cL^{\infty}$ o espaço de variável associado a norma 
$$\| X\|_\infty:= \min\{ t \ : \  P[X>t]=0 \}$$ 
mas acaba sendo muito menos útil.
\end{remark}

\begin{exercise}
Se $X$ e uma variável positiva, mostra que $X\in \cL^p$ equivale a $\int x^{k-1}F_X(x) \dd x <\infty$, onde $X$ e a função acumulada de distribuição.
  Mostre uma fórmula análoga à da Proposição~\ref{p:espera_acumulada}.
\end{exercise}



\begin{exercise}
Seja $X$ uma variável aleatória e $q>p>1$ verificar que se $X\in \cL^q$ então 
$X\in \cL^p$ e $\|X\|_p\le \|X\|_q$.
Concluir que convergência em $\cL^p$ implica convergência em $\cL^q$.
\end{exercise}

\begin{exercise}\label{ex:l1}
Seja $(X_n)_{n\ge 1}$ e uma sequência que converge em $\cL^p$ para $p\ge 1$. 
Mostrar que $\lim_{n\to \infty} E[|X_n|^p]= E[|X|^p]$, 
e que $\lim_{n\to \infty} E[X_n]= E[X]$.

\end{exercise}

Essas convergências são uteis em pratica nas computações mas não são as mais naturais a considerar do ponto de vista probabilístico.

\subsection{Convergência em probabilidade}

\begin{definition}
Seja $(X_n)_{n\ge 1}$ uma sequencia de variáveis aleatórias.
Falaremos que uma sequência de variáveis $X_n$ converge para $X$ em probabilidade se para qualquer $\gep>0$

\begin{equation}
\lim_{n\to \infty} P[ |X_n-X|>\gep ]=0.
\end{equation}
\end{definition}


\begin{proposition}
A convergência em probabilidade é a convergência induzida pela métrica 
definida em $\cL^0$ for 
\begin{equation}\label{l0metric}
 d(X,Y)=E[\max(|X-Y|,1)]
 \end{equation}


\end{proposition}
\begin{proof}
Exercicio. 
\end{proof}


\begin{remark}
Nota que essa noção de convergência em $\cL^p$ para qualquer $p>1$ implica convergência em probabilidade. 
Pela desigualdade de Markov temos 

\begin{equation}
\lim_{n\to \infty} P[ |X_n-X|>\gep ]\le  \gep^{-p} P[ |X_n-X|^p]
\end{equation}

\end{remark}

\subsection{Convergência quase certamente}

Falamos que $X_n$ converge quase certamente par $X$ se 
existe $\bar \gO\subset \gO$ de probabilidade $1$ tal que para todos $\go\in \bar \gO$
\begin{equation}
\lim_{n\to \infty} X_n(\go)=X(\go).
\end{equation}
O limite e único em $\cL^0$.
Nota que convergência quase certamente de uma sequência implica convergência em probabilidade (com o mesmo limite).
Essa noção de convergência parece a mais natural mas vale a pena de reparar que ao contrario das outras, não corresponde a uma topologia.

\medskip

Isso e evidenciado pelo fato seguinte.

\begin{proposition}
Se uma sequência $X_n$ converge em probabilidade por $X$, então existe uma subsequencia que converge quase certamente.
\end{proposition}


\begin{proof}
Extraímos uma subsequência tal que 
$$P(|X_{n(k)}-X|\ge k^{-1})\le k^{-2}.$$
Pelo Lemma de Borel-Cantelli aplicado ao evento
$A_k= [|X_{n(k)}-X|\ge k^{-1}]$
podemos concluir que $A_k$ não acontece infinitas vezes, 
e então que com probabilidade $1$
$$\lim_{k\to \infty} X_{n(k)}(\go)=X(\go).$$
\end{proof}

\begin{remark}
 Observamos que uma consequência deste fato e que para usar teorema de convergência dominada, só precisaremos de 
 ter uma sequência convergindo em probabilidade.
 \end{remark}
 
 
 
\begin{exercise}
Seja $(X_n)_{n\ge 1}$ e uma sequência de variáveis em $\cL^p$ $(p\ge 1)$ que converge em probabilidade para 
$X\in \cL^p$. 
Mostrar que $\lim_{n\to \infty} E[|X_n|^p]\le  E[|X|^p]$. 
\end{exercise}


\subsection{Integrabilidade uniforme}

Vimos que convergência em $\cL^1$ implica convergência em probabilidade.
O contrario e claramente falso, como se pode ver com a sequência de variáveis
$$X_n:= n\1_{\{\go \in [0,n^{-1}]\}},$$
onde $\gO=[0,1]$ e $P=U([0,1])$ que converge quase certamente para $0$ mas não converge em $\cL^1$ (Por exemplo $E[|X_n|=1]$)

\medskip

A razão da ausência de convergência aqui e uma concentração da esperança de $|X_n|$  (poderia ser também só uma parte dela) 
em um evento cada vez menor do espaço de probabilidade ($[0,n^{-1}]$), que não aparece no limite. Em consequência 
temos uma parte da norma $\cL^1$ que se perde no limite 
(cf.\ Exercício \ref{ex:l1}) e temos $\lim_{n\to \infty} E[X_n]$).

\medskip

Uma sequência de variáveis em $\cL^1$ que não tem este tipo de comportamento se chama \textsl{uniformemente integral}.

\begin{definition}
 Uma coleção de variáveis aleatórias $(X_i)_{i\in \cI}$ e chamada uniformemente integrável se 
 para qualquer $\gep>0$, existe $M>0$ tal que 
 para todo $i\in \cI$
 \begin{equation}
  E[|X_{i}|\1_{[|X_i|>M]}]<\gep.
 \end{equation}
\end{definition}


\begin{example}
 \begin{itemize}
  \item [(a)] A coleção $\{X\}$ que contem só um elemento $X\in \cL^1$, é uniformemente integrável (e uma consequência trivial do teorema de convergência dominada).
  \item [(b)] Qualquer coleção finita de variáveis em $\cL^1$ é uniformemente integrável.
  \item [(c)] A bola unidade (o qualquer conjunto limitado) em $\cL^p$, $p>1$ e uniformemente integrável.
 \end{itemize}
\end{example}

Vamos agora ver uma caracterização alternativa da uniforme integrabilidade que justifica o nome é a seguinte.


\begin{proposition}
 Uma coleção de variáveis aleatórias $(X_i)_{i\in \cI}$  e uniformemente integrável se 
 e só se para qualquer $\gep$ existe $\delta$ tal que para todos eventos  $A$ com $P(A)<\delta$ e todo $i \in \cI$
  \begin{equation}\label{eq:lacond}
  E[|X_i|\1_{A}]<\gep.
 \end{equation}
 \end{proposition}

 \begin{proof}
 Se a sequência for uniformemente integrável,
 podemos achar $M>0$ tal que para todo $n$
 \begin{equation}
  E[|X_i|\1_{[|X_i|> M]}]<\gep/2.
 \end{equation}
Agora escolhemos $\delta=\gep/(2M)$.
Se  $P(A)<\delta$ temos 
\begin{multline}
 E[|X_i|\1_{A}]= E[|X_i|\1_{A\cap [|X_i|\le  M]}]+E[|X_i|\1_{A\cap [|X_i|>  M]}]\\
 \le M P(A\cap [|X_i|\le  M])+ E[|X_i|\1_{[|X_i|>  M]}]\le M\delta +\gep/2\le \gep.
\end{multline}
Reciprocamente se \eqref{eq:lacond} for satisfeito, então em particular
$$ C:=\sup_n E[|X_i|]<\infty,$$
e por desigualdade de Markov temos
\begin{equation*}
 P[|X_i|> M] \le \frac{C}{M}.
\end{equation*}
Então dado $\gep>0$, e $\delta$ tal que \eqref{eq:lacond} vale,
podemos concluir escolhendo $M=C\delta^{-1}$ que 
$E[|X_i|\1_{[|X_i|> M]}]\le \gep$.
\end{proof}

Essa noção permite de deduzir convergência em $\cL^1$ de convergência em probabilidade.

\begin{theorem}
 Seja $X_n$ uma sequência de variáveis aleatórias que converge em probabilidade para $X$.
 As seguintes propriedades são equivalentes
 \begin{itemize}
  \item [(i)] $X_n$ converge em $\cL^1$ para $X$.
  \item [(ii)] $(X_n)_{n\ge 1}$ e uniformemente integrável.
 \end{itemize}

 
 
 
\end{theorem}

\begin{proof}
 Mostramos primeiro que $(i)$ implica $(ii)$.
 Primeiro a convergência implica que dado $\gep>0$ podemos achar $N$ tal que $n\ge N$ implica 
 $$E[|X_n-X_N|]\le \gep/2.$$ 
 Agora, o conjunto $\{X_1,\dots,X_N\}$ sendo uniformemente integrável, existe $\delta$ tal que que 
 se $P(A)<\delta$
 $$ \forall n\in \{1,\dots,N\}, \quad E[|X_n|\1_{A}]\le \gep/2.$$
 Para $n\ge N$ e $P(A)<\delta$, temos 
 \begin{equation*}
 E[|X_n|\1_{A}]\le  E[|X_N|\1_{A}]+ E[|X_n-X_N|\1_{A}]\le \gep.
 \end{equation*}
 
 \medskip
 
 Agora provamos que $(ii)$ implica $(i)$.
Notamos se $(X_n)_{n\ge 1}$ e uniformemente integrável então $X\in \cL_1$.
Usando Fatou temos
\begin{equation}
 E[|X|]\le \liminf_{n\to \infty}E[|X_n|]<\infty
\end{equation}
Agora reparamos que
\begin{multline*}
E[|X_n-X|]= E[|X_n-X| \ind_{|X_n-X|\le \gep} ]+ E[|X_n-X| \ind_{ [|X_n-X|>\gep] } ]\\
\le 
E[|X_n| \ind_{ [|X_n-X|>\gep] } ]+E[|X|\ind_{ [|X_n-X|>\gep]}].
\end{multline*}
O primeiro termo e obviamente menor que $\gep$.
Agora por consequência da convergência em probabilidade, a probabilidade de $[|X_n-X|>\gep]$ vai para zero.
Por convergência dominada, isso implica que $E[|X|\ind_{ [|X_n-X|>\gep]}]\le \gep$ para $n$ grande suficiente.

\medskip


Agora usando integrabilidade uniforme, temos que se $P(A)<\delta$ para $\delta$ pequeno suficiente, $E[|X_n| \ind_{A} ]<\gep$, para qualquer $n$.
Como $P [|X_n-X|>\gep]<\delta$ para $n$ grande suficiente $E[|X_n| \ind_{ [|X_n-X|>\gep] } ]<\gep$ o que permite concluir.


 
\end{proof}




\begin{topics}
\section{Espaços $\cL^q$ são completos para $p\in \{0\}\in [1,\infty)$}.


\subsection{Completude de $\cL^0$}

\begin{theorem}
Dado $(\gO,\cF,P)$ um espaço de probabilidade. 
O espaço $\cL^0(P)$ equipado com a métrica  da convergência em probabilidade \eqref{l0metric} e completo. 
 
\end{theorem}


\begin{proof}


Seja $(X_n)_{n\ge 0}$ uma sequência de Cauchy em $\cL^0$. 
Podemos extrair ma subsequência $Y_k=X_{n(k)}$ tal que 
\begin{equation}
\sup_{m>k} d(Y_k,Y_{k+1})\le  4^{-k}.
\end{equation}
Em particular temos usando a desigualdade de Markov, temos
\begin{equation}
P(|Y_k-Y_{k+1}|\ge 2^{-k})\le 2^{-k}.
\end{equation}
Pelo Lemma de Borel-Cantelli, obtemos que a sequência 
$$Y_k=Y_1+\sum_{n=2}^k (Y_k-Y_{k-1}),$$
 converge q.c.\ Chamamos $X$ a variável obtida no limite (vista com elemento de $\cL^0$).
 
 \medskip
 
 Claramente $Y_k$ converge em probabilidade por $X$, mas usando a desigualdade triangular obtemos que 
 $$d(X_n,X)\le d(Y_{n},X_n)+d(Y_n,X).$$
 A sequência $(X_n)_{n\ge 1}$ sendo de Cauchy o primeiro converge para zero e concluímos que 
 $X_n$ converge para $X$ em probabilidade.
\end{proof}

\subsection{Completude de $\cL^p$, $p\ge 1$}
 
 
 \begin{theorem}
Dado $(\gO,\cF,P)$ um espaço de probabilidade. 
O espaço normado $\cL^p(P)$ \eqref{l0metric} e completo para $p\ge 1$. 
 
\end{theorem}

 \begin{proof}
 Seja $(X_n)_{n\ge 0}$ uma sequência de Cauchy em $\cL^q$. 
  Como no caso anterior só precisamos de achar uma subsequência que converge para poder concluir.
 Como $(X_n)_{n\ge 0}$ e de Cauchy em $\cL^p$, podemos extrair uma sequência $(Y_k)_{k\ge 1}$ que converge quase certamente por uma variável $X$.
 Extraindo de nove se for necessário, podemos assumir que  (com a convençao $Y_0=0$).
   $$\sum_{k\ge 1} \| Y_k-Y_{k-1} \|_{p}<\infty.$$
    Por convergência monótona e desigualdade de triangular podemos concluir que 
 $Z=\sum_{k\ge 1} |Y_k-Y_{k-1}|$ e um elemento de $\cL^p$.
 
 \medskip
 
 Agora podemos reparar que $Z^p$ domina a sequência $(Y^p_n)_{n\ge 1}$. 
 Pois por convergência dominada podemos concluir que  $X=\sum_{k\ge 1} Y_k-Y_{k-1}$ pertencia a $\cL^p$ e que $E[|X-X_k|^p]$ converge para $0$.
 \end{proof}

 \begin{remark}
  Não faláramos nada a respeito do caso $\cL^{\infty}$  mas a demostração e muito mais fácil neste caso.
  Pode se verificar em particular que convergência em $\cL^{\infty}$ implica convergência quase certamente.
 \end{remark}


\subsection{Observações sob compacidade em espaços $\cL^p$}

\begin{proposition}\label{prop:laug}
 Seja $1 \le p<q$  $\cA\subset \cL^q(P)$, então se $\cA$ e pre-compacto (o fecho topológico e compacto) em $\cL^0(P)$ e 
 limitado em $\cL^q(P)$, então e pre-compacto em $\cL^p(P)$.
 \end{proposition}
Uma consequência importante deste resultado e a seguinte
\begin{corollary}
 Se $(X_n)_{n\ge 1}$ converge em $\cL^0(P)$ e é limitada em $\cL^q(P)$, $q>1$ então 
ela converge (pelo mesmo limite) em $\cL^p(P)$ para $p\in [1,q)$.
 \end{corollary}


 \begin{proof}
Por $\cA$ ser pre-compacto em $\cL^0$, toda sequência $(X_n)_{n\ge 1}$ em $A$ admite uma subsequência  $(Y_k)_{k\ge 1}$ convergente em $\cL^0$.
Para concluir temos que verificar que a convergência ocorre também em $\cL^p$. Por isso, e suficiente de mostrar que a sequência é de Cauchy.
Consideramos $\gep>0$ arbitrário.
Temos para qualquer $K>0$
\begin{equation}
E\left[ |Y_k-Y_l |^p \right] = E\left[ |Y_k-Y_l|^p \1_{\{|Y_k-Y_l |^p \le K\}} \right] +  
E\left[ |Y_k-Y_l |^p \1_{\{|Y_k-Y_l |^p \ge K\}} \right].
\end{equation}
 Pelo segundo termo observamos que 
 \begin{equation}
 E\left[ |Y_k-Y_l|^p \1_{\{|Y_k-Y_l|^p \ge K\}} \right]\le K^{p-q} E\left[ |Y_k-Y_l|^q \right]\le 2^q K^{p-q} \max_{n\ge 1} E[|X_n|^q]
 \end{equation}
 Se $K$ for grande suficiente, este termo e menos que $\gep/2$.
 
Por convergência dominada, se $k$ e $l$ forem grande suficiente (de um jeito que depende de $K$), temos 
\begin{equation}
 E\left[ |Y_k-Y_l|^p \1_{\{|Y_k-Y_l|^p \le K\}} \right]\le \gep/2.
 \end{equation}

 \end{proof}
 
 
\end{topics}




\todosec{Tópico: Grafos Aleatórios}{fazer erdos renyi...}

\todosec{Tópico: Curie Weiss}{fazer...}


\section{Variância}

Usando desigualdade de Markov podemos reparar que para qualquer $X\in \cL^p$

\begin{equation}
  P[|X| \geq x] = P [X^p \geq x^p] \leq \frac{E(|X|^p)}{x^p}, \text{ para quaisquer $k \geq 1$.}
\end{equation}
Um caso particularmente útil para aplicar esse tipo de desigualdade e o caso $\cL^2$.
Isso por que  $\cL^2$ e equipado de uma estrutura natural de espaço de Hilbert, pois por que neste espaço, 
independência e relacionado a ortogonalidade.

\subsection{Definição}

Digamos que estamos interessados em aproximar uma variável aleatória por uma constante de forma a minimizar o erro da aproximação.
Uma possível formulação desse problema é encontrar $a$ de forma a minimizar
\begin{equation}
  \label{e:EX_aproxima}
  E\Big( (X - a)^2 \Big) = E(X^2) - 2 a E(X) + a^2.
\end{equation}
Essa equação obviamente possui um único mínimo em $a = E(X)$.
Ao erro da aproximação acima damos o nome de variância.
Corresponde a distancia quadrada ao subespaço das constantes no espaço $\cL^2$.


\begin{definition}
  Dada uma variável aleatória $X \in \mathcal{L}^2$, definimos sua variância \index{variancia@variância} como
  \begin{equation}
    \Var(X) = E \Big( \big(X - E(X)\big)^2 \Big) = E(X^2) - E(X)^2.
  \end{equation}
\end{definition}


Podemos alternativamente entender a variância da seguinte maneira.
Sejam $X$ e $Y$ variáveis aleatórias independentes em $\mathcal{L}^2$ de mesma distribuição.
Então,
\begin{equation}
  \frac{1}{2} E\big( (X - Y)^2 \big) = \frac{1}{2}\left[  E(X^2) - 2 E(XY) + E(X^2)\right] = E(X^2) - E(X)^2 = \Var(X).
\end{equation}


\begin{proposition}[Propriedades básica da variança]
Observe pelas definições alternativas dadas acima que
\begin{enumerate}[\quad a)]
\item $\Var(X) \geq 0$.
\item Se $a, b\in \bbR$ $\Var (aX+b)=a^2 \Var(X).$  
\item Mostre que se $X \in \cL^2$, então $\Var(X) = 0$ se e somente se $X = a$ quase certamente.
\end{enumerate}
\end{proposition}

\begin{proof}
 Exercício.
\end{proof}


\begin{exercise}
  Calcule $\Var(X)$ quando $X$ tem distribuições $\Ber(p)$, $U[0,1]$ ou $\Exp(\lambda)$.
\end{exercise}

A seguinte aplicação da propridade de Markov usa a variância para  estimar o quanto uma variável aleatória se desvia de sua média.
\begin{proposition}[Desigualdade de Chebychev]
  Se $X \in \mathcal{L}^2$ e $a > 0$, então
  \begin{equation}
    P [ |X - E(X)| > a] \leq \frac{\Var(X)}{a^2}.
  \end{equation}
\end{proposition}

\begin{proof}
  A desigualdade segue trivialmente da cota de Markov, ao observarmos que
  \begin{enumerate}[\quad a)]
  \item $|X - E(X)| \geq 0$.
  \item $|X - E(X)| > a$ se e somente se $|X - E(X)|^2 > a^2$.
  \item $E\big(|X - E(X)|^2\big) = E\big((X - E(X))^2\big) = \Var(X)$.
  \end{enumerate}
  Isso demostra a proposição.
\end{proof}

\subsection{Variança, Soma e Independência }
Para variáveis aleatórias de média zero, a variância nada mais é que $E(X^2)$, ou em outras palavras $\lVert X \rVert^2_2$, o quadrado de sua norma em $\mathcal{L}^2$.
Isso nos motiva a olhar mais de perto para o produto interno em $\mathcal{L}^2$, que se traduz a $E(XY)$.
Mas para não nos restringirmos a variáveis de média zero, introduzimos a seguinte noção.

\begin{definition}
  Se $X, Y$ são variáveis em $\mathcal{L}^2$, definimos
  \begin{equation}
    \Cov(X,Y) = E\Big( \big(X - E(X)\big) \big(Y - E(Y)\big) \Big) = E(XY) - E(X)E(Y).
  \end{equation}
\end{definition}

Uma observação importante e a seguinte.
\begin{proposition}
  Se $X$ e $Y$ em $\mathcal{L}^2$ são independentes, então $\Cov(X,Y) = 0$.
\end{proposition}

\begin{proof}
 É uma consequência da Proposição \ref{prop:indep}.
\end{proof}


\begin{exercise}
  Sejam $X_1$ e $X_2$ as coordenadas canônicas em $\mathbb{R}^2$.
  Já vimos que elas não são independentes sob a distribuição $U_{S^1}$.
  Mostre que mesmo assim temos $\Cov(X_1, X_2) = 0$.
\end{exercise}

Considerando a covariança como um produto interno, podemos deduzir a seguinte formula para variança da soma.

\begin{proposition}
  Se $X_1, \dots, X_n$ são variáveis em $\mathcal{L}^2$, então
  \begin{equation}
    \Var(X_1 + \dots + X_n) = \sum_{i=1}^n \Var(X_i) + 2\sum_{i < j} \Cov(X_i, X_j).
  \end{equation}
  Em particular, se as variáveis $X_i$ forem independentes duas a duas, então
  \begin{equation}
    \label{e:var_linear}
    \Var(X_1 + \dots + X_n) = \sum_{i=1}^n \Var(X_i).
  \end{equation}
\end{proposition}


\begin{proof}
  Basta fazer o tedioso desenvolvimento
  \begin{equation}
    \begin{split}
      \Var\Big( \sum_{i=1}^n X_i\Big) & = E \Big( \Big( \sum_{i=1}^n X_i - E\Big( \sum_{i=1}^n X_i\Big)\Big)^2\Big)\\
      & = E \Big( \Big( \sum_{i=1}^n X_i - E(X_i)\Big)^2\Big)\\
      & = \sum_{i, j = 1}^n E \big(X_i - E(X_i)\big) E\big(X_j - E(X_j)\big),
    \end{split}
  \end{equation}
  o que termina a prova ao separarmos $i = j$ de $i \neq j$.
\end{proof}


\begin{exercise}
  Calcule $\Var(X)$ quando $X \overset{d}\sim \Bin(n, p)$.
\end{exercise}

\begin{exercise}
  Calcule $E(X)$ quando $X \overset{d}\sim \Geo(p)$.
\end{exercise}

Um dito popular muito comum no Brasil é que não devemos deixar todos os ``ovos no mesmo cesto'', o que nos remete à possibilidade de perdermos todos eles caso o cesto caia.
Uma outra maneira de pensar nas vantagens de se dividir nossos riscos entre várias fontes independentes de incerteza, vem da equação \eqref{e:var_linear}, melhor explicada no exercício abaixo.

\begin{exercise}
  Imagine que $X_1, \dots, X_n$ são variáveis \iid, tomando valores em $[0,1]$ e que temos um certo valor $s \in \mathbb{R}_+$ que temos que guardar em $n$ caixas (dividindo como quisermos em $s_1, \dots, s_n$).
  Ao fim da semana, obteremos $S =  \sum_{i=1}^n s_i X_i$.

  Calcule $E(S)$ e $\Var(S)$ nos dois casos seguinte:
  \begin{enumerate}[\quad a)]
  \item $s_1 = s$ e $s_i = 0$ para todo $i \geq 2$.
  \item $s_i = s/n$ para todo $i$.
  \end{enumerate}
  Compare os resultados.
\end{exercise}

\begin{exercise}
  Calcule $\lim_{p \to 0} F_p(x)$ onde $F_p$ é a função de distribuição acumulada de $p X_p$ com $X_p \overset{d}\sim \Geo(p)$.
  Você reconhece esse limite?
\end{exercise}

\section{Lei fraca dos grandes números}

Nessa seção iremos mostrar um dos resultados mais importantes da Teoria da Probabilidade.
O que nossa intuição tem a nos dizer sobre a probabilidade de obtermos um resultado em um dado é $1/6$?
Uma possível explicação seria por simetria, mas e o que podemos dizer no caso de um dado viciado?

Se dizemos a alguém que a probabilidade de obter $6$ em um certo dado é $1/10$, naturalmente a pessoa pode se perguntar como descobrimos isso.
Um bom jeito de obter tal medida seria jogar o dado várias vezes independentemente e calcular em qual proporção dos ensaios ele retornou um seis.

O objetivo desta seção é confirmar a validade desse experimento de maneira quantitativa.

\begin{theorem}
  \index{Lei!Fraca dos Grandes Numeros@Fraca dos Grandes Números}
  \label{t:lei_fraca}
  Considerando $X_1, X_2, \dots$ são i.i.d.s em $\mathcal{L}^2(P)$, definimos
  \begin{equation}
    S_n = \sum_{i=1}^n X_i.
  \end{equation}
  Para todo $\varepsilon > 0$
  \begin{equation}
    \lim_{n \to \infty} P \Big[\Big| \frac{S_n}{n} - E(X_1)\Big| > \varepsilon \Big] = 0.
  \end{equation}
  Ou seja, $\tfrac{S_n}{n} \to E(X_1)$ em probabilidade.
\end{theorem}



\begin{proof}
  Sabemos que
  \begin{equation}
    P \Big[\Big| \frac{S_n}{n} - E(X_1)\Big| > \varepsilon \Big] \leq \frac{\Var(\tfrac{S_n}{n})}{\varepsilon^2},
  \end{equation}
  pois $E(S_n/n) = (1/n) E(X_1 + \dots + X_n) = E(X_1)$.

  Mas como $\Var(S_n/n) = (1/n^2) \Var (X_1 + \dots + X_n) = (n/n^2) \Var(X_1)$, temos o resultado.
\end{proof}

\begin{remark}
 Observe que nós apenas utilizamos que as variáveis $X_i$ eram independentes duas a duas.
 Isso e suficiente para mostrar que $S_n/n$ converge em $\cL^2$ e em consequência converge em probabilidade.
\end{remark}



Além disso, obtivemos o seguinte resultado quantitativo que vale mesmo para valores finitos de $n$:

\begin{scholia}
  Se $X_1, X_2, \dots$ são i.i.d.s em $\mathcal{L}^2$ e definimos $S_n = \sum_{i=1}^n X_i$ como acima, então, para todo $\varepsilon > 0$ e $n \geq 1$, temos
  \begin{equation}
    P \Big[\Big| \frac{S_n}{n} - E(X_1)\Big| > \varepsilon \Big] \leq \frac{\Var(X_1)}{\varepsilon^2 n}.
  \end{equation}
\end{scholia}





\begin{corollary}
  Se $A_1, A_2, \dots$ são eventos independentes dois a dois com $P(A_i) = p \in [0,1]$ para todo $i$, então
  \begin{equation}
    \lim_{n \to \infty} P \Big[ \Big| \frac{\#\{i \leq n; \omega \in A_i\}}{n} - p \Big| > \varepsilon \Big] = 0,
  \end{equation}
  ou em outras palavras a proporção de ensaios onde o evento $A_i$ ocorre converge em probabilidade para $p$.
\end{corollary}

\begin{proof}
  Basta tomar $X_i = \1_{A_i}$ no Teorema~\ref{t:lei_fraca}.
\end{proof}

\begin{exercise}
  Sejam $(X_i)_{i \geq 1}$ variáveis \iid com distribui\c{c}\~ao Ber$(p)$, $p \in [0,1]$. Mostre que
  \begin{equation}
    \lim_{N \to \infty} \frac 1N \sum_{i = 1}^N X_i X_{i+1} = p^2, \text{ em probabilidade.}
  \end{equation}
\end{exercise}

\begin{exercise}
  Sejam $X_1, \dots, X_n$ e $Y_1, \dots, Y_n$ variáveis independentes com distribuição $\Ber(p)$.
  Defina agora $Z_{i,j} = X_i Y_j$, para $i, j \in \{1, \dots, n\}$.
  \begin{enumerate}[\quad a)]
  \item Calcule a esperança de $S_n = \tfrac{1}{n^2} \sum_{i=1}^n \sum_{j=1}^n Z_{i,j}$.
  \item Estime $P[|S_n - E(S_n)| > a]$ usando o método do segundo momento. Como esse resultado se compara com o caso em que os $Z_{i,j}$ são i.i.d.?
  \end{enumerate}
\end{exercise}

\begin{exercise}
  \label{x:casas_tempestade}
  Considere uma rua infinita com casas $i \in \mathbb{Z}$.
  Para todo $i \in \mathbb{Z}$, existia uma rua entre as casas $i$ e $i+1$, mas após uma grande tempestade essas ruas foram danificadas.
  Mais precisamente, para cada $i \in \mathbb{Z}$, temos variáveis aleatórias $X_i$ que são i.i.d. com distribuição $\text{Ber}(p)$, onde $X_i = 1$ indica que o trecho da rua entre as casas $i$ e $i + 1$ foi danificado e não pode ser utilizado.
  Defina, para $i \in \mathbb{Z}$, $R_i$ como sendo o número de casas que continuaram acessíveis à casa $i$ após a tempestade.
  Por exemplo, se $X_{-2}$ e $X_0 = 1$ e $X_{-1} = 0$, temos que a casa $0$ somente pode acessar a casa $-1$, logo $R_0 = 1$.
  Nesse contexto,
  \begin{enumerate}[\quad a)]
  \item Calcule a distribuição e a esperança de $R_0$,
  \item Use o método do segundo momento para estimar a probabilidade
    \begin{equation}
      P \Big[ \Big| \frac{1}{n} \sum_{i=1}^n R_i - E(R_0) \Big| > a \Big].
    \end{equation}
  \end{enumerate}
\end{exercise}





\begin{topics}

\section{Tópico: Contando triângulos}

Vimos como a Lei Fraca dos Grandes Números seguiu de uma estimativa de segundo momento \index{momento!segundo} (mais precisamente usando a variância).

Nessa seção iremos mostrar como esse método é mais geral, se aplicando mesmo em situações onde as variáveis não são necessariamente independentes duas a duas.

Seja $V_n = \{1, \dots, n\}$ com $n \geq 3$ e $\mathcal{E}_n = \big\{ \{x,y\} \subseteq V_n; x \neq y \big\}$.
Chamamos o par $(V_n, \mathcal{E}_n)$ de grafo completo em $n$ vértices.

Definimos em um certo espaço de probabilidade $P_n$, as variáveis aleatórias $(X_e)_{e \in \mathcal{E}_n}$ de maneira \iid com distribuição $\Ber(p)$, onde $p \in [0,1]$.
Essas variáveis induzem um subgrafo aleatório $(V_n, \mathcal{E}_n')$, onde
\begin{equation}
  \mathcal{E}_n' = \big\{ e \in \mathcal{E}_n\ : \  X_e = 1 \big\}.
\end{equation}
Dizemos que os elos $e$, tais que $X_e = 1$ são abertos.

Definimos nesse espaço a variável aleatória
\begin{equation}
  T_n = \#\big\{\text{triângulos em $(V_n, \mathcal{E}_n')$}\big\}.
\end{equation}
Essa variável claramente pode ser escrita como
\begin{equation}
  T_n = \sum_{x,y,z \in V_n \text{ distintos}} \1_{A_{\{x,y,z\}}},
\end{equation}
onde $A_{\{x,y,z\}} = \big[\text{\{x,y,z\} formam um triângulo em $(V_n, \mathcal{E}_n')$}\big]$.

Gostaríamos de entender algo sobre a distribuição de $T_n$ e começamos calculando
\begin{equation}
  \begin{split}
    E^n(T_n) & = \sum_{\{x,y,z\} \text{ distintos}} P^n(A_{\{x,y,z\}})\\
    & = \binom{n}{3} p^3 = \frac{n(n-1)(n-2)}{6}p^3.
  \end{split}
\end{equation}
Logo, $P[T_n > a] \leq n(n-1)(n-2)p^3/6a$.
Mais ainda,
\begin{equation}
  \begin{split}
    E^n(T_n^2) & = \sum_{\{x,y,z\} \text{ distintos}} \quad \sum_{\{x',y',z'\} \text{ distintos}} P^n(A_{\{x,y,z\}} \cap A_{\{x',y',z'\}})\\
    & = \underbrace{\binom{n}{6} \binom{6}{3} p^6}_{\text{todos distintos}} + \underbrace{\binom{n}{5} \binom{5}{3} \binom{3}{1} p^6}_{\text{$1$-comum}} + \underbrace{\binom{n}{4} \binom{3}{2} \binom{4}{3} p^5}_{\text{$2$ em comum}} + \underbrace{\binom{n}{3}p^3}_{\text{iguais}}
  \end{split}
\end{equation}
Donde
\begin{equation}
  \Var^n(T_n) = \frac{1}{36} n^6 p^6 - \frac{1}{36} n^6 p^6 + c n^5 p^5 + ... \leq c (n^5 p^5 + n^3 p^3),
\end{equation}
para todos $p \in [0,1]$ e $n \geq 1$ se escolhemos bem a constante $c > 0$.

Isso nos permite por exemplo estimar o que acontece em alguns regimes, como por exemplo, se $p = 1/2$, então
\begin{equation}
  E^n(T_n) = \frac{n(n-1)(n-2)}{48},
\end{equation}
que cresce como $n^3$, e $\Var^n(T_n) \leq c n^5$, logo
\begin{equation}
  P^n\Big[ \Big|T_n - E^n(T_n)\Big| > \varepsilon n^3 \Big] \leq \frac{\Var^n(T_n)}{\varepsilon^2 n^6} \leq \frac{c}{\varepsilon^2 n}.
\end{equation}

\end{topics}

\todosec{Tópico: Análise de DNA}{fazer "computational molecular biology" - Pevzner seção 5.5...}

\todosec{Tópico: Método Probabilístico Revisitado}{usando segundo momento agora}





\section{Lei forte dos grandes números}

\begin{theorem}[Lei Forte dos Grandes Números]
  \index{Lei!Forte dos Grandes Numeros@Forte dos Grandes Números}
  \label{t:LFGN}
  Sejam $X_1, X_2, \dots$ \iid em $\mathcal{L}^1$, com $m = E(X_1)$.
  Então,
  \begin{equation}
    \lim_{n \to \infty} \frac{1}{n} \sum_{i=1}^n X_n = m, \text{ $P$-quase certamente.}
  \end{equation}
\end{theorem}

Começamos com uma prova do teorema no caso simples onde as variáveis estão em $\mathcal L^4$, que providencia um exemplo de uso do metodo dos momentos.

\begin{proof}[Demostração no caso $\mathcal L^4$]
Podemos supor que $E(X_i)=0$ (se não for o caso, considera $Y_i=X_i-E(X_i)$).
Podemos calcular o momento de ordem $4$ da soma expandido ela 
Temos 

\begin{equation*}
  E \left[ \left(\frac{1}{n} \sum_{i=1}^n X_n \right)^4 \right]= \frac{1}{n^4} \sum_{i_1,i_2,i_3,i_4=1}^n E[X_{i_1}X_{i_2}X_{i_3}X_{i_4}].
\end{equation*}
Pois usando independência, e $E(X_i)=0$ vemos que so os termos onde os $X_{k}$ apparecem pelo menos duas vezes não são nulos e
então que a soma acima vale
$$\frac{1}{n^4}\left( n E(X^4_1)-3n(n-1) E(X^2_1) \right)\le \frac{C}{n^2}.$$
Então temos 
\begin{equation}
 \sum_{n=1}^{\infty} E \left[ \left(\frac{1}{n} \sum_{i=1}^n X_n \right)^4 \right]<\infty.
\end{equation}
Usando Fubini e equivalente a 
\begin{equation}
 E \left[  \sum_{n=1}^{\infty} \left(\frac{1}{n} \sum_{i=1}^n X_n \right)^4 \right]<\infty.
\end{equation}
E em particular que quase certamente
\begin{equation}
 \sum_{n=1}^{\infty} \left(\frac{1}{n} \sum_{i=1}^n X_n \right)^4<\infty.
\end{equation}

\end{proof}


\begin{remark}
 Se as variáveis são positivas quase-certamente, então, usando o resultado para $\min(X_n,M)$ e o limite $M\to \infty$ 
 podemos concluir que $n^{-1}\sum_{i=1}^n X_i$ vai para infinito quase certamente.
 \end{remark}

 Se as variáveis não são integráveis, então podemos concluir que 
 $n^{-1}\sum_{i=1}^n X_i$ não converge quase certamente. 
 E uma consequência do resultado seguinte
 
 \begin{proposition}
  
 Se $(X_n)_{n\ge 1}$ e uma sequência de variáveis IID com $X_1\notin \cL_1$ então quase-certamente
 \begin{equation}
  \limsup_{n\to \infty} \frac{|X_n|}{n}=\infty.
 \end{equation}

  
 \end{proposition}
 
 \begin{proof}
 Temos para qualquer $M>0$
 $$E[|X_1|]=\int_0^{\infty} P[X_1\ge x] \dd x \le  M\left(1+ \sum_{n\ge 1} P[X_1\ge Mn]\right).$$
  Em particular se $X_1\notin \cL_1$ a soma a direita e infinita.
  
  \medskip
  
  Agora definimos o evento $A_n:= [X_n\ge nM]$.  Os $A_n$ são independente e 
 $$\sum_{n\ge 1} P(A_n) = \sum_{n\ge 1} P[X_1\ge Mn]=\infty.$$
 Como temos
 $$ \left\{ \limsup_{n\to \infty} \frac{X_n}{n}\ge M \right\} \supset \left\{  \bigcap_{k\ge 1} \bigcup_{n\ge k} A_n \right\},$$
podemos concluir usando o Lema de Borel-Cantelli.
$$ P\left[ \limsup_{n\to \infty} \frac{X_n}{n}\ge M\right]\ge P \left[ \bigcap_{k\ge 1} \bigcup_{n\ge k} A_n  \right] =1.$$
Como $M$ e arbitrário podemos concluir.

 \end{proof}






\subsection{Demostração da Lei Fortes dos Grandes Números}

Vamos provar que para todo $a>E[X_1]$ temos quase certamente
\begin{equation}\label{lesup}
M:=\sup_{n\ge 0} S_n-n a <\infty
\end{equation}
onde $S_n:=\sum_{i=1}^n X_i$, $S_0=0$.

\medskip

A desigualdade \eqref{lesup} implica que 
\begin{equation*}
 \limsup_{n\ge 0} \frac{S_n}{n}\le a,
\end{equation*}
e como $a$ pode ser arbitrariamente perto de $E[X_1]$,  $\limsup_{n\ge 0} \frac{S_n}{n}\le E[X_1]$.
A cota inferior pode ser obtido usando a desigualdade \eqref{lesup} para $-S_n$ (que também e uma soma de variáveis IID).


\medskip

Agora consideramos $a$ fixo.
Uma primeira coisa que podemos verificar e que coincide para qualquer $k\in \bbN$ temos
$$\{M<\infty\}=\{ \sup_{n\ge k} S_n-n a<\infty \}= \{\sup_{n\ge k} S_n-S_k-na \}$$
Em particular $\{M<\infty\}\in \sigma(X_{k+1},X_{k+2},\dots )$ para $k$ arbitrário, e por entanto pertencia na $\sigma$ álgebra caudal.
Em consequência da lei do $\{0,1\}$, temos $P[M<\infty]\in \{0,1\}$ e fica suficiente de provar $P[M<\infty]>0$ o de jeito equivalente
$P[M=\infty]<1$. Vamos prosseguir por contradição. 

\medskip

Definimos 
\begin{equation*}\begin{split}
                  M_k&:= \max_{1\le n \le k} (S_n-na)\\
                 M'_k&:= \max_{1\le n \le k} (S_{n+1}-S_n-na)
                 \end{split}
\end{equation*}
Essas duas sequências tem a mesma distribuição: Existe uma função $F$ tal que 
$M_k=F(X_1,\dots,X_k)$ e $M'_k=F(X_2,\dots,X_{k+1})$ e os vetores $(X_1,\dots,X_k)$ e $(X_2,\dots,X_{k+1})$ tem a mesma distribuição.

\medskip

As sequências convergem de jeito crescente para $M$ e $M'$ respetivamente que são também identicamente distribuída
(observe que $P(M\le x)=\lim_{k\to \infty} P(M_k\le x)$).
Usando as definições temos
\begin{equation*}
 M_{k+1}=\max(0,M'_k+X_1-a)=M'_k-\min(M'_k,a-X_1).
\end{equation*}
Usando o fato que $M_k$ e $M'_k$ tem mesma distribuição obtemos

$$E[\min(M'_k,a-X_1)]=E[M'_{k}]-E[M_{k+1}]=E[M_{k}-M_{k+1}]\le 0.$$
Usando o teorema de convergência dominada, ($0\le \min(M'_k,a-X_1)\le |X_1-a|$) obtemos que 
$$E[\min(M',a-X_1)]\le 0.$$
Agora para concluir, se tivermos $P(M'=\infty)=P(M=\infty)=1$, implicaria 
$E[a-X_1]\le 0$,
o que e impossível com a nossa escolha de $a$.

\qed




\todosec{Tópico: Teorema de Weierstrass}{provar o teorema de Weierstrass de aproximação de funções contínuas por polinômios (prova probabilística). Ele é usado em convergência fraca em $\mathbb{R}$}

\todosec{Tópico: Entropia de Shannon}{fazer...}

\todosec{Tópico: Processos de renovação}{fazer...}


\begin{exercise}[Depende de \nameref{s:percolacao}]
  Considere o grafo $G = (\mathbb{Z}^2, E)$, onde $E = \big\{ \{x,y\}; |x - y|_1 = 1 \big\}$.
  Dotamos agora o espaço $\{0,1\}^E$ com a $\sigma$-álgebra $\mathcal{A}$ gerada pelas projeções canônicas $Y_e(\omega) = \omega(e)$, onde $\omega \in \{0,1\}^E$ e $e \in E$.
  Definimos o conjunto $A \subseteq \{0,1\}^E$ por
  \begin{equation}
    A = \Big[
    \begin{array}{c}
      \text{Existe uma sequência $(x_i)_{i\ge 0}$  de elementos distintos de $\mathbb{Z}^2$,}\\
      \text{tais que $e_i = \{x_i, x_{i+1}\} \in E$ e $Y_{e_i} = 1$ para cada $i \geq 0$}
    \end{array}
    \Big].
  \end{equation}
  \begin{enumerate}[\quad a)]
  \item Mostre que $A$ é mensurável com respeito a $\mathcal{A}$.
  \item Mostre que $A$ é um evento caudal, ou seja
    \begin{equation}
      A \in \bigcap_{\{K \subseteq E,\, K \text{ finito}\}} \sigma\big( Y_e; e \not \in K \big).
    \end{equation}
  \item Conclua que $P(A) \in \{0,1\}$.
  \end{enumerate}
\end{exercise}

\begin{exercise}
  Seja $\Omega = E^\mathbb{Z}$ um espaço produto infinito, dotado da $\sigma$-álgebra $\mathcal{A}$ gerada pelas projeções canônicas $(X_i)_{i \in \mathbb{Z}}$.
  Consideramos agora em $(\Omega, \mathcal{A})$ a medida produto $\mathbb{P} = P^{\otimes \mathbb{Z}}$, onde $P$ é uma probabilidade fixada no espaço polonês $(E, \mathcal{B}(E))$.
  \begin{enumerate}[\quad a)]
  \item Mostre que para qualquer evento $A \in \mathcal{A}$ e qualquer $\varepsilon > 0$, existe um $k \in \mathbb{Z}_+$ e um evento $A_k \in \sigma(X_i, |i| \leq k)$ tais que $\mathbb{P}[(A \setminus A_k) \cup (A_k \setminus A)] < \varepsilon$.
  \item Considere o shift $\theta:\Omega \to \Omega$ dado por $\theta(\omega)(i) = \omega(i-1)$ e mostre que se $A = \theta(A)$, então $P(A) \in \{0,1\}$.
  \end{enumerate}
\end{exercise}

\subsection{Lei dos grandes números em $\cL^1$}

Acabamos de provar convergência das somas de variáveis independentes no sentido quase certo. 
Vamos ver agora que a convergência ocorre também em $\cL^1$.

\begin{theorem}
 Se $(X_n)_{n\ge 0}$ é uma sequência de variáveis independentes e de mesma distribuição com $X_1\in \cL^1$ então
 
 $$ \lim_{n\to \infty} \frac{1}{n}\sum_{k=1}^n X_k=E[X] $$
 em $\cL^1$.
\end{theorem}


\begin{proof}
Já sabemos que a convergência ocorre quase certamente e por consequência em probabilidade.
Para concluir, só precisamos mostrar que a sequência e uniformemente integrável usando Teorema \ref{t:ui}.
Usamos a notação $S_n=\sum_{k=1}^n X_k$.
Temos 
\begin{equation}\label{eq:hop}
\frac{1}{n}E[|S_n| \ind_{[|S_n|\ge n M]}]\le \frac{1}{n}\sum_{i=1}^n E[X_i \ind_{[|S_n|\ge n M]}]\le E[X_1\ind_{[|S_n|\ge n M]}]. 
\end{equation}
 Usando a desigualdade de Markov temos que 
 $$ P[|S_n|\ge nM]\le \frac{E[|S_n|]}{nM}\le E[X_1]/M.$$
Em particular  o termo a direita em \eqref{eq:hop} vai para zero quando $M\to \infty$ o que permite concluir. 
 
 \end{proof}

\begin{exercise}
 Mostrar que se $X_1\in \cL^q$, $q>1$ então a convergência na lei dos grandes números vale também em $\cL^p$ para todos $p\in[1,q)$.
\end{exercise}

\begin{remark}
 Na verdade a lei dos grandes números vale em $\cL^p$ desde que $X_1\in \cL^p$ mas a demostração deste ultimo ponto e mais delicada.
\end{remark}

\newpage
\begin{topics}
\section[Tópico:Séries de Kolmogorov]{Tópico: Convergência de series aleatórias,\\
Teoremas de Uma, Duas e Três séries de Kolmogorov}


\subsection{Um criterio geral básico}

Sejà $X_1,\dots, X_n$ uma sequencia de variáveis aleatórias.
Queremos ter uma condição suficiente para a convergência de 
\begin{equation}
S_n:= \sum_{i=1}^n X_i.
\end{equation}
Uma consequência do Teorema de Convergência Dominada e que 
\begin{lemma}
 Se $(X_n)_{n\in \bbN}$ e uma sequencia de variáveis aleatórias integráveis tal que 
 $\sum_{n=1}^{\infty} E[ |X_n| ]<\infty$,
então $\sum_{n=1}^{\infty} X_i$ e integrável.
Em particular a suma converge quase certamente.
\end{lemma}

O objetivo desse tópico e de presentar uns critérios mais eficaz no caso de soma de variáveis independente.

\medskip

\subsection{Caso de variáveis independentes: Os resultados}

Vamos começar com um exemplo para ilustrar o caso:\
Consideramos $(\gep_n)_{n\ge 1}$ uma sequencia de sinais aleatórios: são variáveis independente cuja distribuição e dada por
$$P[\gep_n=\pm 1]=1/2.$$
Queremos saber para quais valor de $p$ a soma $\sum_{n=1}^{\infty} \gep_n n^{-p}$ converge.

\medskip

E bastante fácil reparar (podemos por exemple usar o critério precedente) que temos convergência para $p>1$.

\medskip

O objetivo dos resultados que presentamos agora e de poder achar uma condição necessária e suficiente sobre o valor de $p$ para ter convergência.
E difícil ter uma intuição do resultado: sabemos que  $\sum_{n=1}^{\infty} n^{-p}$ diverge para todo $p\in (0,1)$ mas também que 
$\sum_{n=1}^{\infty}(-1)^n n^{-p}$ converge (a seria satisfaz o teste da série alternada). 

\medskip

Uma consequência dos resultados presentado abaixo e que no caso de sinais aleatórios, a convergência ocorre se $p>1/2$
(veremos mais tarde que esse resultado e o melhor possível).


 

\begin{theorem}[Uma serie]
 Sejà $(X_i)_{i\in \bbN}$ uma sequencia de variáveis independente tal que $E[X_i]=0$ para todo $i$,
 e que satisfaz também
 $$ \sum_{i=1}^\infty \Var X_i<\infty.$$
Temos então
\begin{equation}
P\left[\sum_{i=1}^{\infty} X_i \text{ converge } \right]=1.
\end{equation}
\end{theorem}

\medskip

\begin{theorem}[Duas Series]
 Sejà $(X_i)_{i\in \bbN}$ uma sequencia de variáveis independente
satisfaz
 $$ \sum_{i=1}^\infty \Var X_i<\infty.$$
Temos então
\begin{equation}
P[\sum_{i=1}^{\infty} X_i \text{ converge } ]=\begin{cases} 1 \quad \text{ se } \sum_{i=1}^\infty E[X_i] \text{ converge },\\
                                                0   \quad \text{ se } \sum_{i=1}^\infty E[X_i] \text{ não converge }.
                                              \end{cases}
\end{equation}
\end{theorem}

As vezes a variância pode não ser um bom criterio para avaliar convergência: $\Var(X_i)$ pode ser grande por causa de um evento que 
acontece com pouca probabilidade. O teorema seguinte permite de tratar uns desses casos:
dado um real positivo $c>0$, definimos 
\begin{equation}
X_i^c= X_i \1_{\{|X_i| \ge c\}}= \begin{cases} X_i \quad \text{ se } |X_i|\le c,\\
                                    0 \quad \text{ no caso contrário.} 
                                   \end{cases}
           \end{equation}                        
Chamamos essas variáveis aleatórias as variâveis troncadas.

\begin{theorem}[Tres Series]
 Sejam $c>0$ um real positivo e $(X_i)_{i\in \bbN}$ uma sequencia de variáveis independente que
satisfaz
\begin{itemize}
 \item [(i)]  $\sum_{i=1}^\infty P[ |X_i|>c ]<\infty,$
 \item [(ii)]  $\sum_{i=1}^\infty E[ X^c_i]<\infty,$
 \item [(iii)]  $\sum_{i=1}^\infty \Var X^c_i<\infty.$
\end{itemize}
Então
$$ P[\sum_{i=1}^{\infty} X_i \text{ converge } ]= 1 $$                                              
\end{theorem}



\begin{exercise}
Mostrar que para uma sequencia de variáveis independente $X=(X_i)_{i\in \bbN}$, com as notaçoes do teorema acíma,
$$ \exists c>0,\ \quad X \text{verifica $(i)-(ii)-(iii)$}  \quad \Leftrightarrow \quad \forall c>0,\ \quad X \text{verifica $(i)-(ii)-(iii)$}.$$
 
\end{exercise}


Os segundo a terceiro resultado são consequencia relativamente simples do primeiro.
Vamos dar uma prova deles logo
\begin{proof}[Prova to Teorema de Duas Series]
 Definimos $Z_i=X_i-E[X_i]$ e usando o Teorema de Uma Serie, concluimos que $\sum_{i=1}^{\infty} Z_i$ converge com probabilidade $1$. 
 
 \medskip
 
 Pois usando propriedades basicas de limites, vemos que $$\sum_{i=1}^n X_i=\sum_{i=1}^n Z_i+ \sum_{i=1}^n E[X_i]$$ converge se e sò se 
 a segunda sequência converge.
 \end{proof}

 \begin{proof}[Prova to Teorema de Tres Series]
Essa vez definimos $Y_i:= X_i \1_{\{|X_i| > c\}}$
Temos 
\begin{equation}
 \sum_{i=1}^n X_i =  \sum_{i=1}^n X_i^c + \sum_{i=1}^n Y_i
\end{equation}
Usando o Teorema de Duas Series para $X_i^c$, jà podemos dizer que o primeiro termo converge com probabilidade $1$.
Pelo segundo termo, temos 
\begin{equation}
 \sum_{i=1}^{\infty} P[Y_i\ne 0]=   \sum_{i=1}^{\infty} P[ X_i>c ]< \infty, 
\end{equation}
e então pelo Lemma de Borel Cantelli, $$P[ Y_i \ne 0 \text{ infinitas vezes}]=0.$$ 
 o que permite de concluir.
\end{proof}


\begin{exercise}
Sejam $\alpha, \beta, \gamma$ tres numeros reais positívos e $(X_n)_{n\in \bbN}$ variáveis independentes com distribuição
 \begin{equation} \begin{split}
 P[X_n=1]&=\frac{1}{4}\left(2+n^{-\alpha}-n^{-\beta}\right), \\ 
 P[X_n=-1]&=\frac{1}{4}\left(2-n^{-\alpha}-n^{-\beta}\right) \\
 P[X_n= n]&= \frac{1}{2} n^{-\beta}.
 \end{split}
\end{equation}
Discute da convergência da serie seguinte em função do valor de $\alpha, \beta$ e $\gamma$
\begin{equation}
 \sum_{n=1}^{\infty} n^{-\gamma} X_n.
\end{equation}


 
 
\end{exercise}

\subsection{A desigualdade de Kolmogorov}

Para provar o Teorema, vamos precisar de uma feramenta util tambem em outro contexto.
E uma versão mais forte de desigualdade de Chebychev.

\medskip


\begin{theorem}[Desigualdade de Kolmogorov]
Suponhamos que $X_i$ são variáveis independente tal que $\Var X_i<\infty$ para todos $i$. 
Então temos
\begin{equation}
 P[\max_{1\le k\le n} |S_k-E[S_k]|>\gep ]\le  \frac{\Var S_n}{\gep^2}.
\end{equation}

\end{theorem}


\begin{proof}
 Susbstituindo $X_i$ por $X_i-E[X_i]$ se for necessário, podemos supor que a esperança das variáveis vale zero.
 Vamos decompor o evento 
 $$\left\{ \max_{1\le k\le n} |S_k|>\gep \right\}$$ com respeito ao primeiro valor de $k$ por qual $|S_k|>\gep$.
 \begin{equation}
  \begin{split}
E_1&:=\{|S_1|\ge \gep  \},\\
E_k&:=\{|S_1|<\gep;\, \dots ;\,  |S_{k-1}|<\gep\, ;  \, |S_k|\ge \gep  \}, \quad 2\le k\le n.
   \end{split}
\end{equation}
Obviamente temos $$\left\{\max_{1\le k\le n} |S_k|>\gep\right\}=\bigcup_{i=1}^n E_i.$$
A união sendo disjunta temos 
\begin{equation}
P\left(\max_{1\le k\le n} |S_k|>\gep \right) \le  \sum_{i=1}^n P(E_i)\le  \sum_{i=1}^n E\left[  \frac{S^2_i}{\gep^2}\1_{E_i}\right].
\end{equation}
Vamos poder concluir a prova se mostramos para todo $k$
\begin{equation}\label{therest}
 E[S^2_k\1_{E_k}]\le E[S^2_n\1_{E_k}].
\end{equation}
Isso implica que 
\begin{equation}
  \sum_{i=1}^n E\left[  \frac{S^2_i}{\gep^2}\1_{E_i}\right]\le E\left [  \frac{S^2_n}{\gep^2}  \sum_{i=1}^n \1_{E_i}\right]= \frac{\Var(S_n)}{\gep^2}.
\end{equation}
Para mostrar \eqref{therest}, definimos $\Delta_{k,n}= S_n-S_k$.
Temos 
\begin{equation}
  E[S^2_n\1_{E_k}]= E[(S_k+\Delta_{k,n})^2\1_{E_k}]= E[S_k^2\1_{E_k}]+2 E[ \Delta_{k,n} S_k \1_{E_k}]+  E[ \Delta^2_{k,n}\1_{E_k}].
\end{equation}
O terceiro termo e obviamente positivo, então podemos concluir se mostramos  que $E[ \Delta_{k,n} S_k \1_{E_k}]=0$.
Lembramos que $S_k\1_{E_k}$ e uma funções das $k$ primeiras variáveis e $\Delta_{k,n}$ so depende das outras $n-k$
\begin{equation}
 S_k \1_{E_k}=f(X_1,\dots,X_k) \quad \text{ e } \quad \Delta_{k,n}=g(X_{k+1},\dots,X_{n}),
\end{equation}
Como os $X_i$ são independente, concluímos que  $S_k \1_{E_k}$ e $\Delta_{k,n}$ o são, e então

\begin{equation}
 E[ \Delta_{k,n} S_k \1_{E_k}]=E[\Delta_{n,k}S_k \1_{E_k}]=E[\Delta_{n,k}]E[S_k \1_{E_k}]=0.
\end{equation}
\end{proof}

\subsection{Prova to Teorema de Uma Serie}

Uma consequência imediata da desigualdade de Kolmogorov e que se \\ $\sum_{i=1}^{\infty} \Var X_i<\infty$ e $E[X_i]=0$
$$P\left[ \exists n\in \bbN,  |S_n| \ge A \right] \le \frac{\sum_{i=1}^{\infty} \Var X_i}{A^2}.$$
E obtida simplesmente pegando o limite quando $n$ vai para infinito da desigualdade.

\medskip

Vamos applicar essa formula pelo resto da soma. Definimos $n_k$ do jeito seguinte

\begin{equation}
 n_k:= \inf\left\{ n \sum_{i=n}^{\infty} \Var X_i \right\}<2^{-3k}.
\end{equation}
Aplicando a desigualdade de Kolmogorov temos 
$$ P\left[ \exists r\ge n_k,  |\sum_{i=n_k}^r X_i| \ge 2^{-k} \right]\le 2^{-k}.$$
Então pelo Lema de Borel-Cantelli,
\begin{equation}
 P\left[ \exists k_0 \ \forall k\ge k_0, \forall r\ge n_k, \left|\sum_{i=n_k}^r X_i\right|\le 2^{-k} \right]=1.
\end{equation}
Agora nota que o evento  $$\{\exists k_0 \ \forall k\ge k_0, \forall r\ge n_k, |\sum_{i=n_k}^r X_i|\le 2^{-k}\}$$ 
implica  $$ \forall k\ge k_0,\, \forall m,p\ge n_k, \, 
|S_m-S_p|\le 2^{-k+1}.$$ 
Então a sequencia e de Cauchy com probabilidade $1$ o que permite de concluir a prova. \qed




\subsection{Demostração da LGN usando o Teorema das Três Séries}

Antes de começar a prova, buscando inspiração no Teorema das Três Séries, mostraremos que basta considerar versões truncadas das variáveis $X_i$.
Isso é feito no próximo

\begin{lemma}
  \label{l:LFGN}
  Sejam $Y_i = X_i \1_{[|X_i| \leq i]}$.
  Então, para demonstrar o Teorema~\ref{t:LFGN}, basta provar que
  \begin{equation}
    \lim_{n \to \infty}\frac{1}{n} \sum_{i=1}^n Y_i = m, \text{ $P$-quase certamente.}
  \end{equation}
\end{lemma}

\begin{proof}[Prova do Lema~\ref{l:LFGN}]
  Consideramos os eventos $A_i = [X_i \neq Y_i]$.
  Obviamente,
  \begin{equation}
  \sum_{i=1}^{\infty} P(A_i) = \sum_{i=1}^{\infty} P[|X_i| \geq i] \leq 
  \int_0^\infty P[|X_1| \geq t] \d t = E\big(|X_1|) < \infty.
  \end{equation}
  Logo, pelo Lema de Borel-Cantelli, temos que $P$-quase certamente $A_i$ acontece apenas finitas vezes.
  Digamos que $A_i$ não acontece para $i > N(\omega)$.
  Dessa forma, para qualquer $n \geq 1$,
  \begin{equation}
    \Big|\frac{1}{n}\sum_{i=1}^n (X_i - Y_i)\Big| \leq \frac{1}{n}\sum_{i=1}^n |X_i - Y_i| \leq \frac{1}{n} \sum_{i \leq N(\omega)} |X_i|,
  \end{equation}
  que converge para zero $P$-quase certamente, mostrando o resultado.
\end{proof}

O próximo passo para a prova da Lei Forte dos Grandes Números é cuidar da esperança das novas variáveis $Y_i$.
\begin{lemma}
  \label{l:lim_Z_n_LFGN}
  Sejam $Z_i = Y_i - E(Y_i)$, para $i \geq 1$ como acima.
  Então, para demostrar o Teorema~\ref{t:LFGN}, basta mostrar que
  \begin{equation}
    \label{e:lim_Z_n_LFGN}
    \lim_{n \to \infty}\frac{1}{n} \sum_{i=1}^n Z_i = 0, \text{ $P$-quase certamente.}
  \end{equation}
\end{lemma}

\begin{proof}
  Supondo a convergência em \eqref{e:lim_Z_n_LFGN}, sabemos que
  \begin{equation}
    \lim_{n \to \infty} \frac{1}{n} \sum_{i=1}^n Y_i - E(Y_i) = 0, \text{ $P$-quase certamente.}
  \end{equation}
  Mas $E(Y_i) = E(X_i \1_{[|X_i| \leq i]})$ que converge a $E(X_i) = m$, pelo Teorema da Convergência Dominada, donde concluímos que
  \begin{equation}
    \lim_{n \to \infty} \frac{1}{n} \sum_{i=1}^n E(Y_i) = m.
  \end{equation}
  Dessa forma, obtemos que $\tfrac 1n \sum_{i=1}^n Y_i$ converge quase certamente a $m$, donde concluímos a prova do Teorema~\ref{t:LFGN} por meio do Lema~\ref{l:LFGN}.
\end{proof}

Gostaríamos de utilizar os teoremas das séries para mostrar a convergência de $\tfrac 1n \sum_{n} Z_n$, mas obviamente, o fator $\tfrac 1n$ que precede a soma nos impede de fazê-lo.
O próximo resultado é um simples exercício de análise real, que nos permite reduzir a prova de \eqref{e:lim_Z_n_LFGN} para uma simples convergência de uma série sem pré-fatores.

\begin{lemma}[Lema de Kronecker]
  Suponha que $x_n \in \mathbb{R}$ e $b_n > 0$ sejam tais que $b_n \uparrow \infty$ e $\sum_{i=1}^\infty \frac{x_i}{b_i}$ convirja a $s \in \mathbb{R}$.
  Então
  \begin{equation}
    \lim_{n \to \infty} \frac{1}{b_n} \sum_{i=1}^n x_i = 0.
  \end{equation}
\end{lemma}

\begin{proof}
  Definindo $s_0 = 0$ e $s_n = \tfrac{x_1}{b_1} + \dots + \tfrac{x_n}{b_n}$, temos, por integração por partes,
  \begin{equation}
    \sum_{i=1}^n x_i = \sum_{i=1}^n b_i \frac{x_i}{b_i} = \sum_{i=1}^n b_i s_{i} - \sum_{i=1}^n b_i s_{i-1} = b_n s_n + \sum_{i=1}^{n-1} (b_{i} - b_{i+1}) s_{i}.
  \end{equation}
  Escolhemos agora, para qualquer $\varepsilon > 0$, um $n_0 \geq 1$ tal que $|s_n - s| < \varepsilon$ para todo $n \geq n_0$.
  Dessa forma,
  \begin{equation*}
    \begin{split}
      \frac{1}{b_n} \sum_{i=1}^n x_i & = s_n - \frac{1}{b_n}\sum_{i=1}^{n-1} (b_{i+1} - b_{i}) s_{i}\\
      & = s_n - \frac{1}{b_n}\underbrace{\sum_{i=1}^{n_0-1} (b_{i+1} - b_{i})}_{\Delta_{n_0}} s_{i} - \frac{1}{b_n}\sum_{i=n_0}^{n-1} (b_{i+1} - b_{i}) s_{i}\\
      & = \underbrace{s_n}_{\to s} - \underbrace{\frac{1}{b_n}\Delta_{n_0}}_{\to 0} - \underbrace{\frac{1}{b_n}\sum_{i=n_0}^{n-1} (b_{i+1} - b_i) s}_{= \tfrac{(b_n - b_{n_0})s}{b_n} \to s} - \underbrace{\frac{1}{b_n}\sum_{i=n_0}^{n-1} (b_{i+1} - b_{i}) (s_{i} - s)}_{\leq \varepsilon\tfrac{(b_n - b_{n_0})}{b_n} \leq \varepsilon},
    \end{split}
  \end{equation*}
  onde os limites indicados acima representam o que acontece quando $n \to \infty$.
  A prova segue do fato de $\varepsilon$ ter sido escolhido arbitrariamente.
\end{proof}

Estamos agora em posição de finalizar a
\begin{proof}[Prova do Teorema~\ref{t:LFGN}]
  De acordo com o Lema de Kronecker e o Lema~\ref{l:lim_Z_n_LFGN}, é suficiente mostrar que
  \begin{equation}
    \sum_{i=1}^n \frac{Z_i}{i}, \text{ converge quase certamente}.
  \end{equation}
  Por outro lado, como os $Z_i$'s tem média zero, o Teorema de Uma Série diz que é suficiente mostrar que
  \begin{equation}
    \sum_{i=1}^n \Var\Big(\frac{Z_i}{i}\Big) = \sum_{i=1}^n \frac{1}{i^2} \Var(Z_i) < \infty.
  \end{equation}
  Isso segue da seguinte estimativa
  \begin{equation}
    \begin{split}
      \sum_{i=1}^n \frac{1}{i^2} \Var(Z_i) & = \sum_{i=1}^n \frac{1}{i^2} \Var(Y_i) \leq \sum_{i=1}^n \frac{1}{i^2} E\big( X_i^2 \1_{[|X_i| \leq i]}\big)\\
      & = \sum_{i=1}^n \frac{1}{i^2} \sum_{k=1}^{i} E\big( X_1^2 \1_{[k-1 < |X_1| \leq k]}\big)\\
      & = \sum_{k=1}^n E\big( X_1^2 \1_{[k-1 < |X_1| \leq k]}\big) \sum_{i=k}^{n} \frac{1}{i^2}\\
      & \leq 2 \sum_{k=1}^n \frac{1}{k} E\big( X_1^2 \1_{[k-1 < |X_1| \leq k]}\big)\\
      & \leq 2 \sum_{k=1}^n E\big( X_1 \1_{[k-1 < |X_1| \leq k]}\big) \leq 2E(X_1) < \infty.
    \end{split}
  \end{equation}
  Isso nos permite concluir a prova de \eqref{e:lim_Z_n_LFGN} via o Lema de Kronecker.
  Consequentemente, obtemos o Teorema~\ref{t:LFGN} via o Lema~\ref{l:lim_Z_n_LFGN}.
\end{proof}

\begin{exercise}
  Sejam $Y_k$ variáveis aleatórias independentes e com a seguinte distribuição:
  \begin{equation}
    P[Y_k = i] =
    \begin{cases}
      \frac 12 - \frac 1{k^2} \quad & \text{se $i = 1$ or $i = -1$},\\
      \frac 2{k^2} & \text{se $i = 3$.}
    \end{cases}
  \end{equation}
  Mostre que
  \begin{equation}
    P\Big[ \frac 1n \sum_{k=1}^n Y_k \text{ converge a zero} \Big] = 1.
  \end{equation}
\end{exercise}

\begin{exercise}[Depende de \nameref{s:urna_polya}]
  Mostre que segundo a lei $P$ construida no Exercício~\ref{x:constr_Polya}, vale que
  \begin{equation}
    P\big[ \tfrac 1n \sum_{i-1}^n X_i \text{ converge}] = 1.
  \end{equation}
  Além disso calcule a distribuição do limite acima.
\end{exercise}
\end{topics}

\section{Momentos exponenciais}

Nessa seção desenvolveremos uma outra técnica para estimar a probabilidade de uma variável aleatória se desviar de sua esperança.

Já vimos o método do primeiro, segundo e quarto momento para controlar uma soma de variáveis independentes.
Um exemplo disso foi visto na estimativa
\begin{equation}
  P\Big[ \sum_{i=1}^n (X_i - E(X_i)) \geq a \Big] \leq \frac{\sum_i \Var (X_i)}{a^2}.
\end{equation}

Em geral, quanto maior o momento, melhor a estimativa do decaimento para a probabilidade de que uma variável se desvie de sua esperança.
Nessa seção iremos para momentos exponenciais, que em um certo sentido produzem estimativas ótimas para o comportamento assintótico da probabilidade de desvio.

Note que se quisermos uma pequena probabilidade de erro (como por exemplo $\sim 0.01$), o método do segundo momento é muito bom, como veremos posteriormente.
Mas se quisermos uma probabilidade de erro minúscula (em situações concretas, algo como $10^{-12}$ por exemplo), certamente teremos que aumentar bastante o valor de $n$, mas quanto?
As cotas de segundo momento são muito ruins para esse tipo de estimativa, nos levando a escolher um $n$ maior que o necessário.
Abaixo, desenvolveremos um método mais eficiente para responder a essa pergunta, obviamente sob certas hipóteses na distribuição das variáveis aleatórias.

\begin{definition}
  Dada uma variável aleatória $X$, definimos sua transformada de Laplace \index{trasformada!de Laplace} como
  \begin{equation}
    \phi_X(s) = E(\ex{s X}) \in (0, \infty],
  \end{equation}
  para todos $s \in \mathbb{R}$.
  Essa transformada também é chamada \emph{função geradora de momentos} de $X$. \index{funcao@função!geradora de momentos}
\end{definition}

\begin{exercise}
  Calcule a função geradora de momentos das distribuições $\Ber(p)$, $\Exp(\lambda)$ e $U_{[0,1]}$.
\end{exercise}

\begin{proposition}
  \label{p:propried_phi}
  Se $E(\ex{\delta |X|}) < \infty$, então
  \begin{enumerate}[\quad a)]
  \item $X \in \mathcal{L}^p$ para todo $1 \leq p < \infty$,
  \item $\phi_X(s) < \infty$ para todo $s \in (-\delta, \delta)$,
  \item $\phi_X(s)$ é $C^\infty$ em $(-\delta, \delta)$ e
  \item $\phi_X^{(n)}(s) = E(X^n \ex{sX})$.
  \item $s\mapsto \log \phi_{X_1}(s)$ e convexa no interval onde esta definida.
  \end{enumerate}
\end{proposition}

A última conclusão da proposição acima justifica a nomenclatura função geradora de momentos pois $\phi_X^{(n)}(0) = E(X^n)$.

\begin{proof}
  Obviamente, para todo $p \geq 1$ existe $c > 0$ tal que $\ex{\delta |x|} \geq c |x|^p$, donde $X \in \mathcal{L}^p$.
  Além disso, para todo $s \in (-\delta, \delta)$, temos $\phi_X(s) = E(\ex{s X}) \leq E(\ex{\delta |X|}) < \infty$, donde \textit{2.} segue imediatamente.

  Fixando $s \in \mathbb{R}$, vamos agora calcular
  \begin{equation}
      \frac{\phi_X(s + h) - \phi_X(s)}{h} = \frac{E\big(\ex{(s+h)X} - \ex{sX}\big)}{h} = E\Big(\ex{sX} \frac{\ex{hX} - 1}{h}\Big).
  \end{equation}
  Lembrando que $|\tfrac{1}{y}(e^y - 1)| \leq e^{|y|}$, para todo $y \in \mathbb{R}$, temos que para todos os $h < (\delta - |s|)/2$, o integrando acima é dominado por $|X| \ex{(|s| + h) |X|} \leq |X| \ex{\smash{\tfrac{\delta + |s|}{2} |X|}}$ que pertence a $\mathcal{L}^1$.
  Logo podemos usar o Teorema da Convergência Dominada para trocar o limite $h \to 0$ com a esperança, obtendo
  \begin{equation}
    \phi_X'(s) = E(X \ex{sX}).
  \end{equation}

  Note que para todo $\varepsilon > 0$ e $k \geq 1$, $|x|^k \leq c(k) \ex{\varepsilon |x|}$, isso nos permite repetir o argumento acima indutivamente para obter
  \textit{c)} e \textit{d)}.
  
  \medskip
  
  O ultimo ponto e simplesmente uma consequencia da desigualdade de H\"older. Para $\gl\in (0,1)$ temos 
  \begin{multline}
   \phi_X( \gl a +(1-\gl)b)= E[e^{\gl a X} e^{(1-\gl)b X}] \\ 
   \le E[e^{a X}]^\gl E[ e^{b X}]^{(1-\gl)}=  (\phi_X(a))^{\gl}  (\phi_X(b))^{1-\gl}.
  \end{multline}

\end{proof}

Lembramos que ao usar o método do segundo momento, nos foi bastante útil o fato que a variância se comporta bem com relação a somas independentes.
Mais precisamente, $\Var(X_1 + \dots + X_k) = \Var(X_1) + \dots + \Var(X_k)$.

Uma outra propriedade importante da função geradora de momentos é que ela também se comporta bem com respeito à somas independentes.
\begin{proposition}
  Se $X_1, \dots, X_n$ são variáveis independentes com $\phi_{X_i}(s) < \infty$ para todo $i \leq k$ e $|s| < \delta$, então
  \begin{equation}
    \phi_{X_1 + \dots + X_k}(s) = \phi_{X_1}(s) \dotsm \phi_{X_k}(s), \text{ para todos $|s| < \delta$.}
  \end{equation}
\end{proposition}

\begin{proof}
  Basta observar que
  \begin{equation}
    \begin{split}
      E[\exp & \{s(X_1 + \dots + X_k)\}) = E[(\ex{sX_1} \dotsm \ex{sX_k})]\\
      & = E\big[\ex{sX_1}) \dotsm E(\ex{sX_k}\big] = \phi_{X_1}(s) \dotsm \phi_{X_k}(s),
    \end{split}
  \end{equation}
  usando Fubini.
\end{proof}

Consideraremos agora uma sequência $X_1, X_2, \dots$ de variáveis \iid com $\phi_{X_1}(s) < \infty$ para $|s| < \delta$.
Então podemos tentar estimar, para $a > 0$ e $|s| < \delta$,
\begin{equation*}
  \begin{split}
    P \Big[ & \frac{X_1 + \dots + X_n}{n} - E(X_1) \geq a \Big] = P \Big[ X_1 + \dots + X_n \geq (a + E(X_1)) n \Big]\\
    & \quad = P \Big[ \ex{s(X_1 + \dots + X_n)} \geq \ex{s (a + E(X_1)) n}\Big]\\
    & \quad \leq \phi_{X_1 + \dots + X_n}(s) \ex{-s (a + E(X_1))n } = \phi_{X_1}^n(s) \ex{-s (a + E(X_1))n }.
  \end{split}
\end{equation*}
O primeiro fator na estimativa acima pode crescer exponencialmente com $n$, enquanto o segundo decresce.
Gostaríamos que o comportamento do segundo predominasse, o que podemos concluir do seguinte argumento.

Sabemos que $\phi_{X_1}(s)$ é diferenciável em zero e que $\phi'_{X_1}(0) = E(X_1)$.
Logo, existe $s > 0$ tal que $\phi_{X_1}(s) < 1 + (E(X_1) + \tfrac{a}{2}) s$, donde
\begin{equation*}
  \begin{split}
    P \Big[ & \frac{X_1 + \dots + X_n}{n} - E(X_1) \geq a \Big] \leq \phi_{X_1}^n(s) \ex{-s (a + E(X_1))n }\\
    & \quad \leq \big(1 + (E(X_1) + \frac{a}{2})s \big)^n \ex{-s (E(X_1) + a)n }\\
    & \quad \leq \exp\Big\{ s \Big( E(X_1 + \frac{a}{2} - E(X_1) - a) n \Big) \Big\} = \ex{-san/2}.
  \end{split}
\end{equation*}
Isso nos garante um decaimento exponencial da probabilidade da média dos $X_i$ se desviar da esperança.

\begin{exercise}
  Aplique o método acima para variáveis $X_i$ \iid com distribuição $\Ber(1/2)$ e encontre $s(a)$ que otimize o decaimento da probabilidade $P\big[\sum_{i=1}^n X_i > (1/2 + a) n \big]$.
\end{exercise}

Poderíamos nos perguntar se a cota acima é suficientemente boa.
Talvez pudéssemos esperar um decaimento ainda melhor que exponencial.
Para responder a essa pergunta, vamos considerar o seguinte exemplo.
Sejam $(X_i)_{i \geq 1}$ variáveis \iid com $X_1 \distr \Ber(1/2)$.
Nesse caso temos por exemplo
\begin{equation}
  P\Big[ \big| \frac{X_1 + \dots + X_n}{n} - \frac 12 \big| \geq \frac 14\Big] \geq P[X_i = 1, \forall i \leq n] = 2^{-n}.
\end{equation}
Dessa forma, sabemos que não podemos esperar um decaimento melhor que exponencial, mesmo para variáveis bem simples (como Bernoulli) que satisfazem $\phi_X(s) < \infty$ para todo $s \in \mathbb{R}$.

Note que para variáveis com distribuição $\Ber(1/2)$, obtivemos acima cotas exponenciais em $n$ (superior e inferior), mas elas possuem expoentes diferentes.
Resta agora tentar entender qual é o expoente correto para o decaimento da probabilidade $P[X_1 + \dots + X_n \geq n(E(X_1) + a)]$, o que será feito na próxima seção.

\todosec{Tópico: Processos de ramificação}{fazer...}

\section{Princípio de Grandes Desvios}
\label{s:PGD}

A primeira tarefa nossa será otimizar a estimativa grosseira feita na seção anterior.
Essas estimativas são chamadas de \emph{estimativas de grandes desvios}, pois se referem a probabilidades que a média empírica de $X_i$ se desvie de sua esperança por um valor constante $a$.
Futuramente no curso estudaremos as probabilidades de que esse desvio seja de ordem $a_n \to 0$ que são chamados de \emph{desvios moderados} ou \emph{flutuações}, dependendo se a probabilidade de desvio converge a zero ou não.

\begin{theorem}[Princípio de Grandes Desvios - cota superior]
  \index{Principio@Princípio!de Grandes Desvios@de Grandes Desvios}
  \label{t:PGDleq}
  Consideramos variáveis aleatórias \iid $X_1, X_2, \dots$ tais que $\phi_{X_1}(s) < \infty$, para todo $s \in (-\delta, \delta)$.
  Então, para $a > 0$,
  \begin{equation}
    P\big[ X_1 + \dots + X_n \geq \big(m + a \big) n \big] \leq \ex{-\psi_{X_1}(m + a) n},
  \end{equation}
  onde $m = E(X_1)$ e
  \begin{equation}
    \psi_{X_1}(x) = \sup_{s \in \bbR} \big\{ xs - \log \big( \phi_{X_1}(s) \big) \big\}
  \end{equation}
  é chamada função taxa. \index{funcao@função!taxa}
  De jeito simetrico temos tambem  
    \begin{equation}
    P\big[ X_1 + \dots + X_n \leq \big(m - a \big) n \big] \leq \ex{-\psi_{X_1}(m - a) n}.
  \end{equation}
\end{theorem}

Podemos reparar que 
  \begin{equation}
    \psi_{X_1}(x) = \begin{cases}
                     \sup_{s \ge 0} \big\{ xs - \log \big( \phi_{X_1}(s) \big) \big\}, \text{ se } x\le m \\
                      \sup_{s \ge 0} \big\{ xs - \log \big( \phi_{X_1}(s) \big) \big\}, \text{ se } x\ge m. 
                    \end{cases}
   \end{equation}
Isso e uma consequencia simples to fato que a função $s\mapsto xs - \log \big( \phi_{X_1}(s) \big)$ e concava e que a derivada em $0$ vale 
$$x- \partial_s \big(\log  E[e^s X] \big)=x-m.$$ Então dependendo do sinal de $(x-m)$ o supremo e atingido a direito o a esqueirda de zero.

   
   
%Antes de provar o teorema, vamos fazer uma breve observação sobre como a função geradora de momentos se comporta com respeito a soma de constantes.
%Isso nos permitirá centrar as variáveis para nossas estimativas.
%
%\begin{lemma}
%  \label{l:phi_Xmaisb}
%  Seja $X$ uma variável aleatória tal que para algum $s_0 > 0$ tenhamos $\phi_{X}(s) < \infty$ para todo $s \in (-\delta, \delta)$.
%  Então
%  \begin{equation}
%    \log\big(\phi_{X - b}(s)\big) = \log\big(\phi_{X}(s)\big) -bs < \infty, \text{ para todo $s \leq s_0$.}
%  \end{equation}
%\end{lemma}

%\begin{proof}
%  Basta observar que
%  \begin{equation}
%    \phi_{X - b}(s) = E\big( \ex{s(X-b)} \big) = \ex{-sb} E\Big( \ex{sX}\Big) = \ex{-sb} \phi_X(s),
%  \end{equation}
%  e tomar logarítmos de ambos os lados para obter o resultado.
%\end{proof}

\begin{proof}
  Já sabemos que, para todo $s \geq 0$,
  \begin{equation}
    \begin{split}
      P\big[ X_1 & + \dots + X_n \geq \big(m + a \big) n \big] \leq \phi_{X_1}^n (s) \ex{-s (m + a) n}\\[1mm]
      & = \ex{ \log \big( \phi_{X_1}(s)\big) n - s(m + a) n}\\
      & = \ex{ - \big( (m + a)s - \log \big( \phi_{X_1}(s)\big) \big) n}\\
%      & \overset{\text{Lema}~\ref{l:phi_Xmaisb}\;\;}= & \ex{ \log \big( \phi_{X_1 - m}(s) \big) n - s a n}\\
%      & = & \ex{ - \big(as -\log \big( \phi_{X_1 - m}(s) \big) \big) n}
    \end{split}
  \end{equation}
  O que termina a prova do teorema se tomamos o ínfimo em $s \geq 0$.
\end{proof}

\begin{exercise}
  Calcule $\psi_X(a)$ quando $X$ é distribuída como $\Ber(p)$, $U_{[0,1]}$ e $\Exp(\lambda)$.
\end{exercise}

\begin{exercise}
  Na Nova Caledônia, temos $k$ habitantes.
  Seja $f:\{1, \dots, k\} \to \{0,1\}$ uma função que indica a intenção de voto de cada cidadão.
  Mais precisamente, para cada habitante $i \in \{1, \dots, k\}$, se $f(i) = 0$, então $i$ vota no candidato $0$, enquanto se $f(i) = 1$, o cidadão $i$ vota no candidato $1$.
  Para estimar o número $k_1 = \# f^{-1}(\{1\})$ de pessoas que votam em $1$, nós escolhemos variáveis aleatórias $Y_i$ i.i.d. com distribuição uniforme em $\{1, \dots, k\}$ e queremos estimar
  \begin{equation}
    \text{Err}_n(\epsilon) = P \Big[ \Big| \frac{1}{n} \sum_{i=1}^n f(Y_i) - \frac{k_1}{k} \Big| > \epsilon \Big].
  \end{equation}
  Sabendo que $k$ é par e $k_1 = k/2$, então
  \begin{enumerate}[\quad a)]
  \item use o método do segundo momento para obter um $n$ tal que $\text{Err}_{n}(0.01) < 0.02$ e um $n$ tal que $\text{Err}_{n}(0.01) < 10^{-12}$,
  \item use o método do momento exponencial para obter resolver o ítem acima.
  \end{enumerate}
  Compare os quatro resultados obtidos acima.
\end{exercise}

Vamos agora tomar um exemplo concreto para análise.
Sejam $X_1, X_2, \dots$ variáveis aleatórias \iid com distribuição $\Ber(1/2)$, donde
\begin{equation}
  \phi_{X_1}(s) = \frac{1}{2} (1 + e^s) \quad \text{e} \quad \psi_{X_1}(x) = \sup_{s \geq 0} \{xs - \log(1 + e^s) + \log(2) \}.
\end{equation}
Um cálculo simples nos mostra que, se $x < 1$, o mínimo acima é atingido no único ponto $s_{\text{max}} = \log(\tfrac{x}{1-x})$.
Portanto, podemos concluir do Teorema~\ref{t:PGDleq} que
\begin{equation}
  \begin{split}
    P[X_1 + \dots & + X_n > 1/2 + a] \leq \ex{- \psi_{X_1}(s_{\text{max}})n}\\
    & = \exp\Big\{-n \Big(b \log(b) + (1-b)\log(1-b) + \log(2) \Big)\Big\}
  \end{split}
\end{equation}
Note que $P[X_1 + \dots + X_n = n] = 2^{-n} = \ex{-\log(2)n} = \ex{-\psi_{X_1}(1-)n}$.
Isso nos dá um forte indício de que talvez nossas cotas superiores não estejam tão longe de ser precisas.
Para confirmar essa hipótese, precisamos obter cotas inferiores parecidas.

\begin{figure}[!ht]
  \centering
  \begin{tikzpicture}[scale=3]
    \draw[->] (-0.2,0) -- (1.2,0) node[right] {$b$};
    \draw[-] (1,0.02) -- (1,-0.02) node[below] {$1$};
    \draw[-] (0.02,{ln(2)}) -- (-0.02,{ln(2)}) node[left] {$\log(2)$};
    \node[below left] at (0,0) {$0$};
    \draw[->] (0,-0.2) -- (0,1.1) node[above] {$\psi_{X}(b)$};
    \draw[domain=0.0001:0.9999,smooth,variable=\x,blue] plot ({\x},{\x*ln(\x) + (1 - \x)*ln(1 - \x) + ln(2)});
    \draw[->] (1.8,0) -- (3.2,0) node[right] {$b$};
    \draw[-] (3,0.02) -- (3,-0.02) node[below] {$1$};
    \node[below left] at (2,0) {$0$};
    \draw[->] (2,-0.2) -- (2,1.6) node[above] {$\psi_{X'}(b)$};
    \draw[domain=2.0001:2.9999,smooth,variable=\x,blue] plot ({\x},{(\x-2)*ln((\x - 2)/0.75) + (1 - \x + 2)*ln((1 - \x + 2)/(0.25))});
    \draw[-,dotted] (3,{ln(4/3)}) -- (1.98,{ln(4/3)}) node[left] {$\log(4/3)$};
    \draw[-] (2.02,{ln(4)}) -- (1.98,{ln(4)}) node[left] {$\log(4)$};
  \end{tikzpicture}
  \caption{Funções taxa $\psi_{X}(b)$ de uma variável $X$ com distribuição $\Ber(1/2)$, e $\psi_{X'}(b)$ de uma variável com distribuição $\Ber(3/4)$, para $b \in (0,1)$.}
\end{figure}

Antes de buscar cotas inferiores para as probabilidades de desvio, vamos estabelecer algumas propriedades da função $\psi_X(b)$.
Primeiramente, quando podemos dizer que o supremo na definição de $\psi_X$ é atingido em algum $s_{\text{max}}$?
Certamente, esse nem sempre é o caso, por exemplo se $X = m$ quase certamente, então $\phi_X(s) = e^{sm}$ e o supremo definindo $\psi_X(b)$ não é atingido se $b \neq m$.

\begin{lemma}
  \label{l:smax_PGD}
  Seja $X$ uma variável aleatória tal que $\phi_X(s) < \infty$ para todo $s \in (-\delta, \delta)$.
  Supondo $a \geq 0$ é tal que $P[X > m + a] > 0$, então existe $s_{\text{max}} \geq 0$ tal que
  \begin{equation}
    \psi_X(m + a) = (m + a)s_{\text{max}} - \log\big(\phi_X(s_\text{max})\big).
  \end{equation}
\end{lemma}

\begin{proof}
Para mostrar que o minimo e atingido e em ponto e suficiente mostrar que 
$$ \lim_{s\to \infty} (m + a)s - \log\big(\phi_X(s)\big)=\infty,$$
pois a função e continua.
Por hipótese, existe $x > m + a$ tal que $p = P[X \geq x] > 0$, donde $\phi_X(s) \geq p e^{sx}$ e portanto podemos concluir por que
$$(m + a)s - \log\big(\phi_X(s)\big)\ge (x-m+a)s -\log p.$$

\end{proof}

\begin{lemma}
  Seja $X$ uma variável aleatória tal que $\phi_X(s) < \infty$ para todo $s \in (-\delta, \delta)$.
  Então o conjunto onde a função $\psi_X(s)$ é finita é um intervalo, na qual $\psi_X$ é convexa e portanto contínua.
\end{lemma}

\begin{proof}
  Primeiramente, supomos que $a < b$ são tais que $\psi_X(a)$ e $\psi_X(b)$ são finitas.
  Logo, para todo $c \in (a, b)$, temos que a função linear $cs$ é menor ou igual a $as \vee bs$, daí
  \begin{equation}
    \begin{split}
      \psi_X(c) &= \sup_{s \geq 0} \{cs - \log(\phi_X(s))\} \leq  \sup_{s \geq 0} \{(as \vee bs) - \log(\phi_X(s))\}\\
      & \leq \sup_{s \geq 0} \{as - \log(\phi_X(s))\} \vee \sup_{s \geq 0} \{bs - \log(\phi_X(s))\} < \infty.
    \end{split}
  \end{equation}
  Para mostrar que $\psi_X$ é convexa, observe que $\psi_X(x)$ é dada pelo supremo (para $s \geq 0$) das funções afins $x \mapsto xs - \psi_X(s)$.
  Como o supremo de funções convexas é também convexo, obtemos o enunciado do lemma.
\end{proof}


\begin{exercise}
  Suponha que se $\phi_{X}(s)$ é finita para todo $s \in (-\delta, \delta)$ e mostre que as seguintes afirmações valem.
  \begin{enumerate}[\quad a)]
  \item A função $\psi_{X}(s)$ é não negativa, semi-contínua inferior e convexa em seu domínio.
  \item $\psi_X(a)$ se anula somente em $a = m$.
  \end{enumerate}
\end{exercise}

Buscaremos agora cotas inferiores para a probabilidade de obter um grande desvio.
Gostaríamos que essas estimativas fossem o mais próximas possíveis das estimativas superiores obtidas acima.
Certamente não podemos obter algo como
\begin{equation}
  \label{e:PGDgeq_falso}
  `` P\big[ X_1 + \dots + X_n \geq \big(m + a \big) n \big] \geq \exp\{-\psi_{X_1}(a) n\} ",
\end{equation}
pois senão isso nos daria uma igualdade o que é impossível, pois perdemos um pouco de precisão ao utilizar a desigualdade de Markov na cota superior.

Contudo, gostaríamos de entender se ao menos o expoente $\psi_{X_1}(a)$ na cota superior também possui algum papel na cota inferior.
Isso é confirmado no seguinte resultado.

\begin{theorem}[Princípio de Grandes Desvios - cota inferior]
  \index{Principio de Grandes Desvios@Princípio de Grandes Desvios}
  \label{t:PGDgeq}
  Sejam $X_1, X_2, \dots$ variáveis aleatórias \iid com $\phi_{X_1}(s) < \infty$, para todo $s \in \mathbb{R}$.
  Então, para todo $a > 0$,
  \begin{equation}
    \liminf_{n \to \infty} \frac{1}{n} \log P\big[ X_1 + \dots + X_n \geq \big(m + a \big) n \big] \geq -\psi_{X_1}(m + a),
  \end{equation}
  onde novamente $m = E(X_1)$ e $\psi_{X_1}(x)$ é definida como no Teorema~\ref{t:PGDleq}.
  De jeito analogo 
    \begin{equation}
    \liminf_{n \to \infty} \frac{1}{n} \log P\big[ X_1 + \dots + X_n \leq \big(m - a \big) n \big] \geq -\psi_{X_1}(m - a),
  \end{equation}
\end{theorem}

Note que o resultado do teorema acima é mais fraco que o que vemos na equação \eqref{e:PGDgeq_falso}, mas mostra que $\psi_{X_1}(a)$ é realmente o expoente correto no decaimento da probabilidade de grandes desvios.

Um corolário dos Teoremas~\ref{t:PGDleq} e \ref{t:PGDgeq} é o seguinte

\begin{corollary}
  Se $X_1, X_2, \dots$ variáveis aleatórias \iid com $\phi_{X_1}(s) < \infty$, para todo $s \in \mathbb{R}$, então
  \begin{equation}
    \lim_{n \to \infty} \frac{1}{n} \log P\big[ X_1 + \dots + X_n \geq \big(m + a \big) n \big] = -\psi_{X_1}(m + a).
  \end{equation}
\end{corollary}

A idéia da prova é transformar a distribuição de $X_i$, usando uma exponencial como derivada de Radon-Nikodim.
Essa nova distribuição possuirá esperança maior que $m + a$, de forma que se tomamos a média de variáveis \iid $X'_1, \dots, X'_n$ distribuídas dessa forma, obteremos algo que se concentra acima de $m + a$.
Finalmente, o preço pago para que as variáveis $X_i$ se comportem como as $X'_i$ será aproximadamente $\exp\{-\psi_{X_1}(m + a)\}$, como desejado para nossa cota inferior.

\begin{proof}
  Primeiramente, consideraremos o caso $P[X_1 \leq m + a] = 1$, que se assemelha ao caso que analizamos acima $(\Ber(1/2) \leq 1)$.
  Nesse caso, temos
  \begin{equation*}
    \begin{split}
      P\big[ X_1 + \dots + X_n \geq \big(m + a \big) n \big] & = P[X_i = m + a, \text{ para todo $i \leq n$}]\\
      & = P[X_1 = m + a]^n.
    \end{split}
  \end{equation*}
  Donde o limite acima é igual a $\log(P[X_1 = m + a])$.
  Mas por outro lado,
  \begin{equation*}
    \begin{split}
      - \psi_{X_1}(m + a) & = \inf_{s \geq 0} \big\{ \log\big(E(\ex{s (X_1)})\big) - (m + a)s \big\} = \inf_{s \geq 0} \big\{ \log\big(E(\ex{s (X_1 - m - a)})\big) \big\}\\
      & \leq \liminf_{s \to \infty} \; \log\big(E(\ex{s (X_1 - m - a)})\big) = \log \big(P[X_1 = m + a]\big),
    \end{split}
  \end{equation*}
  pelo Teorema da Convergência Dominada, demonstrando o teorema nesse caso especial.

  Suponhamos agora que $P[X_1 > m + a] > 0$, o que implica que para $b > m + a$ suficientemente próximo de $m + a$, temos $P[X_1 > b] > 0$.
  Observe que basta mostrar que para todo $b > a$ satisfazendo $P[X_1 > b] > 0$ e para todo $\delta > 0$, temos
  \begin{equation}
    \label{e:PGD_perto_b}
    \liminf_n \frac{1}{n} \log \Big(P\Big[\frac{X_1 + \dots + X_n}{n} \in (b - \delta, b + \delta) \Big]\Big) \geq -\psi_{X_1}(b),
  \end{equation}
  pois a função $\psi_{X_1}(x)$ é convexa, portanto contínua.

  Vamos definir uma nova distribuição $\nu$ com derivada de Radon-Nikodim
  \begin{equation}
    \frac{\d \nu}{\d P_{X_1}} = \frac{1}{Z_\sigma} \ex{\sigma x}.
  \end{equation}
  Observamos primeiramente que o valor de $\sigma$ ainda não foi escolhido.
  Além disso após escolhido $\sigma$, teremos que calcular a constante de normalização $Z_{\sigma}$ de forma que $\nu$ seja uma probabilidade.

  Escolheremos $\sigma \geq 0$ como no Lema~\ref{l:smax_PGD}, isto é, tal que $\psi_{X_1}(b) = b\sigma - \log\big( \phi_{X_1}(\sigma) \big)$.
  Isso nos dá imediatamente que $Z_\sigma = E[\ex{\sigma X_1}] = \phi_{X_1}(\sigma)$ 
  por definição.

  Por diferenciabilidade de $\phi_{X_1}$, o máximo deve ser assumido em um ponto de derivada zero para a função $\psi_{X_1}$, ou seja
  \begin{equation}
    b = \frac{\phi_{X_1}'(\sigma)}{\phi_{X_1}(\sigma)} \overset{\text{Prop.~\ref{p:propried_phi}}}= \frac{E(X \ex{\sigma X})}{E(\ex{\sigma X})} = \frac{E(X \ex{\sigma X})}{Z_\sigma} = \int x \nu(\d x).
  \end{equation}
  Isso implica que se uma variável aleatória tem distribuição $\nu$, sua esperança é $b$.
  É possível verificar que uma tal variável aleatória $X'$ satisfaz obrigatoriamente $\phi_{X'}(s) < \infty$ para todo $s \geq 0$, donde $X' \in \mathcal{L}^p$ para todo $p > 1$.

  Como prometido, consideramos variáveis $X_1', X_2', \dots$ \iid com distribuição $\nu$.
  Pela lei fraca dos grandes números, para qualquer $\delta > 0$,
  \begin{equation}
    \lim_n P\Big[ \frac{X_1' + \dots + X_n'}{n} \in (b-\delta,b+\delta) \Big] = 1.
  \end{equation}

  Finalmente vamos relacionar essa probabilidade à probabilidade definida em termos de $X_i$, na qual estamos interessados.
  \begin{equation*}
    \begin{split}
      P\Big[ & \frac{X_1 + \dots + X_n}{n} \in (b-\delta, b+\delta) \Big] = \int_{x_i; \big| \tfrac{1}{n} \sum_{i \leq n} x_i - b\big| < \delta} \;\; \bigotimes_{i=1}^n (X_1 \circ P)(\d x_i)\\
      & = Z_\sigma^n \int_{x_i; \big| \tfrac{1}{n} \sum_{i \leq n} x_i - b \big| < \delta} \;\; \ex{-\sigma \textstyle{\sum_{i=1}^n x_i}} \bigotimes_{i=1}^n \nu(\d x_i)\\[2mm]
      & \geq Z_\sigma^n \exp\{-(b + \delta) \sigma n\} P\Big[ \frac{X_1' + \dots + X_n'}{n} \in (b-\delta,b+\delta) \Big].
    \end{split}
  \end{equation*}
  Tomando o logarítmo, dividindo por $n$ e tomando o liminf quando $n$ vai a infinito, recuperamos
  \begin{equation}
    \begin{split}
      \lim_n \frac{1}{n} \log \Big(P\Big[ & \frac{X_1 + \dots + X_n}{n} \in (b - \delta,b +  \delta) \Big] \Big) \geq \log(Z_\sigma) - (b + \delta) \sigma\\
      & = \log(\phi_{X_1}(\sigma)) - (b + \delta) \sigma = -\psi_{X_1}(\sigma) - \delta \sigma.
    \end{split}
  \end{equation}
  Como isso vale para todo $\delta > 0$, provamos \eqref{e:PGD_perto_b} o que conclui a prova do teorema.
\end{proof}

\begin{exercise}
  Mostre o Teorema~\ref{t:PGDgeq} no caso em que $\phi_{X_1}(s) < \infty$, para todo $s \in (-\delta, \delta)$.
\end{exercise}

\newpage
