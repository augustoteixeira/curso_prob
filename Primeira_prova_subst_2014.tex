\documentclass{article}
\usepackage{amssymb,amsmath}
\usepackage{graphicx}
\usepackage{enumerate}
\usepackage{amssymb,amsmath,amsthm}
\usepackage{graphicx}
\usepackage{tikz}
\usepackage{pgfplots}
\usepackage[utf8]{inputenc}
\usepackage[T1]{fontenc}
\usepackage{enumerate}
\usepackage{color}
\usepackage{mathabx}
\usepackage{calc}
\usepackage{fullpage}

\DeclareMathOperator{\Cov}{Cov}
\DeclareMathOperator{\Var}{Var}
\DeclareMathOperator{\Poisson}{Poisson}
\DeclareMathOperator{\Ber}{Ber}
\DeclareMathOperator{\Geo}{Geo}
\DeclareMathOperator{\Bin}{Bin}
\DeclareMathOperator{\Exp}{Exp}
\DeclareMathOperator{\VT}{VT}


\usepackage{dsfont} % includes bb 1
\newcommand*\1{\mathds{1}}
\usepackage[brazil]{babel}

\newtheorem{theorem}{Teorema}[section]
\newtheorem{corollary}[theorem]{Corolário}
\newtheorem{lemma}[theorem]{Lema}
\newtheorem{proposition}[theorem]{Proposição}
\newtheorem{definition}[theorem]{Definição}
\newtheorem{notation}[theorem]{Notação}


\def\d{\mathrm{d}}


\begin{document}

\title{Probabilidade I}
\author{Primeira prova}

\maketitle

\noindent Observa\c{c}\~oes:
\begin{itemize}
\item A prova ter\'a dura\c{c}\~ao de $ 4 $ horas.
\item Quest\~oes dever\~ao ser respondidas de maneira ma\-te\-ma\-ti\-ca\-men\-te rigorosa e clara.
\item Todos os resultados provados em aula (exceto exerc\'icios) poder\~ao ser utilizados, salvo quando a quest\~ao pedir que algum destes seja provado. Nesse caso, todos os resultados anteriores poder\~ao ser utilizados.
\item Caso uma quest\~ao seja composta de v\'arios itens, cada um poder\'a ser resolvido independentemente, supondo a validade dos anteriores.
\end{itemize}

\vspace{4mm}
\noindent 1) Sejam $X$ e $Y$ variáveis aleatórias independentes com distribuição $\Exp(1)$, calcule a distribuição de
\begin{enumerate}[\quad a)]
\item $\min\{X,Y\}$ e
\item $X + Y$.
\end{enumerate}
\vspace{4mm}

\noindent 2) Sejam $X_1, \dots, X_n$ variáveis aleatórias absolutamente contínuas com relação à medida de Lebesgue, com densidades $f_{X_i}: \mathbb{R} \to \mathbb{R}_+$.
Mostre que são equivalentes
\begin{enumerate}[\quad a)]
\item $X_1, \dots, X_n$ são independentes e
\item a distribuição do vetor $(X_1, \dots, X_n)$ é absolutatmente contínua com relação à medida de Lebesgue em $\mathbb{R}^n$ e sua densidade satisfaz
  \begin{equation}
    f_{(X_1, \dots, X_n)}(x_1, \dots, x_n) = f_{X_1} (x_1) \dots f_{X_n}(x_n).
  \end{equation}
\end{enumerate}
\medskip

\newpage

\noindent 3) Sejam $X_1, \dots, X_n$ e $Y_1, \dots, Y_n$ variáveis independentes com distribuição $\Ber(p)$.
Defina agora $Z_{i,j} = X_i Y_j$, para $i, j \in \{1, \dots, n\}$ e
\begin{enumerate}[\quad a)]
\item calcule a esperança de $S_n = \tfrac{1}{n^2} \sum_{i=1}^n \sum_{j=1}^n Z_{i,j}$ e
\item estime $P[|S_n - E(S_n)| > a]$ usando o método do segundo momento. Como esse resultado se compara com o caso em que os $Z_{i,j}$ são i.i.d.?
\end{enumerate}

\vspace{4mm}

\noindent 4) Sejam $X_1, X_2, \dots$ variáveis aleatórias i.i.d. e defina o primeiro tempo de récorde como
\begin{equation}
  R = \inf\{i \geq 2; X_i \geq X_1\}.
\end{equation}
Supondo que $X_1$ é absolutamente contínua com respeito à medida de Lebesgue, encontre $E(R)$.
\medskip





\end{document}







