\documentclass{article}
\usepackage{amssymb,amsmath}
\usepackage{graphicx}
\usepackage{enumerate}
\usepackage{amssymb,amsmath,amsthm}
\usepackage{graphicx}
\usepackage{tikz}
\usepackage{pgfplots}
\usepackage[utf8]{inputenc}
\usepackage[T1]{fontenc}
\usepackage{enumerate}
\usepackage{color}
\usepackage{mathabx}
\usepackage{calc}
\usepackage{fullpage}

\usepackage{dsfont} % includes bb 1
\newcommand*\1{\mathds{1}}
\usepackage[brazil]{babel}

\newtheorem{theorem}{Teorema}[section]
\newtheorem{corollary}[theorem]{Corolário}
\newtheorem{lemma}[theorem]{Lema}
\newtheorem{proposition}[theorem]{Proposição}
\newtheorem{definition}[theorem]{Definição}
\newtheorem{notation}[theorem]{Notação}


\def\d{\mathrm{d}}


\begin{document}

\title{Probabilidade I}
\author{Segunda prova}

\maketitle

\noindent Observa\c{c}\~oes:
\begin{itemize}
\item A prova ter\'a dura\c{c}\~ao de $ 4 $ horas.
\item Quest\~oes dever\~ao ser respondidas de maneira ma\-te\-ma\-ti\-ca\-men\-te rigorosa e clara.
\item Todos os resultados provados em aula (exceto exerc\'icios) poder\~ao ser utilizados, salvo quando a quest\~ao pedir que algum destes seja provado. Nesse caso, todos os resultados anteriores poder\~ao ser utilizados.
\item Caso uma quest\~ao seja composta de v\'arios itens, cada um poder\'a ser resolvido independentemente, supondo a validade dos anteriores.
\end{itemize}

\vspace{4mm}
\noindent 1) Considere em $(\mathbb{R}, \mathcal{B}(\mathbb{R}))$ as projeções canônicas $X_1$ e $X_2$.
Calcule, em cada um dos exemplos abaixo, a probabilidade condicional regular $P[X_1 \in \cdot|X_2 = x_2]$, justificando sua resposta,
\begin{enumerate}
\item Quando $P$ é a medida uniforme em $T = \{(x,y) \in [0,1]^2; x \leq y\}$ (ou seja, a medida de Lebesgue em $\mathbb{R}^2$ restrita a $T$ e normalizada para ser uma probabilidade).
\item Quando $P$ é a medida $U_{S^1}$ (uniforme em $S^1$).

  
\end{enumerate}



\vspace{4mm}

\medskip


\noindent 2) Sejam $Z_1, Z_2, \dots$ variáveis aleatórias i.i.d. em $\mathcal{L}^1(P)$ com $E(Z_1) = 0$.
\begin{enumerate}[\quad a)]
\item Defina $X_0 = 0$ e
  \begin{equation}
    X_n = \sum_{i = 1}^n Z_i, \text{ para $n \geq 1$.}
  \end{equation}
  Mostre que $E(X_{n + 1} | Z_1, \dots, Z_n) = X_n$.
\item Supondo agora que $Z_1 \in \mathcal{L}^2(P)$ e $E(Z) = 0$, defina $Y_0 = 0$ e
  \begin{equation}
    Y_n = \Big( \sum_{i = 1}^n Z_i \Big)^2 - n E(Z_1^2)
  \end{equation}
  Mostre que $E(Y_{n + 1} | Z_1, \dots, Z_n) = Y_n$.
\end{enumerate}

\newpage

\noindent 3) Considere as medidas
\begin{equation}
  \mu_a = \frac{\delta_{-1} + \delta_1}{2}, \qquad \text{e} \qquad \mu_b = \mathcal{N}(0, 1).
\end{equation}
e $K:\mathbb{R} \times \mathcal{B}(\mathbb{R}) \to [0,1]$ dada por
  \begin{equation}
    K(x, A) =
    \begin{cases}
      \mu_a (A - x), & \text{ se $x < 0$,}\\
      \mu_b (A - x), & \text{ se $x \geq 0$,}
    \end{cases}
  \end{equation}
Mostre que
\begin{enumerate}[\quad a)]
\item $K$ define um núcleo de transição entre $\mathbb{R}$ em $\mathbb{R}$.
\item Se $X_1, X_2, \dots$ for uma cadeia de Markov em $\mathbb{R}$ com núcleo de transição $K$, então calcule
  \begin{enumerate}[\qquad i)]
  \item $E(X_i)$, para todo $i \geq 1$ e
  \item $\text{Var}(X_i)$, para todo $i \geq 1$.
%    \begin{equation}
%      \frac{\sum_{i = 1}^n X_i}{\sqrt{n}} \Rightarrow \mathcal{N}(0,1).
%    \end{equation}
  \end{enumerate}
\end{enumerate}

\newpage

\noindent 4) {\bf Extra -} Sejam $X_1, X_2, \dots$ i.i.d. distribuidas como $\text{Exp}(1)$ e defina
\begin{equation}
  M_n = \max_{i = 1, \dots, n} X_i.
\end{equation}
Mostre que $M_n - \log(n)$ converge fracamente e identifique o limite.
Observe que não precisamos dividir $M_n - \log(n)$ por nada para obter a convergência.

\medskip
\medskip
\medskip

\noindent 4) {\bf Extra -} Sejam $X_1, X_2, \dots$ i.i.d. distribuidas como $\text{Exp}(1)$ e defina
\begin{equation}
  M_n = \max_{i = 1, \dots, n} X_i.
\end{equation}
Mostre que $M_n - \log(n)$ converge fracamente e identifique o limite.
Observe que não precisamos dividir $M_n - \log(n)$ por nada para obter a convergência.

\medskip
\medskip
\medskip

\noindent 4) {\bf Extra -} Sejam $X_1, X_2, \dots$ i.i.d. distribuidas como $\text{Exp}(1)$ e defina
\begin{equation}
  M_n = \max_{i = 1, \dots, n} X_i.
\end{equation}
Mostre que $M_n - \log(n)$ converge fracamente e identifique o limite.
Observe que não precisamos dividir $M_n - \log(n)$ por nada para obter a convergência.

\medskip
\medskip
\medskip

\noindent 4) {\bf Extra -} Sejam $X_1, X_2, \dots$ i.i.d. distribuidas como $\text{Exp}(1)$ e defina
\begin{equation}
  M_n = \max_{i = 1, \dots, n} X_i.
\end{equation}
Mostre que $M_n - \log(n)$ converge fracamente e identifique o limite.
Observe que não precisamos dividir $M_n - \log(n)$ por nada para obter a convergência.

\medskip
\medskip
\medskip


\end{document}







