% !TEX encoding = UTF-8 Unicode
\documentclass{article}
\usepackage{amssymb,amsmath}
\usepackage{graphicx}
\usepackage{enumerate}
\usepackage{amssymb,amsmath,amsthm}
\usepackage{graphicx}
\usepackage{tikz}
\usepackage{pgfplots}
\usepackage[utf8]{inputenc}
\usepackage[brazil]{babel}
\usepackage{enumerate}
\usepackage{color}
\usepackage{mathabx}
\usepackage{calc}
\usepackage{fullpage}
\newcommand{\dd}{\,\text{\rm d}}    
\renewcommand{\tilde}{\widetilde}
\renewcommand{\hat}{\widehat}
\newcommand{\tf}{\textsc{f}}
\newcommand{\cc}{\complement}
\newcommand{\cf}{\mathrm{f}}
\newcommand{\lint}{\llbracket}
\newcommand{\rint}{\rrbracket}
%%%%%%%%%%%% Calligraphic letters 

 
\newcommand{\cA}{\ensuremath{\mathcal A}} 
\newcommand{\cB}{\ensuremath{\mathcal B}} 
\newcommand{\cC}{\ensuremath{\mathcal C}} 
\newcommand{\cD}{\ensuremath{\mathcal D}} 
\newcommand{\cE}{\ensuremath{\mathcal E}} 
\newcommand{\cF}{\ensuremath{\mathcal F}} 
\newcommand{\cG}{\ensuremath{\mathcal G}} 
\newcommand{\cH}{\ensuremath{\mathcal H}} 
\newcommand{\cI}{\ensuremath{\mathcal I}} 
\newcommand{\cJ}{\ensuremath{\mathcal J}} 
\newcommand{\cK}{\ensuremath{\mathcal K}} 
\newcommand{\cL}{\ensuremath{\mathcal L}} 
\newcommand{\cM}{\ensuremath{\mathcal M}} 
\newcommand{\cN}{\ensuremath{\mathcal N}} 
\newcommand{\cO}{\ensuremath{\mathcal O}} 
\newcommand{\cP}{\ensuremath{\mathcal P}} 
\newcommand{\cQ}{\ensuremath{\mathcal Q}} 
\newcommand{\cR}{\ensuremath{\mathcal R}} 
\newcommand{\cS}{\ensuremath{\mathcal S}} 
\newcommand{\cT}{\ensuremath{\mathcal T}} 
\newcommand{\cU}{\ensuremath{\mathcal U}} 
\newcommand{\cV}{\ensuremath{\mathcal V}} 
\newcommand{\cW}{\ensuremath{\mathcal W}} 
\newcommand{\cX}{\ensuremath{\mathcal X}} 
\newcommand{\cY}{\ensuremath{\mathcal Y}} 
\newcommand{\cZ}{\ensuremath{\mathcal Z}} 
 

%%%%%%%%%%%%%%% Blackboard bolds 
 
\newcommand{\bbA}{{\ensuremath{\mathbb A}} } 
\newcommand{\bbB}{{\ensuremath{\mathbb B}} } 
\newcommand{\bbC}{{\ensuremath{\mathbb C}} } 
\newcommand{\bbD}{{\ensuremath{\mathbb D}} } 
\newcommand{\bbE}{{\ensuremath{\mathbb E}} } 
\newcommand{\bbF}{{\ensuremath{\mathbb F}} } 
\newcommand{\bbG}{{\ensuremath{\mathbb G}} } 
\newcommand{\bbH}{{\ensuremath{\mathbb H}} } 
\newcommand{\bbI}{{\ensuremath{\mathbb I}} } 
\newcommand{\bbJ}{{\ensuremath{\mathbb J}} } 
\newcommand{\bbK}{{\ensuremath{\mathbb K}} } 
\newcommand{\bbL}{{\ensuremath{\mathbb L}} } 
\newcommand{\bbM}{{\ensuremath{\mathbb M}} } 
\newcommand{\bbN}{{\ensuremath{\mathbb N}} } 
\newcommand{\bbO}{{\ensuremath{\mathbb O}} } 
\newcommand{\bbP}{{\ensuremath{\mathbb P}} } 
\newcommand{\bbQ}{{\ensuremath{\mathbb Q}} } 
\newcommand{\bbR}{{\ensuremath{\mathbb R}} } 
\newcommand{\bbS}{{\ensuremath{\mathbb S}} } 
\newcommand{\bbT}{{\ensuremath{\mathbb T}} } 
\newcommand{\bbU}{{\ensuremath{\mathbb U}} } 
\newcommand{\bbV}{{\ensuremath{\mathbb V}} } 
\newcommand{\bbW}{{\ensuremath{\mathbb W}} } 
\newcommand{\bbX}{{\ensuremath{\mathbb X}} } 
\newcommand{\bbY}{{\ensuremath{\mathbb Y}} } 
\newcommand{\bbZ}{{\ensuremath{\mathbb Z}} } 


\newcommand{\eps}{\varepsilon}

\newcommand{\vol}[1]{\operatorname{vol}\left(#1\right)}
\newcommand{\reach}[0]{\operatorname{reach}}
\newcommand{\what}{$\clubsuit\otimes\clubsuit$ }

\newcommand{\remarktitle}{\par\noindent{\bf Remark.}}

%shortcuts
\newcommand{\gap}{\mathrm{gap}}
\newcommand{\sobolev}{c_\mathrm{S}}
\newcommand{\compl}{\mathrm{c}}
\newcommand{\Min}{\mathrm{min}}\newcommand{\Max}{\mathrm{max}}
\newcommand{\rel}{\mathrm{rel}}
\newcommand{\tmix}{T_\mathrm{mix}}
\newcommand{\trel}{T_\mathrm{rel}}
\renewcommand{\O}{\Omega}
\renewcommand{\L}{\Lambda}
\newcommand{\mix}{\mathrm{mix}}
\newcommand{\ent}{\mathrm{Ent}}
\newcommand{\cov}{\mathrm{Cov}}

\newcommand{\ml}{\mathrm{\ell}}
\newcommand{\mm}{\mathrm{m}}
\newcommand{\mr}{\mathrm{r}}
%\newcommand{\1}[1]{{\text{\Large $\mathfrak 1$}\left(#1\right)}}
\newcommand{\1}[1]{{\mathbf 1}\left(#1\right)}

%\newcommand{\veci}{\mathbf{i}}
\newcommand{\si}{\sigma}
\renewcommand{\t}{\tau}
\renewcommand{\l}{\lambda}
\renewcommand{\a}{\alpha}

\newcommand{\grad}{\nabla} 
\newcommand{\tigrad}{\tilde\nabla}

\renewcommand{\star}{*}

\renewcommand{\u}{\pi} 
\def\thsp{\thinspace}

\newcommand{\var}{{\rm Var} } 

  
  \def\tc{\thsp | \thsp}
\newcommand{\ga}{\alpha}
\newcommand{\gb}{\beta}
\newcommand{\gga}{\gamma}            % \gg already exists...
\newcommand{\gd}{\delta}
\newcommand{\gep}{\varepsilon}       % \ge already exists...
\newcommand{\gp}{\varphi}
\newcommand{\gr}{\rho}
\newcommand{\gvr}{\varrho}
\newcommand{\gz}{\xi}
\newcommand{\gG}{\Gamma}
\newcommand{\gP}{\Phi}
\newcommand{\gD}{\Delta}
\newcommand{\gk}{\kappa}
% \newcommand{\gK}{\Kappa}
\newcommand{\go}{\omega}
\newcommand{\gto}{{\tilde\omega}}
\newcommand{\gO}{\Omega}
\newcommand{\gl}{\lambda}
\newcommand{\gL}{\Lambda}
\newcommand{\gs}{\sigma}
\newcommand{\gS}{\Sigma}
\newcommand{\gt}{\vartheta}
\newcommand{\gtt}{{\tilde\tau}}
\newcommand{\bA}{{\ensuremath{\mathbf A}} }
\newcommand{\bF}{{\ensuremath{\mathbf F}} }
\newcommand{\bP}{{\ensuremath{\mathbf P}} }
\newcommand{\bE}{{\ensuremath{\mathbf E}} }
\newcommand{\gI}{{\ensuremath{\mathcal I}} }
\newcommand{\bH}{{\ensuremath{\mathbf H}} }
\newcommand{\bC}{{\ensuremath{\mathbf C}} }
\newcommand{\bK}{{\ensuremath{\mathbf K}} }
\newcommand{\bN}{{\ensuremath{\mathbf N}} }
\newcommand{\bL}{{\ensuremath{\mathbf L}} }
\newcommand{\bT}{{\ensuremath{\mathbf T}} }
\newcommand{\bD}{{\ensuremath{\mathbf D}} }
\newcommand{\bQ}{{\ensuremath{\mathbf Q}} }
\newcommand{\tR}{{\ensuremath{\tilde R}} }
\newcommand{\tS}{{\ensuremath{\tilde S}} }
\newcommand{\ttau}{{\ensuremath{\tilde\tau}} }
\newcommand{\bbbP}{\mathbb{P}}
\newcommand{\E}{\mathbb{E}}
%\newcommand{\bbR}{\mathbb{R}}
\newcommand{\Z}{\mathbb{Z}}
\newcommand{\N}{\mathbb{N}}
\newcommand{\ind}{\mathbf{1}}
\newcommand{\tx}{\tilde{x}}
\newcommand{\ty}{\tilde{y}}
\newcommand{\Tm}{T_{\text{mix}}}
\newcommand{\Va}{\text{Var}}
\newcommand{\tbE}{\tilde{\mathbb{E}}}
\newcommand{\tbP}{\tilde{\mathbb{P}}}
\newcommand{\tZ}{\tilde{Z}}
\newcommand{\tK}{\tilde{K}}
\newcommand{\hbE}{\hat{\mathbb{E}}}
\newcommand{\hbP}{\hat{\mathbb{P}}}
\newcommand{\hZ}{\hat{Z}}
\newcommand{\hK}{\hat{K}}


\usepackage{dsfont} % includes bb 1

\usepackage[brazil]{babel}

\begin{document}

\title{Probabilidade I}
\author{Primeira Prova}

\maketitle

\noindent Observa\c{c}\~oes:
\begin{itemize}
\item A prova tera duraçao de $3$ horas.
\item Questoes deverao ser respondidas de maneira matematicamente rigorosa e clara.
\item Todos os resultados provados em aula (exceto exercicios) poderao ser utilizados, salvo quando a questao pedir que algum destes seja provado. Nesse caso, todos os resultados anteriores poderao ser utilizados.
\item Caso uma questao seja composta de varios itens, cada um podera ser resolvido independentemente, supondo a validade dos anteriores.
\item Uma função $f$ de $\bbR$ é crescente se $x<y \Rightarrow f(x)\le f(y)$.
\item Estudar a recíproca significa: dar a prova da recíproca se for verdade ou  dar um contro-exemple se for falsa.
\end{itemize}


\section*{Exercício I {\small (6 Pts)}}

\begin{enumerate}
\item (1 Pt) Dar a definição de um $\gl$-sistema é a de um $\pi$ sistema, e dar o enunciado do teorema de Dynkin.
\item (2 Pts) Para $\gO:=\{1,2,3,4\}$. Exibir um subcojunto $\cA \subset \cP(\gO)$ que é um $\lambda$-sistema mas não é uma $\sigma$-algebra.
\item (1 Pt) Mostrar que para todo $\gO$ com cardinalidade maior de que $4$ existe um $\lambda$-sistema que não é uma $\sigma$-algebra.
\item (2 pts) Para $\#\gO\le 3$, mostrar que todo $\gl$-sistema é uma $\sigma$-algebra.
\end{enumerate}




\section*{Exercício II {\small (5,5 Pts)}}

\begin{enumerate}

\item (1 Pt) Sejam $X$ e $Y$ duas variáveis aleatórias  tal que $X\ge Y$ quase certamente. Mostrar que as funções acumulada de distribuição
$F_X$ e $F_Y$ satisfazem 
$$\forall x \in \bbR, \quad F_Y(x)\le F_X(x).$$ 
\item (1 Pt) Sejam $X$ uma variável aleatória com função acumulada $F$ e $g$ uma função crescente. Dar a função acumulada de distribuição da variável
$g(X)$ em função de $F$ e $g$.
\item (1+1 Pts) Sejam $\mu$ e $\nu$ duas probabilidade em $\bbR$ que satisfazem, para toda $g$ limitada, continua e crescente 
$$ \int g(\go) \mu(\dd \go)\ge \int g(\go) \nu(\dd \go).$$
Mostrar que as funções acumuladas de distribuição de distribuição associadas à $\mu$ e $\nu$ satisfazem 
$$\forall x \in \bbR, \quad F_{\mu}(x)\le F_{\nu}(x).$$
Estudar a recíproca.

\item (1,5 pts) Sejam $F_1$ e $F_2$ duas funções càd-làg crescentes que satisfazem $F_1(x)\le F_2(x)$ para todo $x\in \bbR$.
Mostrar que e possível construir, no mesmo espaço de probabilidade, duas variáveis aleatórias $X_1$ e $X_2$ que tem funções acumuladas de distribuição
respetivas $F_1$ e $F_2$ e tal que $X_1\ge X_2$ quase certamente.

 
\end{enumerate}



\section*{Exercício III {\small (4 Pts)}}


\begin{enumerate}

\item (1 Pt) Dar a definição de elementos aleatorios independentes.
\item (0,5+1 Pts) Sejam $X:\  (\gO,\cF) \to (E_1,\cA_1)$ e $Y:\  (\gO,\cF) \to (E_2,\cA_2)$ 
 dois elementos aleatórios, e $f:\ E_1 \to \bbR $ e $g:\ E_2 \to \bbR$ funções mensuraveis. 
 Mostrar que se $X$ e $Y$ forem independentes, as variáveis 
$f(X)$ e $g(Y)$ são independentes. Estudar a recíproca.
\item (1,5 Pts) Sejam $\mu$ uma probabilidade em $\bbR$, $f$ e $g$ duas funções crescentes limitadas. Mostrar que 
$$ \int f(\go)g(\go) \mu(\dd \go)\ge  \left(\int f(\go) \mu(\dd \go)\right)\left(\int g(\go) \mu(\dd \go)\right).$$
\textit{DICA: Considerar $E[(f(X)-f(Y))(g(X)-g(Y))]$ para $X$ e $Y$ variáveis com distribuição $\mu$.}

\end{enumerate}

\section*{Exercício IV {\small(6,5 Pts)}}
Seja $f:\ [0,1]\to \bbR$ uma função $1$-Lipshitz, $[0,1]\to \bbR$. isso e que satisfaz
$$\forall x,y \in  [0,1], \ |f(x)-f(y)|\le |x-y|.$$
Seja $(X_i)_{i\ge 1}$ uma sequencia de variáveis de Bernouilli de parametro $p$. Escrevemos 
$S_n:=\sum_{i=1}^n X_i$.

\begin{enumerate}
\item (1,5 Pts) Dar a distribuição da varíavel $S_n$. Calcular a esperança e a variança de $S_n/n$.
\item (0,5 Pt) Dar uma cota superior para $P\left(|\frac{S_n}{n}-p|>\gep\right)$ valida para todo $\gep>0$ 
(incluir a prova).
\item (2,5 Pts) Mostra que existe uma constante $C$ tal que para todo $n$ e toda $f$ $1$-Lipshitz
$$E\left(|f(S_n/n)-f(p)|\right)\le C n^{-1/3}.$$
Deduzir cotas inferior e superior para  $E[f(S_n/n)].$
\item (1 Pt) O teorema de Weirstrass fala que toda função continua $[0,1]\to \bbR$ e o limite uniforme de polinomios.
Deduzir das questões anteriores uma prova probabilistica do Teorema de Weierstrass no caso de funções Lipshitz.
\item (1 Pt) Provar o Teorema de Weierstrass no caso geral, adaptando o método acima.
 
\end{enumerate}






\end{document}

