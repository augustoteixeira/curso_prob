\documentclass{article}
\usepackage{amssymb,amsmath}
\usepackage{graphicx}
\usepackage{enumerate}
\usepackage{amssymb,amsmath,amsthm}
\usepackage{graphicx}
\usepackage{tikz}
\usepackage{pgfplots}
\usepackage[utf8]{inputenc}
\usepackage[brazil]{babel}
\usepackage{enumerate}
\usepackage{color}
\usepackage{mathabx}
\usepackage{calc}
\usepackage{fullpage}

\usepackage{dsfont} % includes bb 1
\newcommand*\1{\mathds{1}}
\usepackage[brazil]{babel}

\newtheorem{theorem}{Teorema}[section]
\newtheorem{corollary}[theorem]{Corolário}
\newtheorem{lemma}[theorem]{Lema}
\newtheorem{proposition}[theorem]{Proposição}
\newtheorem{definition}[theorem]{Definição}
\newtheorem{notation}[theorem]{Notação}

\DeclareMathOperator{\Var}{Var}
\def\d{\mathrm{d}}

\begin{document}

\title{Probabilidade I}
\author{Segunda Prova}

\maketitle

\noindent Observa\c{c}\~oes:
\begin{itemize}
\item A prova terá duração de $3$ horas e meia.
\item Questões deverão ser respondidas de maneira matematicamente rigorosa e clara.
\item Todos os resultados provados em aula (exceto exercícios) poderão ser utilizados, salvo quando a questão pedir que algum destes seja provado. Nesse caso, todos os resultados anteriores poderão ser utilizados.
\item Caso uma questão seja composta de vários itens, cada um poderá ser resolvido independentemente, supondo a validade dos anteriores.
\end{itemize}

\vspace{4mm}
\noindent {\bf 1)} Para cada uma das afirmações abaixo, mostre ou dê um contra-exemplo:
\begin{enumerate}[\quad a)]
\item Se $X$ é uma variável aleatória com $P[X > 0] > 0$, então existe $\delta > 0$ tal que $P[X > \delta] > 0$.
\item No espaço $([0, 1], \mathcal{B}([0, 1]))$, dotado da medida de Lebesgue, os eventos $A = (1/2, 3/4)$ e $B = (0, 2/3)$ são independentes.
\item Sejam $X_1, X_2, \dots$ variáveis aleatórias em $(\Omega, \mathcal{F}, P)$ com esperança nula. Se $\lim_{n \to \infty} \Var(X_n) = 0$, então $X_n$ converge para zero em probabilidade.
\item Sejam $X_1, X_2, \dots$ variáveis aleatórias em $(\Omega, \mathcal{F}, P)$ com esperança nula. Se $\lim_{n \to \infty} \Var(X_n) = 0$, então $X_n$ converge para zero quase certamente.
\end{enumerate}

\vspace{4mm}
\noindent {\bf 2)} Definimos o conjunto de Cantor como
\begin{equation*}
  K = \big\{ x \in [0, 1]; \text{ $x$ possui uma decomposição na base $3$ que não utiliza o dígito $1$} \big\}.
\end{equation*}
Definimos a distribuição $\mu$ uniforme em $K$ como a lei de uma variável aleatória $X \in [0, 1]$ tal que sua decomposição $X = 0,X_1 X_2 X_2 \dots$ na base $3$, possui dígitos $(X_i)_{i \geq 1}$ i.i.d. e com lei dada por
\begin{equation*}
  P[X_1 = 0] = P[X_1 = 2] = \frac{1}{2}.
\end{equation*}
Nesse caso, mostre que
\begin{enumerate}[\quad a)]
\item $\mu$ é singular com respeito à distribuição de Lebesgue em $[0, 1]$.
\item Calcule $E(X)$.
\item Calcule $\Var(X)$.
\end{enumerate}

\vspace{4mm}
\noindent {\bf 3)} Sejam $X_1, X_2, \dots$ variáveis aleatórias i.i.d. com $E(X) = m$ e $E(X^2) = a$.
Considere também uma variável aleatória $N$ tomando valores em $\{1, 2, \dots\}$ com $E(N) = n$ e $E(N^2) = b$ e tal que $N$ seja independente de $(X_i)_{i \geq 1}$.
Calcule a esperança e a variância de
\begin{equation*}
  S = \sum_{i = 1}^N X_i
\end{equation*}
em termos de $m$, $a$, $n$ e $b$.

\vspace{4mm}
\noindent {\bf 4)} Sejam variáveis aleatórias $X_1, X_2, \dots$ i.i.d. com distribuição dada por
\begin{equation*}
  P[X_1 = -1] = \frac{3}{4}, \quad P[X_1 = 1] = \frac{1}{4}.
\end{equation*}
Considere também as somas
\begin{equation*}
  S_0 = 0, \quad \text{e} \quad S_n = \sum_{i = 1}^n X_i,
\end{equation*}
assim como a variável aleatória $N$ definida por
\begin{equation*}
  N = \inf \Big\{ n \geq 0; S_n \leq -1 \Big\}.
\end{equation*}
Mostre que
\begin{enumerate}[\quad a)]
\item $N$ está bem definida (no sentido que o ínfimo acima é tomado sobre um conjunto quase certamente não vazio).
\item $\limsup_n \tfrac{1}{n} \log \big( P[N > n] \big) < 0$.
\item $(\star)$ $\lim_n \tfrac{1}{n} \log \big( P[N > n] \big) = a$ e obtenha o valor de $a < 0$.
\end{enumerate}

\end{document}
