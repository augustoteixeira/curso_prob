\documentclass{article}
\usepackage{amssymb,amsmath}
\usepackage{graphicx}
\usepackage{enumerate}
\usepackage{amssymb,amsmath,amsthm}
\usepackage{graphicx}
\usepackage{tikz}
\usepackage{pgfplots}
\usepackage[utf8]{inputenc}
\usepackage[brazil]{babel}
\usepackage{enumerate}
\usepackage{color}
\usepackage{mathabx}
\usepackage{calc}
\usepackage{fullpage}

\usepackage{dsfont} % includes bb 1
\newcommand*\1{\mathds{1}}
\usepackage[brazil]{babel}

\newtheorem{theorem}{Teorema}[section]
\newtheorem{corollary}[theorem]{Corolário}
\newtheorem{lemma}[theorem]{Lema}
\newtheorem{proposition}[theorem]{Proposição}
\newtheorem{definition}[theorem]{Definição}
\newtheorem{notation}[theorem]{Notação}

\def\d{\mathrm{d}}

\begin{document}

\title{Probabilidade I}
\author{Primeira Prova}

\maketitle

\noindent Observa\c{c}\~oes:
\begin{itemize}
\item A prova terá duração de $4$ horas.
\item Questões deverão ser respondidas de maneira matematicamente rigorosa e clara.
\item Todos os resultados provados em aula (exceto exercícios) poderão ser utilizados, salvo quando a questão pedir que algum destes seja provado. Nesse caso, todos os resultados anteriores poderão ser utilizados.
\item Caso uma questão seja composta de vários itens, cada um poderá ser resolvido independentemente, supondo a validade dos anteriores.
\end{itemize}

\vspace{4mm}
\noindent 1) (4 points) Fixed $\lambda > 0$, let $N$ be a random variable with Poisson($\lambda$) distribution, that is
\begin{equation}
  P[N = k] = \frac{\lambda^k e^{-\lambda}}{k!} \text{ for $k = 0, 1, \dots$}
\end{equation}
Consider in the same probability space a sequence of random variables $X_1, X_2, \dots$ which are i.i.d, with distribution Ber($1/2$) and independent of $N$.
\begin{enumerate}[\quad a)]
\item Calculate the distribution of $Z = \sum_{i=1}^N X_i$.
\item Show that $Z$ and $N - Z$ are independent.
\end{enumerate}

\vspace{4mm}
\noindent 2) (3 points) For $a < b \in \mathbb{R}$, we define the probability $U_{[a,b]}$ on $([0,1], \mathcal{B}([0,1]))$ by the following formula $U_{[a,b]}(B) = \mathcal{L}(B \cap [a,b])/(b-a)$.
Also, let $K:[0,1] \times \mathcal{B}([0,1]) \to [0,1]$ be given by $K(x, \cdot) = U_{[0,x]} (\cdot)$.
\begin{enumerate}[\quad a)]
\item Show that $K$ is a transition kernel.
\item Calculate $U_{[0,1]} \star K [X_1 < 1/2]$, and $U_{[0,1]} \star K [X_2 < 1/2]$, where $X_1$ and $X_2$ are the two canonical projections on $[0,1]^2$.
\item Prove that $U_{[0,1]} \star K$ is absolutely continuous with respect to the Lebesgue measure on $[0,1]^2$ and calculate its density.
\end{enumerate}

\newpage

\vspace{4mm}

\noindent 3) (3 points) Let $Y_k$ be independent random variables with the following distribution
\begin{equation}
  P[Y_k = i] =
  \begin{cases}
    \frac 12 - \frac 1{k^2} \quad & \text{if $i = 1$ or $i = -1$},\\
    \frac 2{k^2} & \text{if $i = 3$.}
  \end{cases}
\end{equation}
Show that
\begin{equation}
  P\Big[ \frac 1n \sum_{k=1}^n Y_k \text{ converges to zero} \Big] = 1.
\end{equation}

%\vspace{4mm}
%\noindent 3) (3 points) Consider the graph $G = (\mathbb{Z}^2, E)$, where $E = \big\{ \{x,y\}; |x - y|_2 = 1 \big\}$.
%We endow the measurable space $\{0,1\}^E$ with the $\sigma$-algebra $\mathcal{A}$ generated by the canonical projections $Y_e(\omega) = \omega(e)$, where $\omega \in \{0,1\}^E$ and $e \in E$.
%Let us define the set $A \subseteq \{0,1\}^E$ by
%\begin{equation}
%  A = \Big[
%  \begin{array}{c}
%    \text{there exists a sequence of distinct $x_0, x_1, \dots \in \mathbb{Z}^2$,}\\
%    \text{such that $e_i = \{x_i, x_{i+1}\} \in E$ and $Y_{e_i} = 1$ for every $i \geq 0$}
%  \end{array}
%  \Big].
%\end{equation}
%\begin{enumerate}[\quad a)]
%\item Show that $A$ is measurable with respect to $\mathcal{A}$.
%\item Assuming this, show that $A$ is a tail event. Meaning that
%  \begin{equation}
%    A \in \bigcap_{K \subseteq E; \text{ finite}} \sigma\big( Y_e; e \not \in K \big).
%  \end{equation}
%\end{enumerate}

\vspace{4mm}

\noindent 4) (Extra) Let $\Omega = E^\mathbb{Z}$ be an infinite product space, endowed with $\sigma$-algebra $\mathcal{A}$ generated by the canonical projections $(X_i)_{i \in \mathbb{Z}}$.
Let us consider on $(\Omega, \mathcal{A})$ the product measure $\mathbb{P} = P^{\otimes \mathbb{Z}}$, where $P$ is a fixed probability on the Polish space $E$.
\begin{enumerate}[\quad a)]
\item Show that for any event $A \in \mathcal{A}$ and any $\varepsilon > 0$, there exists a $k \in \mathbb{Z}_+$ and an event $A_k \in \sigma(X_i, |i| \leq k)$ such that $\mathbb{P}[(A \setminus A_k) \cup (A_k \setminus A)] < \varepsilon$.
\item Consider the shift map $\theta:\Omega \to \Omega$ given by $\theta(\omega)(i) = \omega(i-1)$ and show that if $A = \theta(A)$, then $P(A) \in \{0,1\}$.
\end{enumerate}

\end{document}
