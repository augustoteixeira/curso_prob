\documentclass{article}
\usepackage{amssymb,amsmath}
\usepackage{graphicx}
\usepackage{enumerate}

\def\d{\mathrm{d}}


\begin{document}

\title{Probabilidade I}
\author{Primeira prova}

\maketitle

\noindent Observa\c{c}\~oes:
\begin{itemize}
\item A prova ter\'a dura\c{c}\~ao de $ 4 $ horas.
\item Quest\~oes dever\~ao ser respondidas de maneira ma\-te\-ma\-ti\-ca\-men\-te rigorosa e clara.
\item Todos os resultados provados em aula (exceto exerc\'icios) poder\~ao ser utilizados, salvo quando a quest\~ao pedir que algum destes seja provado. Nesse caso, todos os resultados anteriores poder\~ao ser utilizados.
\item Caso uma quest\~ao seja composta de v\'arios itens, cada um poder\'a ser resolvido independentemente, supondo a validade dos anteriores.
\end{itemize}

\vspace{4mm}
\noindent 1) Sejam $X, Y$ vari\'aveis aleat\'orias tais que
\begin{equation}
  P[X \leq x, Y \leq y] =
  \begin{cases}
    0 & \quad \text{if $x < 0$,}\\
    (1-e^{-x}) \Big(\frac 12 + \frac 1\pi \tan^{-1} y \Big), & \quad \text{if $x \geq 0$}.
  \end{cases}
\end{equation}
\begin{enumerate}[\quad a)]
  \item Mostre que a distribui\c{c}\~ao conjunta $\mu_{(X,Y)}$ \'e
  absolutamente cont\'inua com rela\c{c}\~ao \`a medida de Lebesgue em
  $\mathbb{R}^2$.
  \item Mostre que $X$ e $Y$ s\~ao independentes.
\end{enumerate}
\medskip

\noindent 2) Seja $K:\mathbb{R}_+ \times \mathbb{B}(\mathbb{R}_+) \to [0,1]$ dada por $K(x,A) = \int_A x \exp\{-x t\} \d t$.
\begin{enumerate}[\quad a)]
  \item Prove que $K$ \'e um n\'ucleo de transi\c{c}\~ao.
  \item Seja $P$ dada por $P = K \star \textnormal{Exp}(1)$. Obtenha $P[X_2 > x_2]$ para todo $x_2 \geq 0$ (lembrando que $X_2$ denota a segunda coordenada no espa\c{c}o produto onde est\'a definida $P$). Compare a probabilidade acima com $K(1,[x_2, \infty))$.
  \item Mostre que $P[X_1 + X_2 \geq z] = \int_0^z \exp \{-x(z-x+1)\} \d x + \exp\{-z\}$.
\end{enumerate}

\medskip

\pagebreak

\noindent 3) Marcelo coleciona figurinhas de futebol. O \'album completo conter\'a $N$ figurinhas. No $i$-\'esimo dia, ele compra uma nova carta $X_i \in \{1, \dots, N\}$. A cole\c{c}\~ao $(X_i)_{i \geq 0}$ \'e distribuida de maneira i.i.d. e uniforme nas figurinhas.
\begin{enumerate}[\quad a)]
  \item Para $j = 1, \dots, N$, seja $T_j$ o tempo passado at\'e a aquisi\c{c}\~ao da $j$-\'esima nova figurinha, i.e.
    \begin{equation}
      T_1 = 1 \quad \text{ e } \quad T_j = \inf\{i, X_i \not \in \{X_{T_{j'}}; j' < j\}\}.
    \end{equation}
  Mostre que $T_j$ \'e finito quase certamente, para todo $j \leq N$.
  \item Calcule a distribui\c{c}\~ao conjunta de $(T_1, T_2 - T_1, \dots, T_N - T_{N-1})$.
  \item Calcule a esperan\c{c}a de $T_N$ (o dia em que Marcelo completa seu \'album).
\end{enumerate}

\noindent 4) Sejam $(X_i)_{i \geq 1}$ vari\'aveis i.i.d. com distribui\c{c}\~ao Ber$(p)$, $p \in [0,1]$. Mostre que
\begin{equation}
  \lim_{N \to \infty} \frac 1N \sum_{i = 1}^N X_i X_{i+1} = p^2, \text{ em probabilidade.}
\end{equation}
\end{document}







