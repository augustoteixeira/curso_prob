\documentclass{article}
\usepackage{amssymb,amsmath}
\usepackage{graphicx}
\usepackage{enumerate}
\usepackage{amssymb,amsmath,amsthm}
\usepackage{graphicx}
\usepackage{tikz}
\usepackage{pgfplots}
\usepackage[utf8]{inputenc}
\usepackage[T1]{fontenc}
\usepackage{enumerate}
\usepackage{color}
\usepackage{mathabx}
\usepackage{calc}
\usepackage{fullpage}

\usepackage{dsfont} % includes bb 1
\newcommand*\1{\mathds{1}}
\usepackage[brazil]{babel}

\newtheorem{theorem}{Teorema}[section]
\newtheorem{corollary}[theorem]{Corolário}
\newtheorem{lemma}[theorem]{Lema}
\newtheorem{proposition}[theorem]{Proposição}
\newtheorem{definition}[theorem]{Definição}
\newtheorem{notation}[theorem]{Notação}


\def\d{\mathrm{d}}


\begin{document}

\title{Probabilidade I}
\author{Primeira prova}

\maketitle

\noindent Observa\c{c}\~oes:
\begin{itemize}
\item A prova ter\'a dura\c{c}\~ao de $ 4 $ horas.
\item Quest\~oes dever\~ao ser respondidas de maneira ma\-te\-ma\-ti\-ca\-men\-te rigorosa e clara.
\item Todos os resultados provados em aula (exceto exerc\'icios) poder\~ao ser utilizados, salvo quando a quest\~ao pedir que algum destes seja provado. Nesse caso, todos os resultados anteriores poder\~ao ser utilizados.
\item Caso uma quest\~ao seja composta de v\'arios itens, cada um poder\'a ser resolvido independentemente, supondo a validade dos anteriores.
\end{itemize}

\vspace{4mm}
\noindent 1) Sejam $X, Y$ vari\'aveis aleat\'orias em um espaço $(\Omega, \mathcal{F}, P)$, independentes e com distribuição $U_{[0,1]}$.
\begin{enumerate}[\quad a)]
  \item Calcule $(X+Y) \circ P$.
  \item Considere $P'(\cdot) = P\big(\cdot | X + Y \leq 1 \big)$ e calcule $X \circ P'$.
\end{enumerate}

\vspace{4mm}

\noindent 2) Na Nova Caledônia, temos $k$ habitantes.
Seja $f:\{1, \dots, k\} \to \{0,1\}$ uma função que indica a intenção de voto de cada cidadão.
Mais precisamente, para cada habitante $i \in \{1, \dots, k\}$, se $f(i) = 0$, então $i$ vota no candidato $0$, enquanto se $f(i) = 1$, o cidadão $i$ vota no candidato $1$.
Para estimar o número $k_1 = \# f^{-1}(\{1\})$ de pessoas que votam em $1$, nós escolhemos variáveis aleatórias $Y_i$ i.i.d. com distribuição uniforme em $\{1, \dots, k\}$ e queremos estimar
\begin{equation}
  \text{Err}_n(\epsilon) = P \Big[ \Big| \frac{1}{n} \sum_{i=1}^n f(Y_i) - \frac{k_1}{k} \Big| > \epsilon \Big].
\end{equation}
Sabendo que $k$ é par e $k_1 = k/2$, então
\begin{enumerate}[\quad a)]
  \item use o método do segundo momento para obter um $n$ tal que $\text{Err}_{n}(0.01) < 0.02$ e
  \item use o método do momento exponencial para obter um $n$ tal que $\text{Err}_{n}(0.01) < 10^{-12}$.
\end{enumerate}
\medskip

\newpage

\noindent 3) Dizemos que uma probabilidade $P$ no espaço produto $\Omega = \bigtimes_{n \geq 1} E$ (com a $\sigma$-álgebra canônica) é fortemente misturadora se, para todo $k \geq 1$, temos
\begin{equation}
  \lim_{n \to \infty} \sup \Big\{ \big| P(A \cap B) - P(A) P(B) \big| \; ; \;\; A \in \sigma(X_1, \dots, X_k), B \in \sigma(X_n, X_{n+1}, \dots) \Big\} = 0.
\end{equation}
Mostre que nesse caso, a $\sigma$-álgebra dos eventos caudais é trivial.

\vspace{4mm}

\noindent 4) Considere uma rua infinita com casas $i \in \mathbb{Z}$.
Para todo $i \in \mathbb{Z}$, existia uma rua entre as casas $i$ e $i+1$, mas após uma grande tempestade essas ruas foram danificadas.
Mais precisamente, para cada $i \in \mathbb{Z}$, temos variáveis aleatórias $X_i$ que são i.i.d. com distribuição $\text{Ber}(p)$, onde $X_i = 1$ indica que o trecho da rua entre as casas $i$ e $i + 1$ foi danificado e não pode ser utilizado.
Defina, para $i \in \mathbb{Z}$, $R_i$ como sendo o número de casas que continuaram acessíveis à casa $i$ após a tempestade.
Por exemplo, se $X_{-2}$ e $X_0 = 1$ e $X_{-1} = 0$, temos que a casa $0$ somente pode acessar a casa $-1$, logo $R_0 = 1$.
Nesse contexto,
\begin{enumerate}[\quad a)]
  \item Calcule a distribuição e a esperança de $R_0$,
  \item Use o método do segundo momento para estimar a probabilidade
    \begin{equation}
      P \Big[ \Big| \frac{1}{n} \sum_{i=1}^n R_i - E(R_0) \Big| > a \Big].
    \end{equation}
\end{enumerate}
\medskip


\newpage

Solução da quarta questão.

Primeiramente, vamos ver qual é a distribuição de $R_0$.
Vamos escrever $R_0 = E_0 + D_0$, onde $E_0$ é o número de casas acessíveis à esquerda e $D_0$ à direita.
Note que $E_0$ e $D_0$ são independentes e identicamente distribuídas, com
\begin{equation}
  P[D_0 = l] = P[X_l = 1, X_i = 0 \text{ para $i = 0, \dots, l-1$}] = p (1-p)^l.
\end{equation}
Podemos agora calcular
\begin{equation}
  P[R_0 = k] = \sum_{l=0}^k P[D_0 = l, E_0 = k-l] = \sum_{l=0}^k p^2 (1-p)^{k} = p^2 k (1-p)^k.
\end{equation}
Além disso,
\begin{equation}
  E(R_0) = 2 E(D_0) = \sum_{l=0}^\infty l P[D_0 = l] = 2 p \sum_{l=0}^\infty l (1-p)^l = \frac{2(1-p)}{p} =: m.
\end{equation}
O que resolve o primeiro ítem.

O grande problema do segundo ítem é que as variáveis $R_i$ não são independentes, veja por exemplo que $P[R_0=0,R_1=2,R_2=0] = 0$.
Nesse caso, o método do segundo momento deve ser feito com atenção.
Chamando de $S_n = \sum_{i=1}^n R_i$, temos
\begin{equation}
  P \Big[ \Big| \frac{1}{n} S_n - E(R_0) \Big| > a \Big] \leq \frac{\text{Var}(S_n)}{a^2 n^2},
\end{equation}
mas a variância da soma não se torna a soma das variâncias.
De fato
\begin{equation}
  \begin{split}
    \text{Var}(S_n) & = E \Big( \big(\sum_{i=1}^n (R_i - E(R_i)) \big)^2\Big) = \sum_{i=1}^n \sum_{j=1}^n E \Big(\big(R_i - E(R_i)\big) \big(R_j - E(R_j) \big)\Big)\\
    & = \sum_{i=1}^n \sum_{j=1}^n \text{Cov}(R_i, R_j) = n \text{Var}(R_0) + 2 \sum_{k=1}^{n-1} (n-k) \text{Cov}(R_0, R_k).
  \end{split}
\end{equation}
Aqui já temos metade da estimativa resolvida, mas ainda falta obter uma estimativa explícita.

Então precisamos estimar superiormente $\text{Cov}(R_i, R_j) = \text{Cov}(R_0, R_{j-1})$.
Podemos calcular essa quantidade explicitamente, mas vamos evitar contas chatas fazendo uma estimativa do tipo
\begin{equation}
  \text{Cov}(R_0, R_k) \leq c \exp\{-c' k\}, \text{ para todo $k \geq 1$}.
\end{equation}
O que nos daria que
\begin{equation}
  \text{Var}(S_n) \leq n \text{Var}(R_0) + 2 \sum_{k=1}^{n-1} (n-k) c \exp\{-c' k\} \leq c'' n.
\end{equation}
Donde a probabilidade que queríamos estimar é no máximo ${c}/{a^2 n}$, como no caso independente.

Para obter a prometida cota para a covariância, observe que podemos truncar $D_0$ e $E_k$ para obter independência.
Definindo
\begin{equation}
  \tilde{R_0} = E_0 + ( D_0 \wedge \lfloor k/2 \rfloor ) \text{ e } \tilde{R}_k = D_k + ( E_k \wedge \lfloor k/2 \rfloor ),
\end{equation}
temos que $\tilde{R}_0$ e $\tilde{R}_k$ são independentes (pois dependem de elos disjuntos).
Donde
\begin{equation}
  \begin{split}
    \text{Cov}(R_0, R_k) & = E(R_0 R_k) - m^2 = E(\tilde{R}_0 \tilde{R_k}) + E(R_0 R_k \1{[R_0 \neq \tilde{R}_0] \cup [R_k \neq \tilde{R}_k]}) - m^2\\
    & \leq E(\tilde{R}_0)^2 - m^2 + E\big( (E_0 + D_0) (E_k + D_k) \1{[R_0 \neq \tilde{R}_0] \cup [R_k \neq \tilde{R}_k]}\big)\\
    & \leq E\big( (E_0 + k + D_k)^2 \1{[R_0 \neq \tilde{R}_0] \cup [R_k \neq \tilde{R}_k]}\big)\\
    & = E\big( (E_0 + k + D_k)^2 \big) P\big([R_0 \neq \tilde{R}_0] \cup[R_k \neq \tilde{R}_k]\big)\\
    & \leq \big( 2 E(E_0^2) + k^2 + 2k E(E_0) + E(E_0)^2 \big) \cdot 2 \cdot P[R_0 \neq \tilde{R}_0]\\
    & \leq c k^2 (1-p)^{\lfloor k/2 \rfloor} \leq c \exp \{-c' k\}.
  \end{split}
\end{equation}
Finalizando a cota para a covariância.



\end{document}







